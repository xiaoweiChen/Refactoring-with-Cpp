在设置 CI 管道时,通常只需检查代码是否符合既定的格式规则,而不是自动修改源文件。这样可以确保标记不符合样式指南的代码,从而提示开发人员手动修复。 Clang-Format 使用 \verb|--dry-run| 和 \verb|--Werror| 选项支持此用例,如果文件重新格式化,这两个选项组合使用时,会导致工具以非零状态代码退出。

可以扩展现有的 CMake 设置,使其包含一个仅检查代码格式的新自定义目标。操作方法如下:

\begin{cmake}
# Create custom target to check clang-format
if(CLANG_FORMAT_EXECUTABLE)
    add_custom_target(
        check-clang-format
        COMMAND ${CLANG_FORMAT_EXECUTABLE} -style=file -Werror --dryrun ${ALL_SOURCE_FILES}
        COMMENT "Checking code format with clang-format"
    )
else()
    message("clang-format not found! Target 'check-clang-format' will not be available.")
endif()
\end{cmake}

此扩展设置中,添加了一个名为 check-clang-format 的新自定义目标。 \verb|--dry-run| 选项可确保不会修改任何文件,而 \verb|-Werror| 则会导致 Clang-Format 在发现格式差异时退出并显示错误代码。此目标可以使用 \verb|make check-clang-format| 或 \verb|cmake --build . --target check-clang-format| 运行。

现在,可以在 CI 管道脚本中调用此自定义目标,来强制执行代码样式检查。如果代码的格式不符合指定的准则,则构建将失败,并提醒团队存在需要解决的格式问题。

例如,在 .clang-format 文件中,将缩进宽度设置为四个空格,但是 main.cpp 文件只使用了两个:

\begin{cpp}
int main() {
  return 0;
}
\end{cpp}

运行检查器时,就会显示有问题的代码,而不会进行任何修改:

\begin{shell}
make check-clang-format

-- Configuring done
-- Generating done
-- Build files have been written to: /home/user/clang-format-tidy/build
[100%] Checking code format with clang-format
/home/user/clang-format-tidy/main.cpp:2:13: error: code should be clangformatted [-Wclang-format-violations]
int main() {
            ^
make[3]: *** [CMakeFiles/check-clang-format.dir/build.make:71: CMakeFiles/check-clang-format] Error 1
make[2]: *** [CMakeFiles/Makefile2:137: CMakeFiles/check-clang-format.dir/all] Error 2
make[1]: *** [CMakeFiles/Makefile2:144: CMakeFiles/check-clang-format.dir/rule] Error 2
make: *** [Makefile:150: check-clang-format] Error 2
\end{shell}

通过将此自定义目标添加到 CMake 设置中,可以为项目添加质量保证。确保违反既定格式准则的代码都不会进入代码库,这在协作环境中尤其有用。






















