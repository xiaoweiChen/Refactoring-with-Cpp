让我们准备一个简单的示例来演示 Clang-Format 工具的实际操作。创建一个名为 main.cpp 的基本 C++ 源文件,其中包含一些格式问题。然后,将在此文件上运行 Clang-Format 以自动更正格式,并生成所做更改的报告:

\begin{cpp}
#include <iostream>

class Sender {
public:
    void send(const std::string& message) {
        std::cout << "Sending: " << message << std::endl;
    }
};

class Receiver {
public:
    void receive(const std::string& message) {
        std::cout << "Receiving: " << message << std::endl;
    }
};

class Mediator {
public:
    Mediator(Sender sender, Receiver receiver)
        : sender _ {std::move(sender)}, receiver _ {std::move(receiver)} {}
    void send(const std::string& message) {
        sender _ .send(message);
    }

    void receive(const std::string& message) {
        receiver _ .receive(message);
    }

private:
    Sender sender _ ;
    Receiver receiver _ ;
};
\end{cpp}

我们将尝试使用 Clang-Format 工具和之前在 .clang-format 中定义的规则集对其进行分析:

\begin{shell}
make check-clang-format
[100%] Checking code format with clang-format
/home/user/clang-format/clang _ format.cpp:4:2: error: code should be
clang-formatted [-Wclang-format-violations]
{
 ^
/home/user/clang-format/clang _ format.cpp:6:42: error: code should be
clang-formatted [-Wclang-format-violations]
    void send(const std::string& message){
                                       ^
/home/user/clang-format/clang _ format.cpp:7:18: error: code should be
clang-formatted [-Wclang-format-violations]
        std::cout<< "Sending: " <<message<< std::endl;
                  ^
/home/user/clang-format/clang _ format.cpp:7:35: error: code should be
clang-formatted [-Wclang-format-violations]
        std::cout<< "Sending: " <<message<< std::endl;
                                  ^
/home/user/clang-format/clang _ format.cpp:7:42: error: code should be
clang-formatted [-Wclang-format-violations]
        std::cout<< "Sending: " <<message<< std::endl;
                                         ^
/home/user/clang-format/clang _ format.cpp:11:6: error: code should be
clang-formatted [-Wclang-format-violations]
class
     ^
/home/user/clang-format/clang _ format.cpp:12:9: error: code should be
clang-formatted [-Wclang-format-violations]
Receiver {
        ^
/home/user/clang-format/clang _ format.cpp:12:11: error: code should be
clang-formatted [-Wclang-format-violations]
Receiver {
          ^
/home/user/clang-format/clang _ format.cpp:14:36: error: code should be
clang-formatted [-Wclang-format-violations]
    void receive(const std::string&message){
                                   ^
/home/user/clang-format/clang _ format.cpp:14:44: error: code should be
clang-formatted [-Wclang-format-violations]
    void receive(const std::string&message){
                                            ^
/home/user/clang-format/clang _ format.cpp:14:45: error: code should be
clang-formatted [-Wclang-format-violations]
    void receive(const std::string&message){
                                             ^
/home/user/clang-format/clang _ format.cpp:16:6: error: code should be
clang-formatted [-Wclang-format-violations]
    }};
^
/home/user/clang-format/clang _ format.cpp:18:15: error: code should be
clang-formatted [-Wclang-format-violations]
class Mediator{
      ^
/home/user/clang-format/clang _ format.cpp:18:16: error: code should be
clang-formatted [-Wclang-format-violations]
class Mediator{
              ^
/home/user/clang-format/clang _ format.cpp:20:28: error: code should be
clang-formatted [-Wclang-format-violations]
    Mediator(Sender sender,Receiver receiver)
                           ^
/home/user/clang-format/clang _ format.cpp:21:69: error: code should be
clang-formatted [-Wclang-format-violations]
        : sender _ {std::move(sender)}, receiver _ {std::move(receiver)} {}
                                                                          ^
/home/user/clang-format/clang _ format.cpp:21:71: error: code should be
clang-formatted [-Wclang-format-violations]
        : sender _ {std::move(sender)}, receiver _ {std::move(receiver)} {}
^
/home/user/clang-format/clang _ format.cpp:22:44: error: code should be
clang-formatted [-Wclang-format-violations]
    void send(const std::string& message) {sender _ .send(message);}
                                           ^
/home/user/clang-format/clang _ format.cpp:22:66: error: code should be
clang-formatted [-Wclang-format-violations]
    void send(const std::string& message) {sender _ .send(message);}
                                                                     ^
/home/user/clang-format/clang _ format.cpp:24:47: error: code should be
clang-formatted [-Wclang-format-violations]
    void receive(const std::string& message) {
                                               ^
/home/user/clang-format/clang _ format.cpp:25:36: error: code should be
clang-formatted [-Wclang-format-violations]
    receiver _ .receive(message);
                                  ^
/home/user/clang-format/clang _ format.cpp:26:6: error: code should be
clang-formatted [-Wclang-format-violations]
    }
     ^
/home/user/clang-format/clang _ format.cpp:28:11: error: code should be
clang-formatted [-Wclang-format-violations]
    Sender sender _ ;
           ^
make[3]: *** [CMakeFiles/check-clang-format.dir/build.make:71: CMakeFiles/check-clang-format] Error 1
make[2]: *** [CMakeFiles/Makefile2:139: CMakeFiles/check-clang-format.dir/all] Error 2
make[1]: *** [CMakeFiles/Makefile2:146: CMakeFiles/check-clang-format.dir/rule] Error 2
make: *** [Makefile:150: check-clang-format] Error 2
\end{shell}

如您所见,错误不是很具描述性。但大多数时候,开发人员可以了解代码出了什么问题。该工具不仅能够检测问题,还能修复问题。让运行该工具来修复格式问题 make clang-format 并查看结果:

\begin{cpp}
#include <iostream>

class Sender
{
public:
    void send(const std::string& message)
    {
        std::cout << "Sending: " << message << std::endl;
    }
};

class Receiver
{
public:
    void receive(const std::string& message)
    {
        std::cout << "Receiving: " << message << std::endl;
    }
};

class Mediator
{
public:
    Mediator(Sender sender, Receiver receiver)
    : sender _ {std::move(sender)}, receiver _ {std::move(receiver)}
    {
    }
    void send(const std::string& message) { sender _ .send(message); }

    void receive(const std::string& message) { receiver _ .
        receive(message); }

private:
    Sender sender _ ;
    Receiver receiver _ ;
};
\end{cpp}

代码现在已正确格式化,可以在项目中使用。此示例可以包含在您的项目中,以展示 Clang-Format 在真实场景中的实际应用。将来,开发人员可以向 .clang-format 文件添加更多格式化规则,并通过运行 make clang-format 命令重新格式化整个项目。



