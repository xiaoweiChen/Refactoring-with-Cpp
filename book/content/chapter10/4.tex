
随着深入研究静态分析领域, Clang-Tidy 是一款因其多功能性和深度而脱颖而出的工具。 Clang-Tidy 由 LLVM 基金会(Clang 编译器背后的组织)开发,是一款专为 C++ 代码设计的 linter 和静态分析工具。它超越了传统编译器的检查功能,提供一系列诊断,包括风格错误、编程错误,甚至是常规代码审查中经常遗漏的细微错误。之前,探讨了 Clang-Tidy 如何擅长代码格式化;现在,将探索其在静态分析方面的实力,揭示其在确保合规性和卓越编码标准的水平上审查 C++ 代码的能力。

Clang-Tidy 的工作原理是使用 Clang 前端解析 C++ 代码,使其能够深入了解代码的结构和语法。这种深入的理解使 Clang-Tidy 能够执行复杂的检查,而不仅仅是文本分析,检查代码的语义甚至执行流程。它不仅仅是为了发现语法差异,还为了理解代码的行为和意图。

\mySubsubsection{10.4.1}{Clang-Tidy 中的检查类别}

Clang-Tidy 将其检查分为几类,每类针对特定类型的问题。

让我们分解这些类别并探索每个类别的示例:

\begin{itemize}
\item
性能检查:重点识别代码中可能减慢执行速度的低效模式;例如,不必要的对象复制。Clang-Tidy 可以标记对象复制,但可以通过引用移动或传递的情况,以避免复制的开销:

\begin{cpp}
#include <vector>
std::vector<int> createLargeVector();
void processVector(std::vector<int> vec);

int main() {
    std::vector<int> vec = createLargeVector();
    processVector(vec); // Clang-Tidy: Use std::move to avoid copying
    return 0;
}
\end{cpp}

\item
现代化检查:旨在将代码升级到现代 C++ 标准,例如 C++11 及更高版本;例如,用基于范围的 for 循环替换传统的 for 循环,以提高可读性和安全性:

\begin{cpp}
std::vector<int> myVec = {1, 2, 3};
for (std::size_t i = 0; i < myVec.size(); ++i) {
    // Clang-Tidy: Use a range-based for loop instead
    std::cout << myVec[i] << std::endl;
}
\end{cpp}

\item
错误检测:识别代码中的潜在错误或逻辑错误;例如,检测空指针取消引用:

\begin{cpp}
int* ptr = nullptr;
int value = *ptr; // Clang-Tidy: Dereference of null pointer
\end{cpp}

\item
样式检查:强制执行特定的编码样式以确保一致性和可读性;例如,强制执行变量命名约定:

\begin{cpp}
int MyVariable = 42; // Clang-Tidy: Variable name should be lower_case
\end{cpp}

\item
可读性检查:专注于使代码更易于理解和维护;例如,简化复杂的布尔表达式:

\begin{cpp}
bool a, b, c;
if (a && (b || c)) {
    // Clang-Tidy: Simplify logical expression
}
\end{cpp}

\item
安全检查:针对潜在的安全漏洞;例如,突出显示已知会带来安全风险的危险功能的使用:

\begin{cpp}
strcpy(dest, src); // Clang-Tidy: Use of function 'strcpy' is insecure
\end{cpp}

\end{itemize}

\mySubsubsection{10.4.2}{通过自定义检查扩展 Clang-Tidy 的功能}

Clang-Tidy 的多功能性通过对自定义检查的支持而得到进一步增强,允许公司和项目根据其特定需求和编码标准定制静态分析。这种定制功能导致创建了各种类别的检查,每种检查都符合不同组织或项目的指导方针。接下来,我们探讨一些例子:

\begin{itemize}
\item
Google 检查 (google-*):这些检查强制执行 Google C++ 样式指南中指定的准则。包括有关命名约定、运行时引用和构建问题的规则。例如, google-runtime-references 强制Google 优先使用指针而不是非常量引用。

\item
Google 的 Abseil 检查: Abseil 是 Google 开发的 C++ 库代码的开源集合。针对 Abseil 的检查可确保遵守库的最佳实践,例如避免使用某些已弃用的函数或类。

\item
Fuchsia 检查:这些检查专为 Fuchsia 操作系统量身定制,可强制执行特定于 Fuchsia 项目的编码标准和最佳实践。有助于维护为该操作系统做出贡献的代码库的一致性和质量。

\item
Zircon 检查: Zircon 是支持 Fuchsia OS 的核心平台。 Clang-Tidy 包含针对 Zircon 开发量身定制的检查,重点关注其独特的架构和开发标准。

\item
Darwin 检查:这些检查专为 Apple 发布的开源类 Unix 操作系统 Darwin 设计。确保符合 Darwin 的开发实践。

\item
LLVM 检查 (llvm-*):这些检查旨在强制执行 LLVM 编码标准,对于为 LLVM 或其子项目做出贡献的开发人员特别有用。

\item
C++ 核心指南检查: Clang-Tidy 包含强制执行 C++ 核心指南的检查,这是编写现代 C++ 的一套最佳实践。这包括类型安全、资源管理和性能规则。

\item
类型安全检查 (cppcoreguidelines-*):这些检查可确保各种情况下的类型安全。例如, cppcoreguidelines-pro-type-member-init 可确保类成员正确初始化。 cppcoreguidelines-pro-type-reinterpret-cast 警告不要使用 reinterpret\_cast,鼓励使用更安全的强制转换替代方案。

\item
接口指南:这些检查强制执行设计干净且易于管理的接口的规则。例如, cppcoreguidelines-non-private-member-variables-in-classes 不鼓励使用非私有成员变量来维护封装。

\item
并发指南: cppcoreguidelines-avoid-magic-numbers 等检查有助于识别可能没有明显含义的硬编码数字,从而提高可读性和可维护性。

\item
性能增强检查:这些包括 cppcoreguidelines-avoid-c-arrays 和 cppcoreguidelines-avoid-nonconst-global-variables 等检查,这些检查提倡使用现代 C++ 构造(例如 std::array 或 std::vector)而非 C 风格数组,并阻止使用非常量全局变量。

\item
边界安全检查:确保边界安全在 C++ 中至关重要,而 cppcoreguidelines-pro-bounds-array-to-pointer-decay 和 cppcoreguidelines-pro-bounds-constant-array-index 等检查可以警告可能导致越界 (OOB) 错误的常见陷阱。

\item
所有权和智能指针检查:这些检查强调正确使用智能指针来管理资源所有权。 cppcoreguidelines-owning-memory 指导开发人员何时以及如何使用智能指针,例如 std::unique\_ptr 或 std::shared\_ptr。

\item
五规则和零规则检查: Clang-Tidy 在 C++ 类设计中强制执行五规则和零规则,确保管理资源的类正确实现复制和移动构造函数/赋值运算符或避免手动管理资源。

\item
杂项检查:其他杂项检查包括 cppcoreguidelines-special-member-functions(确保特殊成员函数的正确实现)和 cppcoreguidelines-interfaces-global-init(避免全局初始化顺序问题)。

通过 Clang-Tidy 检查遵守 C++ 核心指南可以显著提高 C++ 代码的质量,使其更加稳健、更易于维护,并与现代 C++ 实践保持一致。这些检查涵盖了广泛的最佳实践,通常认为是大多数 C++ 项目(尤其是那些旨在有效利用现代 C++ 功能的项目)的良好实践。

\item
检查软件包是否符合标准: Clang-Tidy 提供检查“包”,帮助确保符合某些高级标准:

\begin{itemize}
\item
高性能 C++ (hi-cpp):这些检查重点确保代码针对性能进行了优化。

\item
认证:对于需要遵守特定认证标准(如 MISRA、 CERT 等)的项目, Clang-Tidy 提供有助于使代码符合这些标准的检查,需要注意的是,仅使用这些检查可能不足以完全符合这些认证。
\end{itemize}
\end{itemize}

添加自定义检查的能力意味着 Clang-Tidy 不仅仅是一个静态分析工具,而是一个可以适应各种编码标准和实践的平台。这种适应性使其成为从开源库到商业软件等项目的理想选择,每个项目都有其独特的要求和标准。通过利用这些专门的检查,团队可以确保他们的代码不仅符合 C++ 中的一般最佳实践,而且还符合其项目或组织的特定准则和细微差别。

\mySubsubsection{10.4.3}{对 Clang-Tidy 进行微调以进行定制静态分析}

有效配置 Clang-Tidy 是充分发挥其在 C++ 项目中潜力的关键。这不仅涉及启用和禁用某些检查,还涉及控制如何分析代码的特定部分。通过自定义其行为,开发人员可以确保 Clang -Tidy 的输出既相关又可操作,并专注于其代码库的重点部分。

\begin{itemize}
\item
启用和禁用检查:使用 \verb|--checks=| 选项启用特定检查,并在前面添加 - 以禁用其他检查。例如,要打开性能检查并关闭特定检查,您可以使用以下命令:

\begin{shell}
clang-tidy my_code.cpp --checks='performance-*, -performancenoexcept-move-constructor'
\end{shell}

\item
忽略特定警告: Clang-Tidy 允许通过两种主要方式抑制警告:

\begin{itemize}
\item
使用 NOLINT 进行一般抑制:可以使用 NOLINT 注释来抑制特定代码行的所有警告。这是一种广泛的方法,可能会隐藏超出预期的内容:

\begin{cpp}
int x = 0; // NOLINT
\end{cpp}

\item
特定抑制:更有针对性的方法是使用 NOLINT(check-name) 来抑制特定警告。这种方法更可取,可以防止过度抑制可能有用的警告:

\begin{cpp}
int x = 0; // NOLINT(bugprone-integer-division)
\end{cpp}

\end{itemize}

\item
将警告视为错误:为了严格执行代码质量,请使用 \verb|--warnings-as-errors=| 选项将警告转换为错误。这可以全局应用或应用于特定检查:

\begin{shell}
clang-tidy my_code.cpp --warnings-as-errors='bugprone-*'
\end{shell}

\item
使用配置文件:为了在整个项目中保持一致和共享的设置,请将 .clang-tidy 文件放在项目的根目录中。此文件应列出已启用的检查和其他配置,如以下示例所示:

\begin{shell}
Checks: 'performance-*, -performance-noexcept-move-constructor' WarningsAsErrors: 'bugprone-*'
\end{shell}
\end{itemize}

正确配置 Clang-Tidy 对于在 C++ 中进行有效的静态分析至关重要。通过有选择地启用检查、必要时特别抑制警告以及将关键警告视为错误,团队可以保持较高的代码质量标准。使用特定的抑制注释逐行微调分析的能力可确保 Clang-Tidy 提供有针对性和相关的反馈,使其成为 C++ 开发人员工具包中不可或缺的工具。














