
管理第三方库是 C++ 开发的一个关键方面。虽然 C++ 没有标准化的包管理器,但已经采用了各种方法和工具来简化此过程,每种方法和工具都有自己的一套实践和支持的平台。

\mySubsubsection{13.2.1}{使用操作系统包管理器安装库}

许多开发人员依赖操作系统的包管理器来安装第三方库。在 Ubuntu 和其他基于 Debian 的系统上,通常使用 apt:

\begin{shell}
sudo apt install libboost-all-dev
\end{shell}

对于基于 Red Hat 的系统, yum 或其后继者 dnf 是首选:

\begin{shell}
sudo yum install boost-devel
\end{shell}

在 macOS 上, Homebrew 是管理软件包的流行选择:

\begin{shell}
brew install boost
\end{shell}

Windows 用户经常使用 Chocolatey 或 vcpkg(后者除了在 Windows 中,还可以作为通用的C++ 库管理器):

\begin{shell}
choco install boost
\end{shell}

这些操作系统包管理器对于通用库来说很方便,但可能并不总是提供开发所需的最新版本或特定配置。

\mySubsubsection{13.2.2}{通过子模块使用 Git 作为第三方管理器}

Git 子模块允许开发人员直接在其存储库中包含和管理第三方库的源代码。此方法有利于确保所有团队成员和构建系统使用库的精确版本。添加子模块并将其与 CMake 集成的典型工作流程可能如下所示:

\begin{shell}
git submodule add https://github.com/google/googletest.git external/googletest
git submodule update --init
\end{shell}

在 CMakeLists.txt 中,您需要包含子模块:

\begin{cmake}
add_subdirectory(external/googletest)
include_directories(${gtest_SOURCE_DIR}/include ${gtest_SOURCE_DIR})
\end{cmake}

此方法将您的项目与库的特定版本紧密结合,并方便通过 Git 跟踪更新。

\mySubsubsection{13.2.3}{使用 CMake FetchContent 下载库}

CMake 的 FetchContent 模块通过在配置时下载依赖项提供了比子模块更灵活的替代方案,而无需将它们直接包含在您的存储库中:

\begin{cmake}
include(FetchContent)
FetchContent_Declare(
    json
    GIT_REPOSITORY https://github.com/nlohmann/json.git
    GIT_TAG v3.7.3
)
FetchContent_MakeAvailable(json)
\end{cmake}

这种方法与 Git 子模块不同,它不需要库的源代码存在于您的存储库中或手动更新它。 FetchContent 动态检索指定的版本,使管理和更新依赖项变得更加容易。











