
CMake 因其强大的脚本功能和跨平台支持而被广泛用于 C++ 项目。将 Conan 与 CMake 集成可以显著简化管理依赖项的过程。以下是实现此集成的方法:

\begin{itemize}
\item
Conan CMake 包装器: Conan 提供了一个可自动执行集成的 CMake 包装器脚本。要使用它,请将 Conan 生成的 conanbuildinfo.cmake 文件包含在项目的 CMakeLists.txt 中:

\begin{cmake}
include(${CMAKE_BINARY_DIR}/conanbuildinfo.cmake)
conan_basic_setup(TARGETS)
\end{cmake}

该脚本设置了必要的包含路径和库路径,并定义了 Conan 管理的依赖项,使其可用于您的 CMake 项目。

\item
使用目标: conan\_basic\_setup() 中的 TARGETS 选项为您的 Conan 依赖项生成 CMake 目标,允许您使用 CMake 中的 target\_link\_libraries() 函数链接它们:

\begin{cmake}
target_link_libraries(my_project_target CONAN_PKG::poco)
\end{cmake}

这种方法提供了一种干净、明确的方式,将项目目标与 Conan 管理的库链接起来。
\end{itemize}

\mySubsubsection{13.4.1}{其他构建系统集成}

Conan 的灵活性也扩展到其他构建系统,使其能够适应各种项目需求:

\begin{itemize}
\item
Makefile:对于使用 Makefile 的项目, Conan 可以为可包含在 Makefile 中的包含路径、库路径和标志生成适当的变量:

\begin{shell}
include conanbuildinfo.mak
\end{shell}

\item
MSBuild:对于使用 MSBuild(在 Windows 环境中很常见)的项目, Conan 可以生成可导入到 Visual Studio 项目中的 .props 文件,从而实现与 MSBuild 生态系统的无缝集成。

\item
Bazel、 Meson 及其他:虽然对某些构建系统(例如 Bazel 或 Meson)的直接支持可能需要自定义集成脚本或工具,但 Conan 社区经常贡献生成器和工具来弥补这些差距,从而将 Conan 的覆盖范围扩展到几乎任何构建系统。
\end{itemize}

\mySubsubsection{13.4.2}{定制集成}

对于没有直接支持的构建系统或具有独特要求的项目, Conan 提供了自定义生成文件甚至编写自定义生成器的功能。这允许开发人员根据其特定的构建过程定制集成,从而使 Conan 成为一种高度适应依赖项管理的工具。

\mySubsubsection{13.4.3}{结论}

Conan 与 CMake 和其他构建系统的集成凸显了其作为 C++ 项目的包管理器的多功能性。通过提供将依赖项合并到各种构建环境中的简单机制, Conan 不仅简化了依赖项管理,还增强了跨不同平台和配置的构建可重复性和一致性。无论您使用的是广泛使用的构建系统(例如 CMake)还是更专业的设置, Conan 灵活的集成选项都能确保您能够保持高效且精简的开发工作流程。


