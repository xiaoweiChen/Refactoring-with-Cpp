Conan 是一款功能强大的 C 和 C++ 软件包管理器,可简化集成第三方库以及跨各种平台和配置管理依赖项的过程。它以处理多个版本的库、复杂的依赖关系图和不同的构建配置的能力而脱颖而出,使其成为现代 C++ 开发的必备工具。

\mySubsubsection{13.3.1}{Conan 配置和功能}

Conan 的配置存储在 conanfile.txt 或 conanfile.py 中,开发人员可以在其中指定所需的库、版本、设置和选项。此文件充当项目依赖项的清单,可实现对项目中使用的库的精确控制。

主要特点:

\begin{itemize}
\item
多平台支持: Conan 设计用于 Windows、 Linux、 macOS 和 FreeBSD,可在不同的操作系统上提供一致的体验

\item
构建配置管理:开发人员可以指定编译器版本、架构和构建类型(调试、发布)等设置,以确保项目的兼容性和最佳构建

\item
版本处理: Conan 可以管理同一个库的多个版本,允许项目根据需要依赖特定版本

\item
依赖项解析:自动解析并下载传递依赖项,确保构建过程中所需的所有库都可用
\end{itemize}

\mySubsubsection{13.3.2}{库位置和Conan Center}

Conan 软件包的主要存储库是 Conan Center,这是一个广泛的开源 C 和 C++ 库集合。 Conan Center 是查找和下载软件包的首选地点,但开发人员也可以为其项目指定自定义或私有存储库。

除了 Conan Center,公司和开发团队还可以托管自己的 Conan 服务器或使用 Artifactory 等服务来管理私有或专有包,从而实现组织内依赖关系管理的集中式方法。

\mySubsubsection{13.3.3}{配置静态或动态链接}

Conan 允许开发人员指定是否对库使用静态或动态链接。这通常通过 conanfile.txt 或 conanfile.py 中的选项来完成。以下是示例:

\begin{shell}
[options]
Poco:shared=True # Use dynamic linking for Poco
Or in conanfile.py:
class MyProject(ConanFile):
    requires = "poco/1.10.1"
    default_options = {"poco:shared": True}
\end{shell}

这些设置指示 Conan 下载,并使用指定库的动态版本。同样,将选项设置为 False 将有利于静态库。必须注意的是,并非所有软件包都支持这两种链接选项,这取决于如何为 Conan 打包。

\mySubsubsection{13.3.4}{使用自定义包扩展 Conan}

Conan 的优势之一是其可扩展性。如果所需的库在 Conan Center 中不可用或不符合特定需求,开发人员可以创建并贡献自己的软件包。 Conan 提供了一个基于 Python 的开发工具包来创建软件包,其中包括用于定义构建过程、依赖项和软件包元数据的工具。

要创建 Conan 软件包,开发人员需要定义 conanfile.py,以描述如何获取、构建和打包库。
此文件包括 Conan 在软件包创建过程中执行的 source()、 build() 和 package() 等方法。
当开发出软件包,就可以通过提交以供纳入的方式在 Conan Center 中进行共享,也可以通过私有存储库进行分发,以保持对分发和使用的控制。

Conan 的灵活性、对多种平台和配置的支持,以及其全面的软件包存储库使其成为 C++ 开发人员的工具。通过利用 Conan,团队可以简化其依赖项管理流程,确保在不同环境中实现一致、可重复的构建。配置静态或动态链接的能力,加上使用自定义软件包扩展存储库的选项,凸显了 Conan 对各种项目需求的适应性。无论是使用广泛使用的开源库还是专有代码, Conan 都提供了一个强大的框架,可以高效、有效地管理 C++ 依赖项。

Conan 是一款专用的 C++ 软件包管理器,擅长管理不同版本的库及其依赖项。其独立于操作系统的软件包管理器运行,并提供高水平的控制和灵活性。典型的 Conan 工作流程包括创建 conanfile.txt 或 conanfile.py 来声明依赖项。



















