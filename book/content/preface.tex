高级语言主导技术领域的时代, C++ 仍然是基石,推动着从嵌入式平台至分布式云原生基础设施的大多数系统。其优势在于能够提供性能敏感的解决方案,同时熟练地处理复杂的数据结构。过去的二十年里, C+ 也经历了重大的演变,以便不断适应现代计算的需求。

本书是一本全面的指南,适合希望掌握编写干净、高效的 C++ 代码技巧的同学。这里会深入探讨 SOLID 原则的实现,以及如何使用 C++ 的最新功能和方法对遗留代码进行重构。读者将深入了解该语言、标准库、Boost 库集合,以及 Microsoft 的指南支持库。

本书从基础知识开始,涵盖了编写干净代码所必需的要素,重点介绍了 C++ 中的面向对象编程。深入介绍了软件测试的设计原则,并通过使用 Google Test 等主流单元测试框架的示例进行了说明。此外,本书还探讨了自动化工具在静态和动态代码分析中的应用,并展示了 Clang Tools 的强大功能。

阅读完本书后,读者将掌握应用业界认可的编码实践的知识和技能,使其能够为实际应用编写干净、可持续且可读的 C++ 代码。

\mySubsectionNoFile{}{适读人群}

本书面向 C++ 社区的各类专业人士。如果您是一名 C++ 工程师,希望提高技能并编写更优雅、更高效的代码,本书将提供提升编程实践所需的见解和技术。对于负责重构和改进现有代码库的人来说,也是一本绝佳的资源,提供实用的建议和策略,使这个过程更易于管理和更有效。

此外,本书对于旨在增强软件开发流程的技术和团队领导者来说,也是一本非常有价值的指南。无论领导一个小团队,还是管理一个更大的开发项目,都会找到有用的技巧和方法,让工作流程更顺畅、更高效。通过实施本书中概述的最佳实践,可以营造一个更高效、更和谐的开发环境,最终带来更高质量的软件和更成功的项目。

\mySubsectionNoFile{}{本书内容}

第 1 章,\textit{C++ 中的编码标准},探讨了干净代码在成功的软件项目中的关键性作用,以及低质量代码如何导致技术债务积累。本章还介绍了代码格式和文档的重要性,强调了它们在维护可管理和有效的代码库方面的作用。我们介绍了 C++ 社区中使用的常见惯例和最佳实践,强调了干净代码和适当文档对于项目的必要性。

第 2 章,\textit{主要软件开发原则},介绍了创建结构良好且可维护的代码的软件设计原则,讨论了 SOLID 原则(单一职责、开放-封闭、里氏替换、接口隔离和依赖倒置),这些原则可帮助开发人员编写易于理解、测试和修改的代码。我们还强调了抽象级别的重要性、副作用和可变性的概念,及其对软件质量的影响。通过应用这些原则,开发人员可以创建更强大、更可靠、更可扩展的软件。

第 3 章,\textit{糟糕代码的原因},指出了导致 C++ 代码质量低下的关键因素。这些因素包括快速交付的压力、 C++ 的灵活性允许对同一问题有多种解决方案、个人编码风格,以及对现代 C++ 功能的缺乏了解。了解这些原因有助于开发人员避免常见的陷阱,并有效改进现有代码库。

第 4 章,\textit{确定重写的理想候选者 - 模式和反模式},重点介绍如何确定 C++ 项目中重构的理想候选。这里将探讨使代码段适合重构的因素,例如:技术债务、复杂性和可读性强弱。我们还将讨论常见的模式和反模式,提供在不改变其行为的情况下改进代码结构、可读性和可维护性的指南和技术。本章旨在让开发人员掌握相关知识,有效提高其 C++ 代码库的质量和稳健性。

第 5 章,\textit{命名的重要性},强调了命名约定在 C++ 编程中的重要作用。变量、函数和类的正确名称可增强代码的可读性和可维护性。我们讨论了命名的最佳实践、不良命名对代码效率的影响,以及一致编码约定的重要性。通过理解和应用这些原则,将能编写更清晰、更有效的代码。

第 6 章,\textit{利用 C++ 中的丰富静态类型系统},探讨了 C++ 中强大的静态类型系统,强调了它在编写健壮、高效且可维护的代码方面的作用。我们讨论了高级技术,例如:使用 <chrono> 库来处理时间持续时间、使用 not\_null 包装器和使用 std::optional 进行更安全的指针处理。此外,还研究了 Boost 等外部库来增强类型安全性。通过实际示例,展示了如何利用这些工具来充分利用 C++ 类型系统的潜力,从而生成更具表现力和鲁棒性的代码。

第 7 章, \textit{C++ 中的类、对象和 OOP},重点介绍 C++ 中的类、对象和面向对象编程 (OOP) 的高级主题。我们涵盖类设计、方法实现、继承和模板使用。关键主题包括优化类封装、高级方法实践、评估继承与组合以及复杂的模板技术。实际示例说明了这些概念,以便创建强大且可扩展的软件架构。

第 8 章,\textit{设计和开发 API},探讨了在 C++ 中设计可维护 API 的原则和实践。我们讨论了 API 设计中清晰度、一致性和可扩展性的重要性。通过具体示例,我们说明了有助于创建直观、易于使用和强大的 API 的最佳实践。通过应用这些原则,将开发出满足用户需求并保持适应性的 API,从而确保软件库的寿命和成功。

第 9 章,\textit{代码格式和命名约定},探讨了代码格式和命名约定在创建强大,且可维护的软件方面的关键作用。虽然这些主题看似微不足道,但极大地提高了代码的可读性、简化了维护并促进了有效的团队协作,尤其是在 C++ 等复杂语言中。我们深入探讨了代码格式的重要性,并提供了使用 Clang-Format 等工具和编辑器特定的插件来实现一致格式的实用知识。本章结束时,将了解这些实践的重要性,以及如何在 C++ 项目中应用它们。

第 10 章,\textit{C++ 中的静态分析},简介讨论了静态分析在确保 C++ 开发中的代码质量和可靠性方面的关键作用。我们讨论了静态分析如何快速且经济高效地识别错误,使其成为软件质量保证的关键组成部分。深入研究了 Clang-Tidy、 PVS-Studio 和 SonarQube 等流行工具,并指导将静态分析集成到开发工作流程中。

第 11 章,\textit{动态分析},探讨了 C++ 中的动态代码分析,重点介绍了在执行期间仔细检查程序行为,以检测内存泄漏、竞争条件和运行时错误等问题的工具。我们介绍了基于编译器的清理器,例如:地址消杀器 (ASan)、线程消杀器 (TSan) 和未定义行为消杀器 (UBSan),以及用于内存调试的 Valgrind。通过了解这些工具,并将其集成到您的开发工作流程中,可以确保 C++ 代码更干净、更高效、更可靠。

第 12 章,\textit{测试},强调了软件测试在确保质量、可靠性和可维护性方面的关键作用。我们介绍了各种测试方法,首先是单元测试以验证各个组件,然后是集成测试以检查集成单元之间的交互。然后进行系统测试以全面评估整个软件系统,最后进行验收测试以确保软件满足最终用户的要求。通过了解这些方法,将了解测试如何支撑强大,且以用户为中心的软件的开发。

第 13 章,\textit{管理第三方的现代方法},讨论了第三方库在 C++ 开发中的重要作用。我们探讨了第三方库管理的基础知识,包括静态编译与动态编译对部署的影响。鉴于 C++ 缺乏标准化的库生态系统,研究了 vcpkg 和 Conan 等工具,以了解它们在集成和管理库方面的优势。此外,还讨论了使用 Docker 创建一致且可重复的开发环境。本章结束时,将了解如何选择和管理第三方库,从而增强开发工作流程和软件质量。

第 14 章,\textit{版本控制},强调了在软件开发中维护干净的提交历史的重要性。我们讨论了清晰且有目的的提交消息的最佳实践,并介绍了 Git、常规提交规范和提交 linting 等工具。通过遵循这些原则,开发人员可以增强沟通、协作和项目可维护性。

第 15 章,\textit{代码审查},探讨了代码审查在确保 C++ 代码稳健且可维护方面的关键作用。虽然自动化工具和方法提供了显著的好处,但它们并非万无一失。由人工审查人员进行的代码审查,有助于发现自动化流程可能遗漏的错误并确保遵守标准。我们讨论了有效代码审查的策略和实用指南,强调了它们在防止错误、提高代码质量,以及培养学习和问责的协作文化方面的作用。

\mySubsectionNoFile{}{环境配置}

为了充分利用本书,应该对 C++ 的基础知识有扎实的了解。熟悉 Make 和 CMake 等构建系统将大有裨益。此外,掌握 Docker 的基本知识和终端技能可以增强学习体验。

\mySubsectionNoFile{}{下载源码}

可以从 GitHub 下载本书的示例代码文件,网址为 \url{https://github.com/PacktPublishing/Refact oring-with-C-}。如果代码有更新,将在 GitHub 库中更新。

还有丰富的书籍和视频目录中的其他代码包,可在 \url{https://github.com/PacktPublishing/} 上找到。快来看看吧!
