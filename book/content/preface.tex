在高级语言主导技术领域的时代, C++ 仍然是基石,推动着从裸机嵌入式平台到分布式云原生基础设施的大量系统。它的优势在于能够提供性能敏感的解决方案,同时熟练地处理复杂的数据结构。在过去的二十年里, C++ 经历了重大的演变,不断适应现代计算的需求。

本书是一本全面的指南,适合那些希望掌握编写干净、高效的 C++ 代码技巧的人。它深入探讨了 SOLID 原则的实现以及使用 C++ 的最新功能和方法对遗留代码进行重构。读者将深入了解该语言、标准库、广泛的 Boost 库集合以及 Microsoft 的指南支持库。

本书从基础知识开始,涵盖了编写干净代码所必需的核心元素,重点介绍了 C++ 中的面向对象编程。它深入了解了软件测试的设计原则,并通过使用 Google Test 等流行单元测试框架的示例进行了说明。此外,本书还探讨了自动化工具在静态和动态代码分析中的应用,突出了 Clang Tools 的强大功能。

在旅程结束时,读者将掌握应用业界认可的编码实践的知识和技能,使他们能够为实际应用编写干净、可持续且可读的 C++ 代码。

\mySubsectionNoFile{}{本书适合哪些人阅读}

本书面向 C++ 社区的各类专业人士。如果您是一名 C++ 工程师,希望提高技能并编写更优雅、更高效的代码,本书将为您提供提升编程实践所需的见解和技术。对于负责重构和改进现有代码库的人来说,它也是一本绝佳的资源,提供实用的建议和策略,使这个过程更易于管理和更有效。

此外,这本书对于旨在增强软件开发流程的技术和团队领导者来说也是一本非常宝贵的指南。无论您是领导一个小团队还是管理一个更大的开发项目,您都会找到有用的技巧和方法,让您的工作流程更顺畅、更高效。通过实施本书中概述的最佳实践,您可以营造一个更高效、更和谐的开发环境,最终带来更高质量的软件和更成功的项目。

\mySubsectionNoFile{}{本书涵盖的内容}

第 1 章"C++ 中的编码标准"探讨了干净代码的世界及其在成功的软件项目中的关键作用。我们讨论了技术债务以及低质量代码如何导致其积累。本章还介绍了代码格式和文档的重要性,强调了它们在维护可管理和有效的代码库方面的作用。我们介绍了 C++ 社区中使用的常见惯例和最佳实践,强调了干净代码和适当文档对于任何项目的必要性。

第 2 章"主要软件开发原则"介绍了创建结构良好且可维护的代码的关键软件设计原则。我们讨论了 SOLID 原则(单一职责、开放-封闭、里氏替换、接口隔离和依赖倒置),这些原则可帮助开发人员编写易于理解、测试和修改的代码。我们还强调了抽象级别的重要性、副作用和可变性的概念及其对软件质量的影响。通过应用这些原则,开发人员可以创建更强大、更可靠、更可扩展的软件。

第 3 章"糟糕代码的原因"指出了导致 C++ 代码质量低下的关键因素。这些因素包括快速交付的压力、 C++ 的灵活性允许对同一问题有多种解决方案、个人编码风格以及对现代 C ++ 功能的缺乏了解。了解这些原因有助于开发人员避免常见的陷阱并有效改进现有代码库。

第 4 章"确定重写的理想候选者 - 模式和反模式"重点介绍如何确定 C++ 项目中重构的理想候选者。我们将探讨使代码段适合重构的因素,例如技术债务、复杂性和可读性差。我们还将讨论常见的模式和反模式,提供在不改变其行为的情况下改进代码结构、可读性和可维护性的指南和技术。本章旨在让开发人员掌握相关知识,以有效提高其 C++ 代码库的质量和稳健性。

第 5 章"命名的重要性"强调了命名约定在 C++ 编程中的重要作用。变量、函数和类的正确名称可增强代码的可读性和可维护性。我们讨论了命名的最佳实践、不良命名对代码效率的影响以及一致编码约定的重要性。通过理解和应用这些原则,您将编写更清晰、更有效的代码。

第 6 章"利用 C++ 中的丰富静态类型系统"探讨了 C++ 中强大的静态类型系统,强调了它在编写健壮、高效且可维护的代码方面的作用。我们讨论了高级技术,例如使用 <chrono> 库来处理时间持续时间、使用 not\_null 包装器和使用 std::optional 进行更安全的指针处理。此外,我们还研究了 Boost 等外部库来增强类型安全性。通过实际示例,我们展示了如何利用这些工具来充分利用 C++ 类型系统的潜力,从而生成更具表现力和抗错误的代码。

第 7 章 C++ 中的类、对象和 OOP 重点介绍 C++ 中的类、对象和面向对象编程 (OOP) 的高级主题。我们涵盖类设计、方法实现、继承和模板使用。关键主题包括优化类封装、高级方法实践、评估继承与组合以及复杂的模板技术。实际示例说明了这些概念,帮助您创建强大且可扩展的软件架构。

第 8 章"在 C++ 中设计和开发 API"探讨了在 C++ 中设计可维护 API 的原则和实践。我们讨论了 API 设计中清晰度、一致性和可扩展性的重要性。通过具体示例,我们说明了有助于创建直观、易于使用和强大的 API 的最佳实践。通过应用这些原则,您将开发出满足用户需求并保持适应性的 API,从而确保您的软件库的寿命和成功。

第 9 章"代码格式和命名约定"探讨了代码格式和命名约定在创建强大且可维护的软件方面的关键作用。虽然这些主题看似微不足道,但它们极大地提高了代码的可读性、简化了维护并促进了有效的团队协作,尤其是在 C++ 等复杂语言中。我们深入探讨了代码格式的重要性,并提供了使用 Clang-Format 等工具和编辑器特定的插件来实现一致格式的实用知识。在本章结束时,您将了解这些实践的重要性以及如何在您的 C++ 项目中有效地应用它们。

第 10 章 C++ 中的静态分析简介讨论了静态分析在确保 C++ 开发中的代码质量和可靠性方面的关键作用。我们讨论了静态分析如何快速且经济高效地识别错误,使其成为软件质量保证的关键组成部分。我们深入研究了 Clang-Tidy、 PVS-Studio 和 SonarQube 等流行工具,并指导将静态分析集成到您的开发工作流程中。

第 11 章"动态分析"探讨了 C++ 中的动态代码分析,重点介绍了在执行期间仔细检查程序行为以检测内存泄漏、竞争条件和运行时错误等问题的工具。我们介绍了基于编译器的清理器,例如地址清理器 (ASan)、线程清理器 (TSan) 和未定义行为清理器 (UBSan),以及用于彻底内存调试的 Valgrind。通过了解这些工具并将其集成到您的开发工作流程中,您可以确保 C++ 代码更干净、更高效、更可靠。

第 12 章"测试"强调了软件测试在确保质量、可靠性和可维护性方面的关键作用。我们介绍了各种测试方法,首先是单元测试以验证各个组件,然后是集成测试以检查集成单元之间的交互。然后我们进行系统测试以全面评估整个软件系统,最后进行验收测试以确保软件满足最终用户的要求。通过了解这些方法,您将掌握测试如何支撑强大且以用户为中心的软件的开发。

第 13 章"管理第三方的现代方法"讨论了第三方库在 C++ 开发中的重要作用。我们探讨了第三方库管理的基础知识,包括静态编译与动态编译对部署的影响。鉴于 C++ 缺乏标准化的库生态系统,我们研究了 vcpkg 和 Conan 等工具,以了解它们在集成和管理库方面的优势。此外,我们还讨论了使用 Docker 创建一致且可重复的开发环境。在本章结束时,您将能够有效地选择和管理第三方库,从而增强您的开发工作流程和软件质量。

第 14 章"版本控制"强调了在软件开发中维护干净的提交历史的重要性。我们讨论了清晰且有目的的提交消息的最佳实践,并介绍了 Git、常规提交规范和提交 linting 等工具。通过遵循这些原则,开发人员可以增强沟通、协作和项目可维护性。

第 15 章"代码审查"探讨了代码审查在确保 C++ 代码稳健且可维护方面的关键作用。虽然自动化工具和方法提供了显著的好处,但它们并非万无一失。由人工审查人员进行的代码审查有助于发现自动化流程可能遗漏的错误并确保遵守标准。我们讨论了有效代码审查的策略和实用指南,强调了它们在防止错误、提高代码质量以及培养学习和问责的协作文化方面的作用。

\mySubsectionNoFile{}{充分利用本书}

为了充分利用本书,您应该对 C++ 的基础知识有扎实的了解。熟悉 Make 和 CMake 等构建系统将大有裨益。此外,掌握 Docker 的基本知识和终端技能可以增强您的学习体验,尽管这些都是可选的。

\mySubsectionNoFile{}{下载示例代码文件}

您可以从 GitHub 下载本书的示例代码文件,网址为 \url{https://github.com/PacktPublishing/Refact oring-with-C-}。如果代码有更新,它将在 GitHub 存储库中更新。

我们还有丰富的书籍和视频目录中的其他代码包,可在 \url{https://github.com/PacktPublishing/} 上找到。快来看看吧!
