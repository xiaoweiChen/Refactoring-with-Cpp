在 C++ 编程领域,遵守一致的编码约定对于确保代码清晰度和可维护性至关重要。在众多可用的样式中,三种突出的约定因其广泛使用和独特的方法而脱颖而出 - Google 的 C++ 样式指南、 LLVM 编码标准和 Mozilla 的编码样式。本概述深入探讨了每个约定的关键方面,重点介绍了它们独特的实践和理念:

\begin{itemize}
\item
Google 的 C++ 风格指南: Google 的指南专为内部使用而设计,但在开源社区中被广泛采用。主要功能包括:
\begin{itemize}
\item
文件名:分别使用 .cc 和 .h 扩展名表示实现文件和头文件

\item
变量名:常规变量使用小写字母加下划线,类成员使用尾随下划线,常量使用 kCamelCase

\item
命名类名:使用 CamelCase 作为类名

\item
缩进和格式:使用空格代替制表符,并使用两个空格的缩进

\item
指针和引用表达式:将 * 或 \& 与变量名放在一起(int* ptr,而不是 int *ptr)

\item
限制:避免使用非常量全局变量,尽可能使用算法而不是循环
\end{itemize}

\item
LLVM 编码标准:在 LLVM 编译器基础结构中使用,这些标准强调可读性和效率:

\begin{itemize}
\item
文件名:源文件使用 .cpp 扩展名,头文件使用 .h。

\item
变量名:变量和函数使用 camelBack 样式。成员变量末尾带有下划线。 类名:类和结构采用 CamelCase 格式。

\item
缩进和格式化:缩进两个空格,重点强调可读性并避免代码过于紧凑。

\item
指针和引用表达式:将 * 或 \& 放置在类型旁边(int *ptr,而不是 int* ptr)。

\item
现代C++ 用法:鼓励使用现代 C++ 功能和模式。
\end{itemize}

\item
Mozilla 编码风格:虽然不像 Google 或 LLVM 那样得到普遍认可,但 Mozilla 的编码风格仍然很重要,尤其是在与其技术相关的项目中:

\begin{itemize}
\item
文件名:使用 .cpp 和 .h 扩展名

\item
变量名:变量和函数使用驼峰命名法,类使用驼峰命名法,常量使用 SCREAMING\_SNAKE\_CASE 命名法

\item
类名:类名采用 CamelCase 命名

\item
缩进和格式化:最好使用四个空格进行缩进,并遵循清晰的块分隔样式 指针和引用

\item
表达式:与 LLVM 类似,将 * 或 \& 放置在类型旁边

\item
强调性能:鼓励编写高效的代码,重点关注浏览器性能
\end{itemize}
\end{itemize}

这些惯例都有自己的理念和原理。 Google 的风格指南强调在庞大的代码库和大量开发人员之间保持一致性。 LLVM 的标准侧重于利用现代 C++ 功能的简洁、高效的代码。 Mozilla 的风格平衡了可读性和性能,反映了其在 Web 技术开发中的起源。选择一种符合项目目标、 团队规模和您使用的特定技术的风格非常重要。


















