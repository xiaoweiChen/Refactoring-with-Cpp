
无论您使用哪种面向对象编程 (OOP) 语言,某些通用命名原则都有助于提高代码的清晰度和可维护性。这些原则旨在确保代码中的名称提供有关其用途和功能的充分信息。

\mySubsubsection{5.1.1}{描述性}

名称应准确描述其所标识的变量、函数、类或方法的用途或值。例如,函数的 getSalary() 比简单的 getS() 更具信息量。

\mySubsubsection{5.1.2}{一致性}

命名约定的一致性是编写清晰且可维护的代码的最重要原则之一。当您在整个代码库中保持一致的命名时,阅读、理解和调试代码就会变得容易得多。原因是,一旦开发人员了解了您的命名模式,他们就可以将他们的理解应用于整个代码库,而不必孤立地弄清楚每个名称的含义。

一致性适用于许多领域,其中包括:

\begin{itemize}
\item
大小写风格:如果您一开始就以蛇形命名法命名变量(例如 employee\_salary),则在整个代码库中都应坚持这种风格。不要在蛇形命名法、驼峰命名法(例如 employeeSalary )和帕斯卡命名法(例如 EmployeeSalary)之间切换。

\item
前缀和后缀:如果使用前缀或后缀来表示某些属性,例如使用 m\_ 表示成员变量(例如m\_value),请确保在任何地方都遵循此规则。

\item
缩写规则:如果您决定缩写某些单词,请确保始终如一地这样做。例如,如果您使用 n um 来表示数字(如 numEmployees),那么当您表示数字时,请始终使用 num。

\item
特定于语言构造的命名约定:在 OOP 语言中,类名通常是名词 (Employee),方法名是动词 (calculateSalary),布尔变量或方法通常以 is、 has、 can 或类似前缀开头 (isAvailable 和 hasCompleted)。始终遵循这些约定。
\end{itemize}

假设您正在处理一个大型代码库,其中的类代表公司中不同类型的员工。您已决定以 Pascal Case 格式将类命名为单数名词,以 snake\_case 格式将方法命名为动词,以 snake\_case 格式将变量命名为 snake\_case。

此命名约定的一致实现可能如下所示:

\begin{cpp}
class SoftwareEngineer {
public:
    void assign_task(std::string task_name) {
        current_task_ = std::move(task_name);
    }
private:
    std::string current_task_;
};
\end{cpp}

让我们分解一下此代码片段:

\begin{itemize}
\item
SoftwareEngineer 类是单数名词,使用 PascalCase

\item
assign\_task 方法是一个动词,使用 snake\_case

\item
变量 current\_task 采用 snake\_case 格式
\end{itemize}

遵循这一惯例将有助于阅读代码的任何人立即识别每个名称代表的含义。这样一来,认知负担就会减轻,开发人员可以专注于实际逻辑,而不会被不一致或令人困惑的名称分散注意力。

\mySubsubsection{5.1.3}{明确性}

明确性意味着名称不应具有误导性。避免使用可能以多种方式解释的名称,或与既定惯例或期望相矛盾的名称。例如,假设您有一个 Document 类和一个名为 process 的方法。如果没有更多上下文,该方法名称就会产生歧义:

\begin{cpp}
class Document {
public:
    void process();
};
\end{cpp}

在这种情况下,流程可能意味着很多事情。我们要解析文档吗?我们要渲染它吗?我们要将其保存到文件中吗?或者我们要执行所有这些操作吗?这还不清楚。

更具体的方法名称有助于阐明其用途。根据方法的用途,可以将其命名为解析、渲染、保存等:

\begin{cpp}
class Document {
public:
    void parse(const std::string& content);
    void render();
    void save(const std::string& file_path);
};
\end{cpp}

每个方法名称都更清楚地表明了该方法的作用,消除了原始过程方法名称的歧义。

\mySubsubsection{5.1.4}{发音}

名称应该容易发音。这有利于开发人员之间就代码进行口头交流。

\mySubsubsection{5.1.5}{范围和生命周期}

作用域越大、生命周期越长的变量通常对系统的影响越大,因此,它们应该具有更周到、 更清晰、更具描述性的名称。这有助于确保它们在所有使用环境中都能被理解。下面是更详细的分类。

全局变量可以从程序中的任何位置访问,并且其生命周期与程序的持续时间相同。因此,在命名时应特别仔细考虑。名称应具有足够的描述性,以清楚地表明其在系统中的作用。此外,全局变量可能会产生意外的依赖关系,这会使程序更难理解和维护。因此,应尽量减少使用全局变量:

\begin{cpp}
// Global variable
constexpr double GRAVITATIONAL_ACCELERATION = 9.8; // Clear and descriptive
\end{cpp}

类成员变量可以通过类中的任何方法访问,并且它们的生存期与类实例的生存期相关。它们应该具有清晰且描述性的名称,以反映其在类中的角色。遵循命名约定以将它们与局部变量区分开来通常很有用(例如, m\_ 前缀或 \_ 后缀):

\begin{cpp}
class PhysicsObject {
    double mass_; // Descriptive and follows naming convention
    // ...
};
\end{cpp}

局部变量仅限于特定函数或块,并且仅在该函数或块的持续时间内存在。与全局变量或类成员相比,这些变量通常不需要太多描述性的名称,但仍应清楚地传达其用途:

\begin{cpp}
double compute_force(double mass, double acceleration) {
    double force = mass * acceleration; // 'force' is clear in this context
    return force;
}
\end{cpp}

循环变量和临时变量的作用域和生命周期最短,通常局限于一个小循环或一小段代码。因此,它们通常具有最简单的名称(例如 i、 j 和 temp):

\begin{cpp}
for (int i = 0; i < num; ++i) { // 'i' is clear in this context
    // ...
}
\end{cpp}

这里的关键思想是,变量的作用域越广,生命周期越长,就越有可能让人对其用途产生混淆,因此变量的名称应该越具有描述性。目标是让代码尽可能清晰易懂。

\mySubsubsection{5.1.6}{避免对类型或范围信息进行编码}

在现代编程语言中,将类型或范围信息编码到名称中(通常称为匈牙利表示法)通常是不必要的,并且可能会导致混淆或错误,尤其是在重构时。虽然这有时很有用,特别是在弱类型语言中,但它有几个缺点,使其不太适合在强类型语言(如 C++)中使用:

\begin{itemize}
\item
变量的类型将来可能会改变,但其名称通常不会改变。这会导致误导性的情况,即变量的名称暗示一种类型,但实际上它是另一种类型。例如,您可能从一个 ID 向量 (std::vector<Id> id\_array) 开始,后来将其更改为 set<Id> 以避免重复,但变量名称仍然暗示它是一个数组或向量。

\item
现代开发环境提供了类型推断、显示类型的悬停工具提示和强大的重构工具等功能,这些功能都使手动将类型编码为名称变得非常多余。例如,安装了 clangd 插件并启用"嵌入提示"功能的 VS Code 可以动态推断类型,包括自动:

\myGraphic{0.5}{content/chapter5/images/1.png}{图 5.1 – VS Code 中的嵌入提示}

\end{itemize}

这也适用于 JetBrains 的 CLion:

\begin{itemize}
\item
匈牙利表示法中的前缀会使变量名更难阅读,尤其是对于那些不熟悉该表示法的人来说。对于新开发人员来说, dwCount(DWORD 或双字,通常用于表示无符号长整数)的含义可能不是立即显而易见的。

\item
强类型语言(例如 C++)已经在编译时检查类型安全性,从而减少了在变量名称中编码类型信息的需要。在以下示例中,整数声明为 std::vector<int>,句子声明为 std::string。C++ 编译器知道这些类型,并将确保对这些变量的操作是类型安全的:
\end{itemize}

\begin{cpp}
#include <vector>
#include <string>

int main() {
    std::vector<int> integers;
    std::string sentence;

    // The following will cause a compile-time error because
    // the type of 'sentence' is string, not vector<int>.
    integers = sentence;

    return 0;
}
\end{cpp}

当代码尝试将 sentence 赋值给整数时,会产生编译时错误,因为 sentence 不是正确的类型 (s td::vector<int>)。尽管两个变量名都没有编码类型信息,但还是会发生这种情况。

编译器的类型检查消除了在变量名称中包含类型的需要(例如 strSentence 或 vecIntegers) ,这种做法在不执行如此强大的编译时类型检查的语言中很常见。整数和句子变量名称无需对类型信息进行编码即可充分描述。

在编程中,您经常会遇到使用相同基础类型表示多个逻辑概念的情况。例如,您的系统中可能有用户和产品的标识符,它们都以整数表示。虽然 C++ 的静态类型检查提供了一定程度的安全性,但它不会区分 UserId 和 ProductId - 就编译器而言,它们都只是整数。

但是,对这些不同的概念使用相同的类型可能会导致错误。例如,完全有可能错误地将 Use rId 传递到需要 ProductId 的地方,而编译器不会捕获此错误。

为了解决这个问题,您可以利用 C++ 的丰富类型系统来引入表示这些不同概念的新类型,即使它们共享相同的底层表示。这样,编译器就可以在编译时捕获这些错误,从而增强软件的稳健性:

\begin{cpp}
// Define new types for User and Product IDs.
struct UserId {
    explicit UserId(int id): value(id) {}
    int value;
};

struct ProductId {
    explicit ProductId(int id): value(id) {}
    int value;
};

void process_user(UserId id) {
    // Processing user...
}

void process_product(ProductId id) {
    // Processing product...
}

int main() {
    UserId user_id(1);
    ProductId product_id(2);

    // The following line would cause a compile-time error because
    // a ProductId is being passed to process_user.
    process_user(product_id);

    return 0;
}
\end{cpp}

在上面的示例中, UserId 和 ProductId 是不同的类型。即使它们的底层表示相同(int),将ProductId 传递给需要 UserId 的函数也会导致编译时错误。这为您的代码增加了一层额外的类型安全性。

这只是对如何利用 C++ 丰富的静态类型系统来创建更强大、更安全的代码的简要介绍。我们将在第 6 章"在 C++ 中使用丰富的静态类型系统"中更详细地探讨这个主题。

\mySubsubsection{5.1.7}{类和方法命名}

在 OOP 语言中,类表示概念或事物,而它们的实例(对象)是这些事物的具体表现。因此,类名及其实例最合适的命名方式是使用名词或名词短语。它们表示系统中的实体,无论它们是有形的(例如 Employee 和 Invoice)还是概念性的(例如 Transaction 和 DatabaseCon nection)。

另一方面,类中的方法通常表示该类的对象可以执行的操作,或可以发送给它的消息。因此,使用动词或动词短语来命名它们最有效。它们充当对象可以执行的指令,允许它以有意义的方式与其他对象交互。

考虑一个带有打印方法的 Document 类。我们可以说"文档,打印"或"打印文档",这是一个清晰的命令式语句,与我们在日常语言中表达操作的方式一致。

以下是一个例子:

\begin{cpp}
class Document {
public:
    void print();
};

Document report;
report.print(); // "report, print!"
\end{cpp}

这种在命名类和方法时使用名词动词一致性的做法与我们自然理解和交流现实世界中的对象和动作的方式非常吻合,有助于提高代码的可读性和可理解性。此外,它还非常适合 OO P 中的封装原则,即对象管理自己的行为(方法)和状态(成员变量)。

遵循这一惯例可让开发人员编写更直观、自文档化且更易于维护的代码。它可在开发人员之间建立共同的语言和理解,减少阅读代码时的认知负担,并使代码库更易于导航和推理。因此,建议在 OOP 中遵守这些惯例。

\mySubsubsection{5.1.8}{命名变量}

变量名应该反映其所保存的数据。好的变量名应该描述变量所包含的值的类型,而不仅仅是其在所编写算法中的用途。

避免使用魔法数字,即源代码中含义不明的数值。它们会导致代码更难阅读、理解和维护。让我们考虑一个发送消息的 MessageSender 类,如果消息大小大于某个限制,它会将消息拆分为多个块:

\begin{cpp}
class MessageSender {
public:
    void send_message(const std::string& message) {
        if (message.size() > 1024) {
            // Split the message into chunks and send
        } else {
            // Send the message
        }
    }
};
\end{cpp}

在上述代码中, 1024 是一个神奇的数字。它可能代表最大消息大小,但并不是立即清楚的。它会让阅读代码的其他人(或未来的你)感到困惑。这是一个带有命名常量的重构示例:

\begin{cpp}
class MessageSender {
    constexpr size_t MAX_MESSAGE_SIZE = 1024;
public:
    void send_message(const std::string& message) {
        if (message.size() > MAX_MESSAGE_SIZE) {
            // Split the message into chunks and send
        } else {
            // Send the message
        }
    }
};
\end{cpp}

在此重构版本中,我们用命名常量 MAX\_MESSAGE\_SIZE 替换了魔数 1024。现在很明显, 1024 是最大消息大小。以这种方式使用命名常量可让您的代码更具可读性和可维护性。
如果将来需要更改最大消息大小,您只需在一个地方更新它即可。

\mySubsubsection{5.1.9}{利用命名空间}

C++ 中的命名空间对于防止命名冲突和正确构造代码非常有用。当程序中的两个或多个标识符具有相同的名称时,就会出现命名冲突或冲突的问题。例如,您可能在应用程序的两个子系统中都有一个名为 Id 的类 - 网络代表连接 ID,而用户管理中的用户 ID 则代表用户。如果在没有命名空间的情况下同时使用它们,则会导致命名冲突,并且编译器不知道您在代码中引用的是哪个 Id。

为了缓解这种情况, C++ 提供了 namespace 关键字,以将功能封装在唯一名称下。命名空间旨在解决名称冲突问题。通过将代码包装在命名空间内,可以防止它与代码其他部分或第三方库中的同名标识符发生冲突。

以下是一个例子:

\begin{cpp}
namespace product_name {
    class Router {
        // class implementation
    };
}

// To use it elsewhere in the code
product_name::Router myRouter;
\end{cpp}

在这种情况下, product\_name::Router 不会与产品代码或第三方库中的任何其他 Router 类冲突。如果您开发库代码,强烈建议将其所有实体(例如类、函数和变量)包装在命名空间中。这将防止与其他库或用户代码发生名称冲突。

在 C++ 中,使用命名空间来映射项目的目录结构是很常见的,这样可以更容易地理解代码库的不同部分的位置。例如,如果您在 ProductRepo/Networking/Router.cpp 路径下有一个文件,您可以像这样声明 Router 类:

\begin{cpp}
namespace product_name {
    namespace networking {
        class Router {
            // class implementation
        };
    }
}
\end{cpp}

然后,您可以引用具有完全限定名称 product\_name::networking::Router 的类。

然而,值得注意的是,直到 C++20,该语言本身并不支持可以替代或增强命名空间提供的功能的模块系统。随着 C++20 中模块的出现,一些做法可能会发生变化,但了解命名空间及其在命名中的用法仍然至关重要。

使用命名空间的另一种方法是表达代码的复杂程度。例如,库代码可能包含预期由库使用者和内部人员使用的实体。以下代码片段演示了这种方法:

\begin{cpp}
// communication/client.hpp
namespace communication {
class Client {
public:
    // public high-level methods
private:
    using HttpClient = communication::advanced::HttpClient;
    HttpClient inner_client_;
};
} // namespace communication

// communication/http/client.hpp
namespace communication::advanced::http {
class Client {
    // Lower-level implementation
};
} // namespace communication::advanced
\end{cpp}

在此扩展示例中, communication::Client 类提供了用于发送和接收消息的高级接口。它使用advanced::http::Client 类进行实际实现,但此细节对库的用户是隐藏的。他们不需要了解 adv anced 类,除非他们对默认客户端提供的功能不满意并且需要更多控制。

在 communication::http::advanced 命名空间中的 Client 类提供了更多低级功能,使用户可以更好地控制通信的细节。

这种组织方式明确了哪些功能适用于大多数用户(客户端),哪些功能适用于更高级的用途(HttpClient)。以这种方式使用命名空间还有助于避免名称冲突,并使代码库保持井然有序。许多库和框架都采用这种方法 - 例如, Boost 库通常有一个详细的命名空间用于内部实现。

\mySubsubsection{5.1.10}{使用领域特定语言}

如果问题领域中有成熟的术语,请在代码中使用它们。这可以让熟悉该领域的人更容易理解您的代码。例如,在金融领域,常用的术语有"投资组合"、"资产"、"债券"、" 股票"、"股票代码"和"股息"。如果您正在编写与金融相关的应用程序,则在类名和变量名中使用这些术语会很有帮助,因为它们可以向任何有金融背景的人清楚地传达它们的作用。

考虑以下代码片段:

\begin{cpp}
class Portfolio {
public:
    void add_asset(std::unique_ptr<Asset> asset) {
        // add the asset to the portfolio
    }

    double total_dividend() const {
        // calculate the total dividends of the portfolio
    }

private:
    std::vector<std::unique_ptr<Asset>> assets_;
};

using Ticker = std::string;

class Asset {
public:
    Asset(const Ticker& ticker, int64_t quantity) :
        ticker_{ticker},
        quantity_{quantity} {}
    virtual Asset() = default;
    virtual double total_dividend() const = 0;
    auto& ticker() const { return ticker_; }
    int64_t quantity() const { return quantity_; }
private:
    Ticker ticker_;
    int64_t quantity_;
};

class Bond : public Asset {
public:
    Bond(const Ticker& ticker, int64_t quantity) :
        Asset{ticker, quantity} {}
    double total_dividend() const override {
        // calculate bond dividend
    }
};

class Equity : public Asset {
public:
    Equity(const Ticker& ticker, int64_t quantity) :
        Asset{ticker, quantity} {}
    double total_dividend() const override {
        // calculate equity dividend
    }
};
\end{cpp}

在此示例中, Portfolio、 Asset、 Bond、 Equity、 Ticker 和 total\_dividend() 都是直接借用自金融领域的术语。熟悉金融的开发人员或利益相关者只需通过名称就能了解这些类和方法的用途。这有助于在开发人员、利益相关者和领域专家之间创建一种通用语言,从而大大促进沟通和理解。请注意,不建议在实际金融应用中使用 double,因为它没有足够准确的表示来防止在对货币值进行算术运算时累积舍入误差。

请记住,这些原则的目的是让你的代码尽可能清晰易懂。编写代码不仅仅是与计算机沟通,它还与其他开发人员(包括未来的自己)沟通。

\mySubsubsection{5.1.11}{平衡代码中的长名称和注释}

正确的命名惯例对于代码的清晰度和可读性起着至关重要的作用。类、方法和变量的名称应具有足够的描述性,以传达其目的和功能。理想情况下,一个精心挑选的名称可以取代对额外注释的需要,使您的代码一目了然。

然而,需要找到一个微妙的平衡点。虽然长而描述性的名称很有用,但过长的名称也会很麻烦,并且会降低代码的可读性。另一方面,过短的名称可能会产生歧义,并使代码更难理解。关键是要找到正确的平衡点——名称应该足够长以传达其目的,但又不能太长而难以处理。

考虑一下这个来自假设的网络应用程序的例子:

\begin{cpp}
class Router {
public:
    void route(const Message& message, Id receiver) {
        auto message_content = message.get_content();
        // Code to route the 'message_content' to the appropriate 'receiver'
    }
private:
    // Router's private members
};
\end{cpp}

在这种情况下,路由方法名称以及消息、接收器和 message\_content 变量名称都具有足够的描述性,可以理解该方法的作用以及每个变量代表什么。无需添加其他注释来解释它们的作用。

话虽如此,在某些情况下,语言结构无法完全表达代码的意图或细微差别,例如依赖第三方库中的特定行为或编写复杂算法时。在这些情况下,需要添加额外的注释来提供背景信息或解释为什么做出某些决定。

举个例子:

\begin{cpp}
void route(const Message& message, Id receiver) {
    auto message_content = message.get_content();

    // Note: The routing_library has an idiosyncratic behavior where
    // it treats receiver id as one-indexed. Hence we need to
    increment by 1.
    receiver++;
    // Code to route the 'message_content' to the appropriate 'receiver'
}
\end{cpp}

在这种情况下,有必要通过注释来强调第三方路由库的特定行为,而这仅从语言结构上是无法立即显现出来的。

作为一般规则,请努力通过良好的命名实践使代码尽可能不言自明,但如果需要注释来提供重要背景信息或阐明复杂逻辑,请毫不犹豫地使用注释。请记住,最终目标是创建易于阅读、理解和维护的代码。








