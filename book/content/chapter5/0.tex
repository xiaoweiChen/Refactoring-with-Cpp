随着对 C++ 或任何其他编程语言的深入了解,有一件事变得越来越清晰——命名的力量。本章中,我们将探讨命名约定在编写干净、可维护和高效的 C++ 代码方面的重要性。

在计算机编程中,变量、函数、类和许多其他实体都会赋予名称。这些名称充当标识符,作为开发者如何与代码组件交互方面起着关键作用。虽然对某些人来说这似乎是一件微不足道的事情,但选择正确的名称会对软件项目的可理解性和可维护性产生深远的影响。我们选择用来表示程序不同元素的名称是任何人(包括未来的自己)在接触代码时所拥有的第一层文档。

假设一位名叫 Mia 的开发人员正在使用一个名为 WeatherData 的类。该类有两个 getter 方法- get\_tempreture() 和 get\_hydrity()。前一种方法仅返回存储在成员变量中的当前温度值。这是一个 O(1) 操作,涉及返回已存储的值。后者的作用不仅仅是返回一个值,实际上会启动与远程气象服务的连接,检索最新的湿度数据,然后返回它。此操作成本相当高,涉及网络通信和数据处理,因此远非 O(1) 操作。 Mia 专注于优化项目中的函数,她看到了这两个 getter,并根据它们的命名假设它们在效率方面相似。她在循环中使用 get\_hydrity(),期望它是一个简单、高效的存储值检索,类似于 get\_temperature()。由于重复调用 get\_hydrity(),成本高昂,该函数的性能急剧下降。每次调用所涉及的网络请求和数据处理会显著降低执行速度,导致资源使用效率低下,应用程序性能下降。如果将该方法命名为 fetch\_hydrity(),而不是 get\_hydrity(),则可能避免这种情况。名称 fetch\_hydrity() 可以清楚地表明该方法不是一个简单的获取器,而是一个更昂贵的操作,涉及从远程服务获取数据。

命名的艺术需要仔细考虑,并充分理解问题域和编程语言。本章全面讨论了在 C++ 中创建和命名变量、类成员、方法和函数的一般方法。我们将讨论长名称与短名称的权衡,以及注释在阐明开发意图方面的作用。

我们将探讨编码约定的重要性,以及它们为个人开发人员和团队带来的好处。始终如一地应用经过深思熟虑的命名约定可以简化编码过程,减少错误,并大大提高代码库的可读性。

读完本章后,将明白为什么良好的命名实践不仅仅是事后的想法,而且是良好软件开发的重要组成部分。我们将为您提供策略和惯例,以帮助您编写代码,让其他人可以轻松阅读、理解和维护代码,而当您几个月或几年后重新查看自己的代码时,也可以轻松阅读、理解和维护代码。
