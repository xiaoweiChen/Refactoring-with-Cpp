本章中,我们探讨了命名编程的关键作用。认识到良好、一致的命名实践可提高代码的可读性和可维护性,同时也有助于代码的自文档化。

我们仔细考虑了使用长描述性名称和辅以注释的较短名称之间的平衡,并了解到这两种名称在不同上下文中都有其适用之处。建议在命名中使用特定领域的语言,以提高清晰度,同时警告不要使用"魔数",因为它们不够透明。

还讨论了变量的范围和生命周期对其命名的影响,强调需要对范围更大、生命周期更长的变量使用更具描述性的名称。

本章最后强调了遵守命名编码约定的价值,这可以确保整个代码库的一致性,从而简化代码阅读和理解过程。

本章所获得的见解为接下来的讨论奠定了基础,即如何有效利用 C++ 中丰富的静态类型系统来编写更安全、更干净、更清晰的代码。在下一章中,我们将重点关注如何有效利用 C++ 丰富的静态类型系统。













