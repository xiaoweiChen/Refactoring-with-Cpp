
在 C++ 中开发共享库需要仔细考虑以确保兼容性、稳定性和可用性。最初,共享库旨在促进代码重用、模块化和高效的内存使用,允许多个程序同时使用相同的库代码。这种方法有望减少冗余、节省系统资源并提供仅替换应用程序部分的功能。虽然这种方法对于广泛使用的库(例如 libc、 libstdc++、 OpenSSL 等)效果很好,但事实证明它对应用程序效率较低。应用程序提供的共享库很少能完美地替换为较新版本。通常,需要替换整个安装套件,其中包括应用程序及其所有依赖项。

如今,共享库通常用于实现不同编程语言之间的互操作性。例如, C++ 库可用于用 Java 或Python 编写的应用程序。这种跨语言功能扩展了库的可用性和覆盖范围,但也带来了开发人员必须考虑的某些复杂性和注意事项。

\mySubsubsection{8.5.1}{单个项目内的共享库}

如果共享库设计为在单个项目中使用并由使用相同编译器编译的可执行文件加载,则具有 C ++ 接口的共享对象(或 DLL)通常是可以接受的。但是,这种方法有一些注意事项,例如使用单例,这可能会导致多线程问题和意外的初始化顺序。当使用单例时,在多线程环境中管理它们的初始化和销毁可能具有挑战性,从而导致潜在的竞争条件和不可预测的行为。此外,确保全局状态的初始化和销毁顺序正确也很复杂,这可能会导致难以诊断的细微错误。

\mySubsubsection{8.5.2}{共享库可更广泛地分发}

如果共享库预计会更广泛地分发,开发人员无法预测最终用户使用的编译器,或者该库可能被其他编程语言使用,那么 C++ 共享库就不是理想的选择。这主要是因为 C++ 应用程序二进制接口 (ABI) 在不同编译器甚至同一编译器的不同版本之间并不稳定。 C++ ABI 不稳定性源于语言的复杂性、不断发展的标准、特定于平台的变化、编译器优化以及运行时和标准库实现的差异。这种不稳定性可能导致二进制兼容性问题,使分发共享库变得困难并增加维护开销。例如, C++ 模板和内联函数在每个使用它们的翻译单元中实例化,导致编译器或编译器版本之间可能存在差异。 C++ 标准的变化和编译器更新也会改变 ABI。不同的操作系统和硬件平台可能有自己的调用约定和二进制格式,这增加了复杂性。编译器优化可能有所不同, C++ 标准库的实现在不同的编译器和平台上也有所不同,这进一步导致了ABI 不兼容。当使用不同编译器或版本编译的模块交互时,这些问题可能会导致崩溃或未定义的行为,因此需要仔细进行库的版本控制和管理。缓解 ABI 不稳定性的一种有效方法是使用 C 接口,因为 C ABI 更稳定。尽管没有正式标准化,但 C ABI 实际上是编程语言之间的标准粘合剂,因为所有语言本质上都需要支持 C 接口才能与 libc 或操作系统系统调用进行通信,而这些调用也是用 C 编写的。解决此问题的一个常见方法是围绕 C++ 代码开发一个 C 包装器并发送 C 接口。

\mySubsubsection{8.5.3}{示例 – MessageSender 类}

接下来是一个演示此方法的示例,我们创建一个 C++ MessageSender 类并为其提供一个 C 包装器。该类有一个构造函数,它使用指定的接收器初始化 MessageSender 实例,还有两个重载的发送方法,允许将消息作为 std::vector<uint8\_t> 实例或具有指定长度的原始指针发送。该实现将消息打印到控制台以演示功能。

以下是 C++ 库的实现:

\begin{cpp}
// MessageSender.hpp
#pragma once
#include <string>
#include <vector>

class MessageSender {
public:
    MessageSender(const std::string& receiver);
    void send(const std::vector<uint8_t>& message) const;
    void send(const uint8_t* message, size_t length) const;
};

// MessageSender.cpp
#include "MessageSender.h"
#include <iostream>

MessageSender::MessageSender(const std::string& receiver) {
    std::cout << "MessageSender created for receiver: " << receiver <<
std::endl;
}
void MessageSender::send(const std::vector<uint8_t>& message) const {
    std::cout << "Sending message of size: " << message.size() <<
std::endl;
}
void MessageSender::send(const uint8_t* message, size_t length) const
{
    std::cout << "Sending message of length: " << length << std::endl;
}
\end{cpp}

以下是 C 包装器的实现:

\begin{cpp}
// MessageSender.h (C Wrapper Header)
#ifdef __cplusplus
extern "C" {
#endif

typedef void* MessageSenderHandle;
MessageSenderHandle create_message_sender(const char* receiver);
void destroy_message_sender(MessageSenderHandle handle);
void send_message(MessageSenderHandle handle, const uint8_t* message, size_t length);

#ifdef __cplusplus
}
#endif

// MessageSenderC.cpp (C Wrapper Implementation)
#include "MessageSenderC.h"
#include "MessageSender.hpp"

MessageSenderHandle create_message_sender(const char* receiver) {
    return new(std::nothrow) MessageSender(receiver);
}

void destroy_message_sender(MessageSenderHandle handle) {
    MessageSender* instance = reinterpret_cast<MessageSender*>(handle);
    assert(instance);
    delete instance;
}

void send_message(MessageSenderHandle handle, const uint8_t* message, size_t length) {
    MessageSender* instance = reinterpret_cast<MessageSender*>(handle);
    assert(instance);
    instance->send(message, length);
}
\end{cpp}

在此示例中, C++ MessageSender 类在 MessageSender.hpp 和 MessageSender.cpp 文件中定义。该类有一个构造函数,用于使用指定的接收器初始化 MessageSender 实例,还有两个重载的发送方法,允许将消息作为 std::vector<uint8\_t> 实例或具有指定长度的原始指针发送。该实现将消息打印到控制台以演示功能。

为了使此 C++ 类可从其他编程语言或不同的编译器使用,我们创建了一个 C 包装器。 C 包装器在 MessageSender.h 和 MessageSenderC. cpp 文件中定义。头文件使用 extern "C" 块来确保 C++ 函数可从 C 调用,从而防止名称混淆。 C 包装器使用不透明句柄 void*(typedef 为MessageSenderHandle)来表示 C 中的 MessageSender 实例,从而抽象实际的 C++ 类。

create\_message\_sender 函数分配并初始化 MessageSender 实例并返回其句柄。请注意,它使用 new(std::nothrow) 来避免在内存分配失败时引发异常。 C 或任何其他不支持异常的编程语言仍可使用此函数而不会出现问题。

destroy\_message\_sender 函数释放 MessageSender 实例以确保正确清理。 send\_message 函数使用句柄调用 MessageSender 实例上的相应 send 方法,以便发送消息。
通过在同一二进制文件中处理内存分配和释放,这种方法避免了最终用户使用不同内存分配器导致内存损坏或泄漏的问题。 C 包装器提供了一个稳定且一致的接口,可以在不同的编译器和语言中使用,从而确保更高的兼容性和稳定性。这种方法解决了开发共享库的复杂性,并确保了其广泛的可用性和可靠性。

如果预计 C++ 库会抛出异常,则必须在 C 包装函数中正确处理这些异常,以防止异常传播给调用者。例如,我们可以有以下异常类型:

\begin{cpp}
class ConnectionError : public std::runtime_error {
public:
    ConnectionError(const std::string& message) : std::runtime_error(message) {}
};

class SendError : public std::runtime_error {
public:
    SendError(const std::string& message) : std::runtime_error(message) {}
};
\end{cpp}

然后, C 包装函数可以捕获这些异常并向调用者返回适当的错误代码或消息:

\begin{cpp}
// MessageSender.h (C Wrapper Header)
typedef enum {
    OK,
    CONNECTION_ERROR,
    SEND_ERROR,
} MessageSenderStatus;
// MessageSenderC.cpp (C Wrapper Implementation)
MessageSenderStatus send_message(MessageSenderHandle handle, const
uint8_t* message, size_t length) {
    try {
        MessageSender* instance = reinterpret_
        cast<MessageSender*>(handle);
        instance->send(message, length);
        return OK;
    } catch (const ConnectionError&) {
        return CONNECTION_ERROR;
    } catch (const SendError&) {
        return SEND_ERROR;
    } catch (...) {
        std::abort();
    }
}
\end{cpp}

请注意,我们在发生未知异常时使用 std::abort,因为跨语言边界传播未知异常是不安全的。

此示例说明如何围绕 C++ 库创建 C 包装器,以确保开发共享库时的兼容性和稳定性。通过遵循这些准则,开发人员可以创建强大、可维护且广泛兼容的共享库,确保其在各种平台和编程环境中的可用性。










