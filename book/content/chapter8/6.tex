在本章中,我们探讨了在 C++ 中设计和开发共享库的关键方面。共享库最初旨在通过允许多个程序同时使用相同的库代码来促进代码重用、模块化和高效的内存使用。这种方法减少了冗余并节省了系统资源。

我们深入研究了为不同环境开发共享库的细微差别。当共享库旨在用于单个项目并使用同一编译器进行编译时,具有 C++ 接口的共享对象(或 DLL)可能适用,但需谨慎使用单例和全局状态,以避免多线程问题和不可预测的初始化顺序。

但是,对于最终用户的编译器或编程语言可能不同的更广泛分布,由于 C++ ABI 在不同的编译器和版本中不稳定,因此不建议直接使用 C++ 共享库。为了解决这个问题,我们讨论了围绕 C++ 代码创建一个 C 包装器,利用稳定的 C ABI 实现更广泛的兼容性和跨语言功能。

我们提供了一个使用 MessageSender 类的综合示例,说明如何创建 C++ 库及其对应的 C 包装器。该示例强调了安全的内存管理,确保在同一二进制文件中进行分配和释放,并通过在 C 接口中使用枚举状态表示异常来妥善处理异常。

通过遵循这些准则,开发人员可以创建强大、可维护且广泛兼容的共享库,确保其在各种平台和编程环境中的可用性。本章为开发人员提供必要的知识,以解决常见的问题并实施共享库开发的最佳实践,从而促进有效且可靠的软件解决方案。

在下一章中,我们将把重点转移到代码格式,探索创建清晰、一致和可读的代码的最佳实践,这对于协作和长期维护至关重要。
