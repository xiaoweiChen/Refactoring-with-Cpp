为了巩固对这些概念的理解,我们将研究 C++ 中一些真实的 API 设计示例。这些示例将重点介绍常见的挑战和有效的解决方案,展示如何在实际场景中应用良好 API 设计的原则。
通过这些示例,我们旨在提供清晰、可操作的见解,您可以将其应用于自己的项目,确保您的 API 不仅功能齐全,而且优雅且易于维护。让我们深入了解现实世界 API 设计的复杂性,看看这些原则如何在实践中发挥作用:

\begin{itemize}
\item
适用于现代 C++ 的 JSON (nlohmann/json):此库是极简 API 设计的绝佳示例。它提供了直观、直接的方法来解析、序列化和操作 C++ 中的 JSON 数据,并具有以下优点:

\begin{itemize}
\item
简单:界面清晰简洁,易于使用。

\item
功能分解:每个功能处理与 JSON 处理相关的特定任务。

\item
最小依赖性:设计用于与 C++ 标准库配合使用,避免不必要的外部依赖性:

\begin{cpp}
#include <nlohmann/json.hpp>
nlohmann::json j = {
    {"pi", 3.141},
    {"happy", true},
    {"name", "Niels"},
    {"nothing", nullptr},
    {"answer", {
            {"everything", 42}
    }},
    {"list", {1, 0, 2}},
    {"object", {
            {"currency", "USD"},
            {"value", 42.99}
    }}
};
\end{cpp}
\end{itemize}

\item
SQLite C++ 接口 (SQLiteCpp):此库提供了一个极简接口,用于与 C++ 中的 SQLite 数据库进行交互。它具有以下优点:

\begin{itemize}
\item
简单:为数据库操作提供简单明了的 API。

\item
接口隔离:为不同的数据库操作(如查询和事务)提供单独的类。

\item
最小依赖性:使用 SQLite 和 C++ 标准库构建:

\begin{cpp}
#include <SQLiteCpp/SQLiteCpp.h>

SQLite::Database db("test.db", SQLite::OPEN_READWRITE|SQLite::OPEN_CREATE);

db.exec("CREATE TABLE test (id INTEGER PRIMARY KEY, valueTEXT)");

SQLite::Statement query(db, "INSERT INTO test (value) VALUES(?)");

query.bind(1, "Sample value");
query.exec();
\end{cpp}
\end{itemize}
\end{itemize}






















