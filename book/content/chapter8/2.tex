功能分解是将复杂功能分解为更小、更易于管理的单元的过程。此技术对于创建简约的 API 至关重要,因为它可以促进简单性和模块化。通过分解功能,您可以确保 API 的每个部分都有明确、定义良好的用途,从而提高可维护性和可用性。

功能分解的关键方面包括:

\begin{itemize}
\item
模块化设计:设计 API 时,应使每个模块或功能处理整体功能的特定方面。这种关注点分离 (SoC) 可确保 API 的每个部分都有明确、定义良好的用途。

\item
单一职责原则 (SRP):每个函数或类应该只有一个更改原因。此原则有助于保持 API 简单且专注。

\item
可重用组件:通过将功能分解为更小的单元,您可以创建可重用的组件,这些组件可以以不同的方式组合来实现各种任务,从而增强 API 的灵活性和可重用性。
\end{itemize}

接口隔离旨在保持接口精简并专注于特定任务,避免设计试图覆盖太多用例的单片接口。此原则确保客户端只需要了解与他们相关的方法,从而使 API 更易于使用和理解。

接口隔离的关键方面包括:

\begin{itemize}
\item
特定接口:不要设计一个大型的通用接口,而要设计多个较小的特定接口。每个接口都应满足功能的特定方面。

\item
以用户为中心的设计:考虑 API 最终用户的需求。设计直观的界面,仅提供用户完成任务所需的方法,避免不必要的复杂性。

\item
减少对客户的影响:更小、更集中的界面可在需要进行更改时将对客户的影响降至最低。使用特定界面的客户端不太可能受到不相关功能更改的影响。
\end{itemize}

让我们考虑一个例子,其中一个复杂的 API 类负责各种功能,例如加载、处理和保存数据:

\begin{cpp}
class ComplexAPI {
public:
    void initialize();
    void load_data_from_file(const std::string& filePath);
    void load_data_from_database(const std::string& connection_string);
    void process_data(int mode);
    void save_data_to_file(const std::string& filePath);
    void save_data_to_database(const std::string& connection_string);
    void cleanup();
};
\end{cpp}

主要问题是该类承担了太多职责,混合了不同的数据源和接收器,导致复杂性和缺乏重点。让我们从将加载和处理功能提取到单独的类开始:

\begin{cpp}
class FileDataLoader {
public:
    explicit FileDataLoader(const std::string& filePath) : filePath(filePath) {}
    void load() {
        // Code to load data from a file
    }

private:
    std::string filePath;
};

class DatabaseDataLoader {
public:
    explicit DatabaseDataLoader(const std::string& connection_string)
    : _connection_string(connection_string) {}
    void load() {
        // Code to load data from a database
    }

private:
    std::string _connection_string;
};

class DataProcessor {
public:
    void process(int mode) {
        // Code to process data based on the mode
    }
};
\end{cpp}

下一步是将保存功能提取到单独的类中:

\begin{cpp}
class DataSaver {
public:
    virtual void save() = 0;
    virtual ~DataSaver() = default;
};

class FileDataSaver : public DataSaver {
public:
    explicit FileDataSaver(const std::string& filePath) :
    filePath(filePath) {}
    void save() override {
        // Code to save data to a file
    }

private:
    std::string filePath;
};

class DatabaseDataSaver : public DataSaver {
public:
    explicit DatabaseDataSaver(const std::string& connection_string) : _connection_string(connection_string) {}

    void save() override {
        // Code to save data to a database
    }

private:
    std::string _connection_string;
};
\end{cpp}

最小化 API 所需的依赖项数量对于实现极简主义至关重要。依赖项越少, API 就越稳定、 可靠且易于维护。依赖项会使集成变得复杂,增加兼容性问题的风险,并使 API 更难理解。

减少依赖的关键策略包括:

\begin{itemize}
\item
核心功能重点:专注于实现 API 本身的核心功能,除非绝对必要,否则避免依赖外部库或组件。

\item
选择性使用库:当需要外部库时,请选择那些稳定、维护良好且广泛使用的库。确保它们与您的 API 需求紧密结合。

\item
解耦设计:尽可能以能够独立于外部组件的方式设计 API。使用依赖注入 (DI) 或其他设计模式将实现与特定依赖项解耦。

\item
版本管理:谨慎管理和指定任何依赖项的版本,以避免兼容性问题。确保依赖项的更新不会破坏 API 或引入不稳定性。
\end{itemize}

