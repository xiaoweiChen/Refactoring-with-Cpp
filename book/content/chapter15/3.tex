

深入研究代码审查的协作过程之前,团队必须做好充分的准备,以确保这些审查尽可能有效和高效。这种准备不仅可以为更顺畅的审查过程奠定基础,还可以最大限度地减少花在可避免问题上的时间,使团队能够专注于更实质性和更有影响力的讨论。

\mySubsubsection{15.3.1}{明确的指导方针}

有效代码审查流程的基础是制定和记录针对 C++ 定制的清晰、具体的编码指南。这些指南应涵盖编码的各个方面,包括风格、实践和特定于语言的功能的使用。通过设置这些标准,团队可以创建一种通用语言,以减少歧义并确保整个代码库的一致性。

这些准则中尽可能多地加入自动化功能,可以大大简化审查流程。格式化程序等工具可确保一致的编码风格,而静态分析工具可以在人工审查人员查看代码之前,自动检测出潜在的错误或代码异味。此外,确保所有代码提交都通过单元测试,并在适用的情况下通过端到端测试,可以防止许多常见的软件缺陷进入审查阶段。这种自动化水平不仅为审查人员节省了时间,还降低了因编码风格主观偏好或疏忽而产生争议的可能性。

\mySubsubsection{15.3.2}{自我审查}

准备代码审查的另一个重要方面是自我审查的实践。在提交代码进行同行评审之前,开发人员应该彻底检查自己的工作。这时, linter 和静态分析等工具就会发挥作用,帮助发现容易被忽视的常见问题。

自我审查鼓励开发人员对其代码的初始质量负责,减轻同行评审人员的负担并培养一种责任文化。允许开发人员在让其他人参与之前反思自己的工作并考虑潜在的改进,从而可以提高提交的质量并提高评审会议的效率。开发人员在提交代码进行同行评审之前,应通过询问以下问题来系统地评估自己的工作。这种反思性实践有助于改进代码,使其与项目目标更加一致,并为开发人员在代码评审过程中的后续讨论做好准备:

\begin{enumerate}
\item
我是否需要编写代码?(我的更改是否合理?)在添加新代码之前,请考虑更改是否必要。评估功能是否必不可少,并论证添加的合理性,同时牢记增加代码库复杂性的可能性。

\item
我可以做些什么来避免编写代码?(是否有可以利用的第三方库或工具?)始终寻找利用现有解决方案的机会。使用经过充分测试的第三方库或工具通常可以在不添加新代码的情况下实现所需的功能,从而减少潜在的错误和维护开销。

\item
我的代码是否可读?评估代码的清晰度。好的代码应该对其他可能不熟悉它的工程师来说一目了然。使用有意义的变量名,保持干净的结构,并在必要时添加注释来解释复杂的逻辑。

\item
其他工程师是否需要了解我的代码逻辑?考虑一下你的代码是否可以被其他开发人员独立理解。其他人无需大量解释就能理解逻辑,这一点至关重要,这有助于更轻松地进行维护和集成。

\item
我的代码看起来与代码库的其余部分相似吗?确保您的代码符合项目中既定的编码风格和模式。整个代码库的一致性有助于保持一致性,使软件更易于阅读,并且在集成过程中不易出错。

\item
我的代码是否高效?评估代码的效率。考虑资源使用情况(例如 CPU 时间和内存)非常重要,尤其是在性能至关重要的应用程序中。检查算法和数据结构,以确保它们最适合该任务。

\item
我的测试是否涵盖了极端情况?确认您的测试是全面的,特别是检查它们是否涵盖了极端情况。强大的测试对于确保代码在异常或意外条件下的弹性和可靠性至关重要。
\end{enumerate}

通过在自我审查过程中回答这些问题,开发人员不仅可以提高他们编写的代码的质量,还可以简化同行评审过程。这种深思熟虑的方法,可以最大限度地减少同行评审期间进行重大修订的可能性,并提高代码审查周期的整体效率。因为开发人员已经意识到并解决了许多潜在问题,这种准备工作可以在代码审查期间促成更有见地、更有建设性的讨论。




















