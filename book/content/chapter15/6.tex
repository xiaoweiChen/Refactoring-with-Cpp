
审阅者在代码审阅过程中的作用至关重要,这不仅是为了确保代码的技术质量,也是为了维持一个建设性、尊重和教育性的环境。以下是定义优秀审阅者的一些关键做法。

\mySubsubsection{15.6.1}{发起对话}

通过与被审阅者进行友好交谈来开始审阅流程。这可以是一条简短的消息,承认他们在拉取请求 (PR) 中付出的努力,并为即将到来的审阅奠定积极的基调。亲切的开场有助于与被审阅者建立融洽的关系,使后续的交流更加开放和协作。

\mySubsubsection{15.6.2}{保持礼貌和尊重}

评论时,一定要礼貌而有礼貌。请记住,被评论者已经为他们的程序代码投入了大量精力。批评应该是建设性的,重点是程序代码及其改进,而不是个人。以问题或建议的形式,而不是指令的形式表达反馈,也有助于保持积极和鼓舞人心的语气。

\mySubsubsection{15.6.3}{回顾可管理的区块}

如果可能的话,将您一次审查的代码量限制在 400 行左右。审查大量代码可能会导致疲劳,从而增加错过小问题和关键问题的可能性。将审查分解为可管理的部分不仅可以提高审查过程的效率,还有助于保持对细节的高度关注。

\mySubsubsection{15.6.4}{避免个人偏见}

审查时,重要的是要区分由于客观原因(例如语法错误、逻辑错误或偏离项目标准)而必须更改的代码和反映个人编码偏好的更改。例如,看看以下代码:

\begin{cpp}
std::string toString(bool done) {
    if (done) {
        return "done";
    } else {
        return "not done";
    }
}
\end{cpp}

有些工程师可能更喜欢按以下方式重写该函数:

\begin{cpp}
std::string toString(bool done) {
    if (done) {
        return "done";
    }
    return "not done";
}
\end{cpp}

虽然第二个版本更简洁一些,但第一个版本同样有效,并且符合项目的编码标准。如果您强烈认为个人偏好可以增强代码,请明确标记。表明这是基于个人偏好的建议,而不是强制性更改。这种清晰度有助于受审者了解哪些更改对于符合项目标准是必不可少的,哪些是可选的增强功能。

\mySubsubsection{15.6.5}{注重可理解性}

作为审阅者,要问自己的一个最关键问题是代码是否足够易于理解,以便您或团队中的其他人可以在夜间修复它。这个问题直指代码可维护性的核心。如果答案是否定的,那么讨论如何提高代码的清晰度和简单性就很重要了。易于理解的代码更易于维护和调试,这对于项目的长期健康至关重要。

成为一名优秀的审阅者不仅仅需要识别代码中的缺陷。还需要发起和维持支持性对话、尊重和认可同事的努力、有效管理审阅工作量,并提供清晰、有用的反馈,将项目标准置于个人偏好之上。通过营造积极、高效的审阅环境,不仅可以提高代码质量,还可以促进开发团队的成长和凝聚力。



















