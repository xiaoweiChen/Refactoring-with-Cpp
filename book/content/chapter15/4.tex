
成功完成代码审查流程,对于维护软件质量和营造积极高效的团队环境至关重要。我们在此概述了开发人员的基本策略,旨在确保其代码顺利有效地通过审查。

\mySubsubsection{15.4.1}{编写代码之前与审阅者和代码所有者讨论重要功能}

在着手开发重要功能之前,建议先咨询代码审阅者和所有者。初步讨论应侧重于拟议的设计、实施方法,以及该功能如何融入现有代码库。尽早参与此类对话有助于协调预期、降低后期进行重大修订的可能性,并确保该功能与项目的其他部分无缝集成。

\mySubsubsection{15.4.2}{发布之前检查代码}

将自己的代码提交给同行评审之前,请彻底检查代码。自我评估应涵盖逻辑、风格以及是否遵守项目的编码标准。寻找需要改进或简化的地方。确保提交的代码尽可能完善,这不仅有助于更顺畅的评审过程,而且还能体现您的勤奋和对评审者时间的尊重。

\mySubsubsection{15.4.3}{确保代码符合代码约定}

遵守既定的代码惯例至关重要。这些惯例涵盖从命名方案到布局和程序实践的方方面面,可确保整个代码库的一致性。一致性可使代码更易于阅读、理解和维护。在提交审核之前,请检查代码是否严格遵循这些准则,以避免在审核过程中出现不必要的反复。

\mySubsubsection{15.4.4}{代码审查是一种对话,而不是命令}

必须将代码审查视为一种对话,而不是指令。审查人员通常会提供意见和建议,旨在发起讨论,而不是单方面发号施令。特别是对于初级开发人员,重要的是要明白,他们鼓励你参与这些讨论。如果建议或更正不清楚,请寻求澄清,而不是默默地做出改变。这种互动不仅有助于你的专业发展,而且还能增强审查过程中的协作精神。

\mySubsubsection{15.4.5}{记住 --- 你的代码不代表你}

我从前任老板 Vladi Lyga 那里学到了一条重要教训:“你的代码不代表你。”开发人员通常会投入大量精力和自豪感来编写代码,接受批评可能很困难。但请务必记住,对你的代码的反馈并不是对你作为开发人员或个人的个人批评。这样做的目的是改进项目并确保最高质量的结果,有时这需要建设性的批评。将个人身份与工作分开,可以让开发人员更客观地对待反馈,并将其用作成长机会。

通过充分准备、进行开放式对话以及以建设性的角度看待反馈,开发人员可以有效地完成代码审查流程。这些做法不仅可以提高代码质量,还可以营造一个更具支持性和协作性的团队环境。
