
成功完成代码审查流程对于维护软件质量和营造积极高效的团队环境至关重要。我们在此概述了开发人员的基本策略,旨在确保他们的代码顺利有效地通过审查。

\mySubsubsection{15.4.1}{在编写代码之前与审阅者和代码所有者讨论重要功能}

在着手开发重要功能之前,建议先咨询代码审阅者和所有者。初步讨论应侧重于拟议的设计、实施方法以及该功能如何融入现有代码库。尽早参与此类对话有助于协调预期、降低后期进行重大修订的可能性,并确保该功能与项目的其他部分无缝集成。

\mySubsubsection{15.4.2}{发布之前检查代码}

在将自己的代码提交给同行评审之前,请彻底检查代码。自我评估应涵盖逻辑、风格以及是否遵守项目的编码标准。寻找任何需要改进或简化的地方。确保您提交的代码尽可能完善,这不仅有助于更顺畅的评审过程,而且还能体现您的勤奋和对评审者时间的尊重。

\mySubsubsection{15.4.3}{确保代码符合代码约定}

遵守既定的代码惯例至关重要。这些惯例涵盖从命名方案到布局和程序实践的方方面面,可确保整个代码库的一致性。一致性可使代码更易于阅读、理解和维护。在提交审核之前,请检查您的代码是否严格遵循这些准则,以避免在审核过程中出现任何不必要的反复。

\mySubsubsection{15.4.4}{代码审查是一种对话,而不是命令}

必须将代码审查视为一种对话,而不是指令。审查人员通常会提供意见和建议,旨在发起讨论,而不是单方面发号施令。特别是对于初级开发人员,重要的是要明白,他们鼓励你参与这些讨论。如果建议或更正不清楚,请寻求澄清,而不是默默地做出改变。这种互动不仅有助于你的专业发展,而且还能增强审查过程中的协作精神。

\mySubsubsection{15.4.5}{记住——你的代码不代表你}

我从前任老板 Vladi Lyga 那里学到了一条重要教训:"你的代码不代表你。"开发人员通常会投入大量精力和自豪感来编写代码,因此接受批评可能很困难。但是,请务必记住,对你的代码的反馈并不是对你作为开发人员或个人的个人批评。这样做的目的是改进项目并确保最高质量的结果,有时这需要建设性的批评。将个人身份与工作分开可以让开发人员更客观地对待反馈并将其用作成长机会。

通过充分准备、进行开放式对话以及以建设性的角度看待反馈,开发人员可以有效地完成代码审查流程。这些做法不仅可以提高代码质量,还可以营造一个更具支持性和协作性的团队环境。
