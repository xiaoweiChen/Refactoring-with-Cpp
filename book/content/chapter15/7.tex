本章深入探讨了进行有效代码审查的基本实践和原则,这是 C++ 中软件开发过程的关键组成部分。通过一系列结构化的子章节,我们探索了代码审查过程的各个方面,这些方面共同确保了高质量、可维护的代码,同时营造了积极的团队环境。

我们首先讨论了代码审查的起源,它由 Michael Fagan 在 20 世纪 70 年代提出,强调了它在将软件开发从一项孤立任务转变为一项协作任务方面的变革性作用,从而提高了代码质量并减少了错误。
在"准备代码审查"部分,我们强调了明确的指导方针和自我审查的重要性。我们鼓励开发人员在代码接受同行评审之前使用 linters 和静态分析器等工具来完善代码,确保遵守编码标准并减少代码修订的迭代周期。

如何通过代码审查部分概述了开发人员确保其代码在审查期间受到好评的策略。这包括在编码前讨论重大更改、将代码审查理解为建设性对话,并记住将个人身份与代码分开以客观地看待反馈。

"如何在代码审查期间有效解决争议"部分讨论了如何有效处理分歧。我们讨论了明确解释变更理由、使用直接沟通避免误解以及在必要时引入其他观点以解决冲突并达成共识的重要性。

最后,在"如何成为一名优秀的审阅者"部分,我们提供了有关如何以积极的互动启动审阅、以可管理的部分审阅代码、避免个人偏好偏见以及在关键情况下评估代码的清晰度和易理解性的指导。

在本章中,基本主题是代码审查不仅仅是批评代码,而是要建立一个支持性的开发者社区,他们分享知识、不断改进,并致力于在编码实践中追求卓越。目标是提高软件的技术质量和参与团队成员的专业发展。通过遵循这些最佳实践,团队可以实现更强大、更高效、 无错误的代码,为 C++ 项目的成功做出重大贡献。
