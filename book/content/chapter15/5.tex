
分歧是代码审查过程中的自然组成部分。不同的观点可能导致方法、实施或最佳实践解释方面的冲突。有效处理这些争议,对于维持高效的审查流程和健康的团队环境至关重要。以下是在代码审查期间有效管理分歧的关键策略。

\mySubsubsection{15.5.1}{明确改变的理由}

审阅者不仅要指出需要改进的地方,还要清楚地解释为什么需要进行更改。建议修改时,审阅者应提供基于最佳实践、性能考虑或与项目相关的设计原则的理由。包括指向编码标准、文章、文档或其他权威资源的链接,可以大大提高论点的清晰度和说服力。这种方法有助于被审阅者理解反馈背后的原因,使其更有可能看到建议更改的价值。

\mySubsubsection{15.5.2}{受审者的解释}

同样,如果受审者不同意评论或建议,他们也应该清楚地阐明自己的理由。这种解释应该详细说明为什么选择他们的方法或解决方案,并以项目内的相关技术论证、文档或先例为依据。通过提供合理的论据,受审者可以促进更明智的讨论,从而可以更好地理解或改进解决方案。

\mySubsubsection{15.5.3}{直接沟通}

如果争议涉及不止几条来回评论,建议将对话从书面评论转变为直接对话。这可以通过视频通话、电话或面对面会议进行,具体取决于可行的方式。直接沟通可以避免在基于文本的讨论中经常出现的沟通不畅和升级,这种讨论很快就会变得毫无成效且充满争议,就像 Reddit 等平台上的长篇帖子一样。

\mySubsubsection{15.5.4}{涉及更多视角}

如果评审者和被评审者之间无法达成解决方案,那么让其他观点参与进来可能会有所帮助。引入第三位工程师、产品经理、质量保证专家甚至架构师,可以提供新的见解并帮助调解分歧。这些各方可能会提供替代解决方案、折衷方法或基于更广泛的项目优先级和影响的决策。他们的意见对于打破僵局,并确保决策全面并与总体项目目标保持一致至关重要。

代码审查期间,有效的争议解决对于保持审查过程的建设性,并专注于提高代码库的质量至关重要。通过解释反馈背后的理由、鼓励直接沟通以及在必要时引入其他观点,团队可以有效解决分歧并保持积极的协作环境。这种方法不仅可以解决冲突,还可以增强团队协作解决未来挑战的能力。
