
分歧是代码审查过程中的自然组成部分。不同的观点可能导致方法、实施或最佳实践解释方面的冲突。有效处理这些争议对于维持高效的审查流程和健康的团队环境至关重要。以下是在代码审查期间有效管理分歧的关键策略。

\mySubsubsection{15.5.1}{明确改变的理由}

审阅者不仅要指出需要改进的地方,还要清楚地解释为什么需要进行更改。在建议修改时,审阅者应提供基于最佳实践、性能考虑或与项目相关的设计原则的理由。包括指向编码标准、文章、文档或其他权威资源的链接可以大大提高论点的清晰度和说服力。这种方法有助于被审阅者理解反馈背后的原因,使他们更有可能看到建议更改的价值。

\mySubsubsection{15.5.2}{受审者的解释}

同样,如果受审者不同意评论或建议,他们也应该清楚地阐明自己的理由。这种解释应该详细说明为什么选择他们的方法或解决方案,并以项目内的相关技术论证、文档或先例为依据。通过提供合理的论据,受审者可以促进更明智的讨论,从而可以更好地理解或改进解决方案。

\mySubsubsection{15.5.3}{直接沟通}

如果争议涉及不止几条来回评论,建议将对话从书面评论转变为直接对话。这可以通过视频通话、电话或面对面会议进行,具体取决于可行的方式。直接沟通可以避免在基于文本的讨论中经常出现的沟通不畅和升级,这种讨论很快就会变得毫无成效且充满争议,就像 R eddit 等平台上的长篇帖子一样。

\mySubsubsection{15.5.4}{涉及更多视角}

如果评审者和被评审者之间无法达成解决方案,那么让其他观点参与进来可能会有所帮助。引入第三位工程师、产品经理、质量保证专家甚至架构师可以提供新的见解并帮助调解分歧。这些各方可能会提供替代解决方案、折衷方法或基于更广泛的项目优先级和影响的决策。他们的意见对于打破僵局并确保决策全面并与总体项目目标保持一致至关重要。

在代码审查期间,有效的争议解决对于保持审查过程的建设性并专注于提高代码库的质量至关重要。通过解释反馈背后的理由、鼓励直接沟通以及在必要时引入其他观点,团队可以有效解决分歧并保持积极的协作环境。这种方法不仅可以解决冲突,还可以增强团队协作解决未来挑战的能力。
