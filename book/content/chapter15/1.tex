我们今天所理解的代码审查实践起源于 Michael Fagan,他在 20 世纪 70 年代中期开发了软件检查的正式流程。当时,软件工程通常是一项孤独的追求,各个开发人员都像孤胆牛仔一样负责编写、测试和审查自己的代码。这种方法导致项目之间的标准不一致,并且由于个人偏见和盲点得不到控制,导致被忽视错误的发生率更高。

认识到这种单独方法的局限性后, Fagan 引入了一种结构化的方法来系统地检查软件。他的流程不仅旨在发现错误,还旨在检查软件的整体设计和实现。这一转变标志着软件开发的重大变革,强调协作、细致检查和共担责任。通过让多名审阅者参与, Fagan 的方法为评估过程带来了不同的视角,从而提高了软件的审查力度和整体质量。























