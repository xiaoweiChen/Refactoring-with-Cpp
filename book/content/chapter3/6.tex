在本章中,我们探讨了 C++ 中出现错误代码的各种原因,以及缺乏对现代 C++ 实践的了解如何导致效率低下、容易出错或行为不明确。通过研究具体示例,我们强调了持续学习和适应以跟上 C++ 不断发展的功能的重要性。

我们首先讨论了使用原始指针和手动内存管理的缺陷,展示了 std::vector 等现代 C++ 实践如何消除手动内存管理的需要并降低内存泄漏的风险。强调了使用 std::unique\_ptr 实现独占所有权和使用 std::shared\_ptr 实现共享所有权的优势,同时突出了内存分配效率低下、不必要的复制和循环依赖等常见问题。

在 std::shared\_ptr 的上下文中,我们展示了使用 std::make\_shared 而不是构造函数的好处,以减少内存分配并提高性能。我们还解释了由于原子引用计数器,通过引用而不是值传递 std::shared\_ptr 所获得的效率。我们说明了循环依赖的问题以及如何使用 std::weak\_ptr 来中断循环并防止内存泄漏。我们还介绍了检查和使用 std::weak\_ptr 的正确方法,即锁定它并检查生成的 std::shared\_ptr 以确保线程安全。

讨论了如何高效使用移动语义,通过减少不必要的临时对象复制来优化性能。使用 std::mov e 和 std::make\_move\_iterator 可以显著提高程序性能。强调了 const 正确性的重要性,展示了如何将 const 应用于方法以明确意图并提高代码安全性。

我们讨论了使用 C 风格字符数组的危险,以及 std::string 如何简化字符串处理、减少错误并提供更好的内存管理。探讨了 C 风格数组的误用,并介绍了 std::array 作为更安全、更强大的替代方案。通过使用 std::array,我们可以避免未定义的行为并利用 C++ 标准库算法,例如 std::sort。

最后,我们讨论了 lambda 函数的正确使用方法,以及通过引用捕获变量的潜在缺陷,这可能会导致悬空引用。通过值捕获变量可确保 lambda 的使用安全。

通过这些示例,我们了解到采用现代 C++ 功能和最佳实践对于编写更安全、更高效、更易于维护的代码至关重要。通过及时了解最新标准并不断提高对 C++ 的理解,我们可以避免常见的陷阱并制作出高质量的软件。