
导致代码质量低下的主要原因之一是对 C++ 缺乏了解。 C++ 是一种复杂且不断发展的语言,具有广泛的功能,要跟上其最新标准需要不断学习。不熟悉现代 C++ 实践的开发人员可能会无意中编写出效率低下或容易出错的代码。本节探讨了对 C++ 理解上的差距如何导致各种问题,并使用示例来说明常见的陷阱。

以两位开发人员 Bob 和 Alice 为例。 Bob 对 C++ 的旧版本有丰富的经验,但尚未跟上最新更新,而 Alice 则精通现代 C++ 功能。

\mySubsubsection{3.4.1}{使用原始指针和手动内存管理}

Bob 可能会使用原始指针和手动内存管理,这是旧版 C++ 代码中的常见做法:

\begin{cpp}
void process() {
    int* data = new int[100];
    // ... perform operations on data
    delete[] data;
}
\end{cpp}

如果 delete[] 遗漏或与 new 错误匹配,此方法容易出现内存泄漏和未定义行为等错误。例如,如果在分配之后但在 delete[] 之前抛出异常,则内存将泄漏。熟悉现代 C++ 的 Alice 将使用 std::vector 来安全有效地管理内存:

\begin{cpp}
void process() {
    std::vector<int> data(100);
    // ... perform operations on data
}
\end{cpp}

使用 std::vector 无需手动管理内存,从而降低了内存泄漏的风险,并使代码更加健壮且更易于维护。

\mySubsubsection{3.4.2}{智能指针的错误使用}

Bob 尝试采用现代做法,但误用了 std::shared\_ptr,导致潜在的性能问题:

\begin{cpp}
std::shared_ptr<int> create() {
    std::shared_ptr<int> ptr(new int(42));
    return ptr;
}
\end{cpp}

这种方法涉及两个单独的分配:一个用于整数,另一个用于 std::shared\_ptr 的控制块。 Alice 知道 std::make\_shared 的好处,并使用它来优化内存分配:

\begin{cpp}
std::shared_ptr<int> create() {
    return std::make_shared<int>(42);
}
\end{cpp}

std::make\_shared 将分配合并为单个内存块,从而提高性能和缓存局部性。

\mySubsubsection{3.4.3}{高效使用移动语义}

Bob 可能不完全理解移动语义以及它们如何在处理临时对象时提高性能。考虑一个将元素附加到 std::vector 的函数:

\begin{cpp}
void append_data(std::vector<int>& target, const std::vector<int>&
source) {
    for (const int& value : source) {
        target.push_back(value); // Copies each element
    }
}
\end{cpp}

此方法涉及将每个元素从源复制到目标,这可能效率低下。 Alice 理解移动语义,将使用 std ::move 对此进行优化:

\begin{cpp}
void append_data(std::vector<int>& target, std::vector<int>&& source)
{
    for (int& value : source) {
        target.push_back(std::move(value)); // Moves each element
    }
}
\end{cpp}

通过使用 std::move, Alice 可以确保每个元素都被移动而不是复制,从而提高效率。此外,如果不再需要源, Alice 可能还会考虑对整个容器使用 std::move:

\begin{cpp}
void append_data(std::vector<int>& target, std::vector<int>&& source)
{
    target.insert(target.end(), std::make_move_iterator(source.
        begin()), std::make_move_iterator(source.end()));
}
\end{cpp}

这种方法可以有效地移动整个容器的元素,利用移动语义来避免不必要的复制。

\mySubsubsection{3.4.4}{误用 const 正确性}

Bob 可能会忽略 const 的正确性,从而导致潜在的错误和不清楚的代码:

\begin{cpp}
class MyClass {
public:
    int get_value() { return value; }
    void set_value(int v) { value = v; }
private:
    int value;
};
\end{cpp}

如果没有 const 正确性,则无法确定 get\_value 是否修改了对象的状态。 Alice 应用 const 正确性来澄清意图并提高安全性:

\begin{cpp}
class MyClass {
public:
    int get_value() const { return value; }
    void set_value(int v) { value = v; }
private:
    int value;
};
\end{cpp}

将 get\_value 标记为 const 可保证它不会修改对象,从而使代码更清晰并防止意外修改。

\mySubsubsection{3.4.5}{字符串处理效率低下}

Bob 可能会使用 C 风格的字符数组处理字符串,这可能会导致缓冲区溢出和复杂的代码:

\begin{cpp}
char message[100];
strcpy(message, "Hello, world!");
std::cout << message << std::endl;
\end{cpp}

这种方法容易出错,而且难以管理。 Alice 了解 std::string 的功能,因此简化了代码并避免了潜在的错误:

\begin{cpp}
std::string message = "Hello, world!";
std::cout << message << std::endl;
\end{cpp}

使用 std::string 可以提供自动内存管理和丰富的字符串操作函数,使代码更安全、更具表现力。

\mySubsubsection{3.4.6}{lambda 表达式的未定义行为}

C++11 中引入的 Lambda 函数提供了强大的功能,但如果使用不当,可能会导致未定义的行为。 Bob 可能会编写一个 lambda,通过引用捕获局部变量并返回它,从而导致悬空引用:

\begin{cpp}
auto create_lambda() {
    int value = 42;
    return [&]() { return value; };
}

auto lambda = create_lambda();
int result = lambda(); // Undefined behavior
\end{cpp}

Alice 了解这些风险,因此通过值捕获变量以确保其仍然有效:

\begin{cpp}
auto create_lambda() {
    int value = 42;
    return [=]() { return value; };
}

auto lambda = create_lambda();
int result = lambda(); // Safe
\end{cpp}

按值捕获避免了悬空引用的风险,并确保lambda仍然可以安全使用。

\mySubsubsection{3.4.7}{误解未定义的行为}

Bob 可能会无意中编写出因依赖未初始化的变量而导致未定义行为的代码:

\begin{cpp}
int sum() {
    int x;
    int y = 5;
    return x + y; // Undefined behavior: x is uninitialized
}
\end{cpp}

访问未初始化的变量可能会导致不可预测的行为和难以调试的问题。 Alice 了解初始化的重要性,确保所有变量都已正确初始化:

\begin{cpp}
int sum() {
    int x = 0;
    int y = 5;
    return x + y; // Defined behavior
}
\end{cpp}

正确初始化变量可以防止未定义的行为并使代码更可靠。

\mySubsubsection{3.4.8}{滥用 C 风格数组}

使用 C 样式数组可能会导致各种问题,例如缺乏边界检查和难以管理数组大小。请考虑以下示例,其中一个函数在堆栈上创建一个 C 数组并返回它:

\begin{cpp}
int* create_array() {
    int arr[5] = {1, 2, 3, 4, 5};
    return arr; // Undefined behavior: returning a pointer to a local array
}
\end{cpp}

返回指向本地数组的指针会导致未定义的行为,因为函数返回时数组超出范围。更安全的方法是使用 std::array,它可以安全地从函数返回。它提供了 size 方法,并与 C++ 算法兼容,例如 std::sort:

\begin{cpp}
std::array<int, 5> create_array() {
    return {1, 2, 3, 4, 5};
}
\end{cpp}

使用 std::array 不仅可以避免未定义行为,还可以增强安全性和与 C++ 标准库的互操作性。
例如,对数组进行排序变得非常简单:

\begin{cpp}
std::array<int, 5> arr = create_array();
std::sort(arr.begin(), arr.end());
\end{cpp}
