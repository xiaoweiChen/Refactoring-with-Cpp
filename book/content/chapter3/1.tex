开发人员检查现有代码时,可能会质疑为什么代码编写得不那么优雅或缺乏可扩展性。批评别人所做的工作很容易,但了解原始开发人员的情况至关重要。假设该项目最初是在一家初创公司开发的。这种情况下,重要的是要考虑到初创企业文化,非常强调快速的产品交付和超越竞争对手的需求。虽然这可能有利,但也可能导致开发出糟糕的代码。造成这种情况的主要原因是快速交付的压力,这可能导致开发人员为了赶上最后期限而偷工减料或跳过遵循必要的编码原则(例如,前几章中提到的 SOLID 原则)。这可能导致代码缺乏适当的文档、难以维护,并且容易出错。

此外,初创公司的资源有限、开发团队规模较小,这也加剧了对速度的需求,开发人员可能没有足够的人力来专注于优化和完善代码库。因此,代码可能会变得混乱和低效,导致性能下降和错误增多。

此外,创业文化注重快速交付,这可能会让开发人员难以跟上 C++ 的最新进展。这可能会导致过时的代码缺乏重要功能、使用效率低下或已弃用的功能,并且未针对性能进行优化。
