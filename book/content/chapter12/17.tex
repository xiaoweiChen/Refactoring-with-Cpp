
\mySubsubsection{12.17.1}{复杂相互作用}

当应用程序的组件进行复杂的交互(可能跨越不同的技术和平台)时,单元测试可能会失效。 E2E 测试对于确保这些组件的集体行为符合预期结果至关重要,尤其是在以下方面: 图中概述的架构代表一个典型的 Web 应用程序,具有多个相互连接的服务,每个服务在系统中发挥着不同的作用。

\myGraphic{0.5}{content/chapter12/images/1.png}{图 12.1 – E2E 测试}

前端有一个用户 UI,即用户与应用程序交互的图形界面。它旨在通过 API 网关向后端服务发送和接收数据。 API 网关充当中介,将来自用户 UI 的请求路由到适当的后端服务,并汇总响应以发送回 UI。

图中显示了几个后端服务:

\begin{itemize}
\item
帐户管理:此服务处理用户帐户,包括身份验证、配置文件管理和其他与用户相关的数据

\item
计费:负责管理账单信息、订阅和发票

\item
付款:处理金融交易,例如信用卡处理或与支付网关交互

\item
通知:向用户发送警报或消息,可能由帐户管理或计费服务中的某些事件触发
\end{itemize}

外部服务(可能是第三方应用程序或数据提供商)也可以与 API 网关交互,提供支持主应用程序的附加功能或数据。对于该系统的端到端测试,测试将模拟用户在用户界面上的操作,例如注册帐户或付款。

然后,测试将验证用户界面是否正确地通过 API 网关向后端服务发送了适当的请求。随后,测试将确认用户用户界面是否正确响应从后端收到的数据,确保整个工作流程(从用户界面到通知)按预期运行。这种全面的测试方法可确保每个组件单独运行并与系统的其余部分协同工作,为用户提供无缝体验。

总而言之,在应用程序组件进行复杂交互的情况下,尤其是当这些交互跨越不同的技术和平台时,必须考虑 E2E 测试。 E2E 测试可确保这些组件的集体行为与预期结果一致,从而对应用程序的功能和性能进行全面评估。以下是 E2E 有益的一些最常见情况:

\begin{itemize}
\item
多层应用程序:具有多层或多层级的应用程序(例如客户端-服务器架构)可从 E2E 测试中受益,以确保各层有效通信

\item
分布式系统:对于分布在不同环境或服务中的应用程序, E2E 测试可以验证这些分布式组件之间的数据流和功能
\end{itemize}

\mySubsubsection{12.17.2}{真实环境测试}

E2E 测试的主要优势之一是它能够复制接近生产环境的条件。这包括在实际硬件上测试应用程序、与真实数据库交互以及浏览真正的网络基础设施。此级别的测试对于以下方面至关重要:

\begin{itemize}
\item
性能验证:确保应用程序在预期的负载条件和用户流量下实现最佳性能

\item
安全保障:验证应用程序的安全措施在现实环境中是否有效,防范潜在的漏洞
\end{itemize}

E2E 测试是软件发布前的最后一个检查点,可全面评估应用程序的部署准备情况。通过模拟真实场景, E2E 测试可确保应用程序不仅符合其技术规格,而且还提供可靠且用户友好的体验,使其成为软件开发生命周期的重要组成部分。




