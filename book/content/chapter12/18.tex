
在确保软件项目全面测试的过程中,自动测试覆盖率跟踪工具发挥着关键作用。这些工具可以提供有关应用程序源代码在测试期间执行程度的宝贵见解,突出显示经过充分测试的区域以及可能需要额外关注的区域。

\mySubsubsection{12.18.1}{自动测试覆盖率跟踪工具及示例}

确保全面的测试覆盖率是可靠软件开发的基石。诸如用于 GNU 编译器集合 (GCC) 的 gcov 和用于 LLVM 项目的 llvm-cov 等工具可自动跟踪测试覆盖率,从而提供有关测试对代码的全面程度的重要见解。

\mySamllsectionNoContent{带有示例的工具概述}

C++ 项目中用于自动测试覆盖率跟踪的主要工具有两种:

\begin{itemize}
\item
gcov:作为 GCC 不可或缺的一部分, gcov 会在测试运行期间分析代码中的执行路径。
例如,在使用 g++ -fprofile- arcs -ftest-coverage example.cpp 编译 C++ example.cpp 文件后,运行相应的测试套件会生成覆盖率数据。之后运行 gcov example.cpp 会生成一份报告,详细说明每行代码的执行次数。

\item
llvm-cov: llvm-cov 在 LLVM 项目中发挥类似作用,它与 Clang 配合使用,提供详细的覆盖率报告。使用 clang++ -fprofile-instr-generate -fcoverage-mapping example.cpp 进行编译,然后使用 LLVM\_PROFILE\_FILE=" example.profraw" ./example 执行测试二进制文件,即可准备覆盖率数据。 llvm-profdata merge -sparse example.profraw -o example.profdata 后跟 llvm-cov show ./example -instr-profile=example.profdata,即可生成 example.cpp 的覆盖率报告。
\end{itemize}

\mySamllsectionNoContent{与 C++ 项目集成}

将这些工具集成到 C++ 项目中涉及使用覆盖标志编译源代码、执行测试以生成覆盖数据,然后分析这些数据以生成报告。

对于包含多个文件的项目,您可以使用以下命令进行编译:

\begin{shell}
g++ -fprofile-arcs -ftest-coverage file1.cpp file2.cpp -o testExecutable
\end{shell}

运行\verb|./testExecutable|来执行测试后,使用\verb|gcov file1.cpp file2.cpp| 为每个源文件生成覆盖率报告。

使用 llvm-cov 时,该过程类似,但针对 Clang 进行了定制。在编译和测试执行之后,将配置文件数据与 llvm-profdata 合并,并使用 llvm-cov 生成报告,可以全面了解测试覆盖率。

\mySamllsectionNoContent{解读覆盖率报告}

这些工具生成的覆盖率报告提供了几个指标:

\begin{itemize}
\item
行覆盖率:这表示已执行的代码行的百分比。例如, gcov 报告可能会显示"已执行的行数: 100 的 90.00\%,"表示在测试期间运行了 100 行中的 90 行。

\item
分支覆盖率:这提供了对条件语句测试的洞察。 gcov 报告中的一行,例如 Branches exe cuted:85.00\% of 40,表示已测试了所有分支的 85\%。

\item
函数覆盖率:显示被调用函数的百分比。 gcov 报告中的条目(例如 Functions executed:9 5.00\% of 20)表示测试期间调用了 95\% 的函数。
\end{itemize}

例如,简化的 gcov 报告可能如下所示:

\begin{shell}
File 'example.cpp'
Lines executed:90.00% of 100
Branches executed:85.00% of 40
Functions executed:95.00% of 20
\end{shell}

类似地, llvm-cov 报告提供详细的覆盖率指标,以及覆盖的具体行和分支,增强了查明需要额外测试的区域的能力。

这些报告通过突出显示未经测试的代码路径和函数来指导开发人员提高测试覆盖率,但它们不应成为测试质量的唯一指标。设计不良的测试的高覆盖率会给人一种虚假的安全感。有效使用这些工具不仅需要追求高覆盖率,还需要确保测试有意义并反映实际使用场景。

\mySubsubsection{12.18.2}{利用命中图增强测试覆盖率分析}

命中图由 gcov 和 llvm-cov 等测试覆盖率跟踪工具生成,可以详细了解测试如何执行代码,为旨在提高测试覆盖率的开发人员提供详细指南。这些命中图不仅仅是简单的百分比指标,还精确显示了测试期间执行了哪些代码行以及执行了多少次,从而可以采用更明智的方法来增强测试套件。

\mySamllsectionNoContent{了解命中图}

命中图本质上是对源代码的详细注释,每行都附有执行计数,指示测试运行该特定行的次数。这种详细程度不仅有助于识别未经测试的代码部分,还有助于识别可能过度测试或需要更多不同测试场景的区域。

gcov 生成的 .gcov 文件和 llvm-cov 生成的带注释的源代码提供了这些命中图,清晰地展现了行级测试覆盖率。

\begin{shell}
-: 0:Source:example.cpp
-: 0:Graph:example.gcno
-: 0:Data:example.gcda
-: 0:Runs:3
-: 0:Programs:1
3: 1:int main() {
-: 2: // Some comment
2: 3: bool condition = checkCondition();
1: 4: if (condition) {
1: 5: performAction();
...
\end{shell}

在这个例子中,第 3 行(bool condition = checkCondition();)被执行了两次,而 if 语句中的 perfo rmAction(); 行被执行了一次,表明在其中一次测试运行中条件为真。

与 gcov 类似,在使用 -fprofile-instr-generate -fcoverage-mapping 标志使用 clang++ 进行编译并执行测试后, llvm-cov 可以使用 llvm-cov show 命令生成命中图,其中 -instr-profile 标志指向生成的配置文件数据。例如, llvm-cov show ./example -instr-profile=example.profdata exampl e.cpp 输出带有执行计数的注释源代码。

输出将类似于以下内容:

\begin{shell}
example.cpp:
int main() {
    | 3| // Some comment
    | 2| bool condition = checkCondition();
    | 1| if (condition) {
    | 1| performAction();
    ...
\end{shell}

在这里,每一行的执行计数都是前缀,一目了然地提供了测试覆盖率的清晰图像。

\mySamllsectionNoContent{利用命中图改进测试}

通过检查命中图,开发人员可以识别未被任何测试用例覆盖的代码部分,这些部分由执行计数为零表示。这些区域表示未检测到的错误的潜在风险,应优先进行额外测试。相反,执行计数异常高的代码行可能表示测试冗余或过于集中的区域,这表明有机会多样化测试场景或将测试工作重新集中在代码库中覆盖较少的部分。

将命中图分析纳入常规开发工作流程可鼓励采取积极主动的方法来维护和提高测试覆盖率,确保测试保持有效并与不断发展的代码库保持一致。与所有测试策略一样,目标不仅仅是实现高覆盖率,还要确保测试套件在各种情况下全面验证软件的功能和可靠性。

随着集成开发环境 (IDE) 插件的出现,将覆盖率可视化直接集成到编码环境中,将命中图纳入开发工作流程变得更加容易。一个值得注意的例子是 Markis Taylor 为 Visual Studio Code ( VSCode) 开发的"代码覆盖率"插件。此插件将命中图叠加到 VSCode 编辑器中的源代码上,提供有关测试覆盖率的即时、视觉反馈。

"代码覆盖率"插件可处理由 gcov 或 llvm-cov 等工具生成的覆盖率报告,并在 VSCode 中以可视化方式注释源代码。测试覆盖的代码行会突出显示(通常为绿色),而未覆盖的代码行则标记为红色。这种直观的可视化表示使开发人员无需离开编辑器或浏览外部覆盖率报告即可快速识别未测试的代码区域。

\mySubsubsection{12.18.3}{代码覆盖率的建议}

代码覆盖率是软件测试领域的一个重要指标,可以洞悉测试套件对代码库的执行程度。对于 C++ 项目,利用 GCC 的 gcov 和 LLVM 项目的 llvm-cov 等工具可以提供详细的覆盖率分析。这些工具不仅可以跟踪单元测试的覆盖率,还可以跟踪端到端测试的覆盖率,从而可以全面评估不同测试级别的测试覆盖率。

强大的测试策略包括集中式单元测试(单独验证各个组件)和更广泛的 E2E 测试(评估整个系统的功能)的组合。通过使用 gcov 或 llvm-cov,团队可以汇总两种测试类型的覆盖率数据,从而提供项目测试覆盖率的整体视图。这种组合方法有助于识别代码中测试不足或根本没有测试的区域,从而指导提高测试套件有效性的努力。

建议密切关注代码覆盖率指标,并努力防止覆盖率百分比下降。覆盖率下降可能表明新代码未经充分测试就被添加,从而可能将未检测到的错误引入系统。为了降低这种风险,团队应将覆盖率检查集成到其持续集成 (CI) 管道中,确保及时发现并解决任何降低覆盖率的更改。

定期专门分配时间来提高测试覆盖率是有益的,特别是在被确定为关键或有风险的领域。这可能涉及为以前被忽略的复杂逻辑、极端情况或错误处理路径编写额外的测试。投资于覆盖率改进计划不仅可以提高软件的可靠性,而且从长远来看还有助于建立更易于维护和更强大的代码库。












