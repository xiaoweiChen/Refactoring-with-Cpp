
E2E 测试是一种全面的测试方法,可从头到尾评估应用程序的功能和性能。与隔离和测试单个组件或代码单元的单元测试不同, E2E 测试将应用程序作为一个整体进行检查,模拟真实的用户场景。此方法可确保应用程序的所有各个组件(包括其接口、数据库、网络和其他服务)协调工作,以提供所需的用户体验。

\mySubsubsection{12.16.1}{E2E 测试框架}

鉴于 E2E 测试通常涉及从外部与应用程序交互,其并不局限于编写应用程序的语言。对于 C++ 应用程序(可能是更大生态系统的一部分或用作后端系统),可以使用不同语言的各种框架进行 E2E 测试。一些流行的 E2E 测试框架包括:

\begin{itemize}
\item
Selenium: Selenium 主要用于 Web 应用程序,可以自动化浏览器来模拟用户与 Web 界面的交互,使其成为 E2E 测试的多功能工具

\item
Cypress:另一个强大的 Web 应用程序工具, Cypress 提供更现代、更适合开发人员的 E2E 测试方法,具有丰富的调试功能和强大的 API

\item
Postman:对于公开 RESTful API 的应用程序, Postman 允许进行全面的 API 测试,确保应用程序的端点在各种条件下都能按预期运行
\end{itemize}

\mySubsubsection{12.16.2}{何时使用 E2E 测试}

E2E 测试在应用程序组件必须在复杂的工作流中交互(通常涉及多个系统和外部依赖项) 的情况有价值。它对于以下情况至关重要:

\begin{itemize}
\item
测试复杂的用户工作流程: E2E 测试在验证跨多个应用程序组件的用户旅程方面表现出色,确保从用户的角度获得无缝体验

\item
集成场景:当应用程序与外部系统或服务交互时, E2E 测试将验证这些集成是否按预期工作,并发现单独情况下可能不明显的问题

\item
关键路径测试:对于对应用程序核心功能至关重要的特性和路径, E2E 测试可确保在实际使用条件下的可靠性和性能
\end{itemize}


