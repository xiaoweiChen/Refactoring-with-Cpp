确定单元测试的最佳候选对象对于制定可靠的测试策略至关重要,单元测试在应用于适合隔离和细粒度验证的代码库时效果很好。以下是一些关键示例和建议:

具有明确边界和职责的类和函数,是单元测试的主要候选对象。这些组件理想情况下应体现单一职责原则,处理应用程序功能的特定方面。测试这些独立的单元可以精确验证其行为,确保在各种条件下都能正确执行预期任务。

纯函数完全依赖于其输入参数,不会产生副作用,是单元测试的绝佳目标,确定性(即给定的输入始终会产生相同的输出)使它们易于测试和验证。纯函数通常出现在实用程序库、数学计算和数据转换操作中。

通过定义良好的接口与依赖项交互的组件更易于测试,尤其是当这些依赖项可以模拟或保留。这有助于单独测试组件,专注于其逻辑而不是依赖项的实现细节。

业务逻辑层封装了应用程序的核心功能和规则,通常非常适合单元测试。此层通常涉及计算、数据处理和决策,这些可以独立于用户界面和外部系统进行测试。

虽然应用程序的许多方面都适合进行单元测试,但谨慎起见,要认识到会带来挑战的场景。需要与外部资源(如数据库、文件系统和网络服务)进行复杂交互的组件可能难以有效模拟,或者由于依赖外部状态或行为,而导致测试不稳定。虽然模拟可以模拟其中一些交互,但在单元测试的背景下,复杂性和开销可能并不总是物有所值。

尽管单元测试对于验证单个组件非常有用,但它们也有局限性,尤其是在集成和端到端交互方面。对于本质上难以隔离或需要复杂外部交互的代码,端到端 (E2E) 测试至关重要。 E2E 测试模拟了真实的使用场景,涵盖了从用户界面到后端系统和外部集成的流程。下一节中,将深入研究 E2E 测试,探索其在补充单元测试和全面覆盖应用程序功能的作用。












