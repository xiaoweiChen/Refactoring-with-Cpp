基于编译器的清理器和 Valgrind 为调试和分析过程带来了明显的优势和挑战。

ASan、 TSan 和 UBSan 等基于编译器的工具通常更易于访问,也更易于集成到开发工作流程中。它们在引入的性能开销方面“更便宜”,并且配置和使用相对简单。这些清理工具直接集成到编译过程中,方便开发人员定期使用。其主要优势在于能够在开发阶段提供即时反馈,在编写和测试代码时发现错误和问题。但是,由于这些工具在运行时执行分析,其有效性与测试覆盖范围直接相关。测试越全面,动态分析就越有效,因为只分析执行的代码路径。这方面突出了全面测试的重要性:测试覆盖率越高,这些工具可能发现的问题就越多。

另一方面, Valgrind 提供了更强大、更彻底的分析,能够检测更广泛的问题,特别是在内存管理和线程方面。它的工具套件(Memcheck、 Helgrind、 DRD、 Cachegrind、 Callgrind、 Massif 和 DHAT)提供了对程序性能和行为各个方面的全面分析。然而,这种能力是有代价的:与基于编译器的工具相比, Valgrind 通常使用起来更复杂,并且会带来显著的性能开销。

选择使用 Valgrind 还是基于编译器的清理程序,通常取决于项目的具体需求和所针对的问题。虽然 Valgrind 的广泛诊断功能可以深入了解程序,但基于编译器的清理程序易于使用且性能成本较低,所以更适合在 CI 管道中定期使用。

总之,虽然基于编译器的工具和 Valgrind 在动态分析领域都有一席之地,但它们在诊断、 易用性和性能影响方面的差异,使它们适合软件开发过程的不同阶段和方面。强烈建议将这些工具作为常规 CI 管道的一部分使用,可以尽早发现和解决问题,从而大大提高软件的整体质量和稳健性。后续章节将深入探讨用于测量测试覆盖率的工具,深入了解代码库的测试效率,从而补充动态分析过程。
