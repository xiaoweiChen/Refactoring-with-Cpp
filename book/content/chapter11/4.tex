基于编译器的清理器和 Valgrind 为调试和分析过程带来了明显的优势和挑战。

ASan、 TSan 和 UBSan 等基于编译器的工具通常更易于访问,也更易于集成到开发工作流程中。它们在引入的性能开销方面"更便宜",并且配置和使用相对简单。这些清理工具直接集成到编译过程中,方便开发人员定期使用。它们的主要优势在于能够在开发阶段提供即时反馈,在编写和测试代码时发现错误和问题。但是,由于这些工具在运行时执行分析,因此它们的有效性与测试覆盖范围直接相关。测试越全面,动态分析就越有效,因为只分析执行的代码路径。这方面突出了全面测试的重要性:测试覆盖率越高,这些工具可能发现的问题就越多。

另一方面, Valgrind 提供了更强大、更彻底的分析,能够检测更广泛的问题,特别是在内存管理和线程方面。它的工具套件(Memcheck、 Helgrind、 DRD、 Cachegrind、 Callgrind、 Massif 和 DHAT)提供了对程序性能和行为各个方面的全面分析。然而,这种能力是有代价的:与基于编译器的工具相比, Valgrind 通常使用起来更复杂,并且会带来显著的性能开销。

选择使用 Valgrind 还是基于编译器的清理程序通常取决于项目的具体需求和所针对的问题。虽然 Valgrind 的广泛诊断功能可以深入了解程序,但基于编译器的清理程序易于使用且性能成本较低,因此更适合在 CI 管道中定期使用。

总之,虽然基于编译器的工具和 Valgrind 在动态分析领域都有一席之地,但它们在诊断、 易用性和性能影响方面的差异使它们适合软件开发过程的不同阶段和方面。强烈建议将这些工具作为常规 CI 管道的一部分使用,因为它可以尽早发现和解决问题,从而大大提高软件的整体质量和稳健性。后续章节将深入探讨用于测量测试覆盖率的工具,深入了解代码库的测试效率,从而补充动态分析过程。
