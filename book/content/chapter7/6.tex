在本章中,我们探讨了高级 C++ 编程的复杂性,重点关注类设计、继承和模板。从有效类设计的原则开始,强调封装最少的必要功能和数据以实现更好的模块化和可维护性的重要性。通过实际示例,强调了好的和坏的设计实践。

接下来讨论继承,研究了它的优点,例如代码重用、层次结构和多态性,同时也解决了它的缺点,包括紧密耦合、 复杂的层次结构和潜在的违背 LSP。我们提供了有关何时使用继承,以及何时考虑组合等替代方案的指导。

在模板部分,我们深入研究了它们在实现泛型编程方面的作用,允许使用灵活且可重用的组件来处理任何数据类型。论了模板的优点,例如代码可重用性、 类型安全性和性能优化,但也指出了它们的缺点,包括增加编译时间、代码膨胀以及理解和调试模板繁重代码的复杂性。在整个讨论过程中,强调了在使用这些强大功能时需要仔细考虑和理解,以确保 C++ 应用程序的健壮性和可维护性。

在下一章中,我们将重点转移到 API 设计上,探索在 C++ 中创建清晰、高效且用户友好的界面的最佳实践。
