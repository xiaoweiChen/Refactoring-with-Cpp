模板和泛型编程是 C++ 的关键特性,可以创建灵活且可重用的组件。虽然本章概述了这些强大的工具,但需要注意的是,模板(尤其是模板元编程)的主题非常广泛,足以写满整本书。对于那些寻求深入探索的人,建议阅读有关 C++ 模板和元编程的专用资源。

\mySubsubsection{7.3.1}{模板有何用途?}

在需要对不同类型的数据执行类似操作的情况下,模板特别有用。它们允许您编写适用于任何类型的单段代码。以下小节概述了一些常见用例和示例。

\mySubsubsection{7.3.2}{通用算法}

算法可以对不同类型进行操作,而无需为每种类型重写代码。例如,标准库的 std::sort 函数可以对任何类型的元素进行排序,只要这些元素可以进行比较:

\begin{cpp}
#include <algorithm>
#include <vector>
#include <iostream>

template <typename T>
void print(const std::vector<T>& vec) {
    for (const T& elem : vec) {
        std::cout << elem << " ";
    }
    std::cout << std::endl;
}

int main() {
    std::vector<int> int_vec = {3, 1, 4, 1, 5};
    std::sort(int_vec.begin(), int_vec.end());
    print(int_vec); // Outputs: 1 1 3 4 5

    std::vector<std::string> string_vec = {"banana", "apple", "cherry"};
    std::sort(string_vec.begin(), string_vec.end());
    print(string_vec); // Outputs: apple banana cherry

    return 0;
}
\end{cpp}

\mySubsubsection{7.3.3}{容器类}

模板在标准库中被广泛用于诸如 std::vector、 std::list 和 std::map 等容器,它们可以存储任何类型的元素:

\begin{cpp}
#include <vector>
#include <iostream>

int main() {
    std::vector<int> int_vec = {1, 2, 3};
    std::vector<std::string> string_vec = {"hello", "world"};

    for (int val : int_vec) {
        std::cout << val << " ";
    }
    std::cout << std::endl;

    for (const std::string& str : string_vec) {
        std::cout << str << " ";
    }
    std::cout << std::endl;

    return 0;
}
\end{cpp}

如果不使用模板,开发人员使用集合的选项将仅限于为每种类型的集合创建单独的类(例如, IntVector、 StringVector 等),或者要求使用公共基类,这将需要类型转换并失去类型安全性,例如:

\begin{cpp}
class BaseObject {};

class Vector {
    public:
        void push_back(BaseObject* obj);
};
\end{cpp}

另一种选择是存储一些空指针,并在检索它们时将它们转换为所需的类型,但这种方法更容易出错。

标准库使用智能指针模板,例如 std::unique\_ptr 和 std::shared\_ptr,它们管理动态分配对象的生命周期:

\begin{cpp}
#include <memory>
#include <iostream>

int main() {
    std::unique_ptr<int> ptr = std::make_unique<int>(42);
    std::cout << "Value: " << *ptr << std::endl; // Outputs: Value: 42

    std::shared_ptr<int> shared_ptr = std::make_shared<int>(100);
    std::cout << "Shared Value: " << *shared_ptr << std::endl; // Outputs: Shared Value: 100

    return 0;
}
\end{cpp}

模板通过允许编译器在模板实例化期间检查类型来确保类型安全,从而减少运行时错误:

\begin{cpp}
template <typename T>
T add(T a, T b) {
    return a + b;
}

int main() {
    std::cout << add<int>(5, 3) << std::endl; // Outputs: 8
    std::cout << add<double>(2.5, 3.5) << std::endl; // Outputs: 6.0
    return 0;
}
\end{cpp}

\mySubsubsection{7.3.4}{模板的工作原理}

C++ 中的模板不是实际的代码,而是代码生成的蓝图。当使用特定类型实例化模板时,编译器会生成该模板的具体实例,并使用指定类型替换模板参数。

\mySamllsectionNoContent{函数模板}

\begin{cpp}
template <typename T>
T add(T a, T b) {
    return a + b;
}

int main() {
    std::cout << add<int>(5, 3) << std::endl; // Outputs: 8
    std::cout << add<double>(2.5, 3.5) << std::endl; // Outputs: 6.0
    return 0;
}
\end{cpp}

模板实例化后实际生成的函数将是这样的(取决于编译器):

\begin{cpp}
int addInt(int a, int b) {
    return a + b;
}

double addDouble(double a, double b) {
    return a + b;
}
\end{cpp}

\mySamllsectionNoContent{类模板}

类模板为可以对不同数据类型进行操作的类定义了一种模式:

\begin{cpp}
template <typename T>
class Box {
private:
    T content;
public:
    void set_content(const T& value) {
        content = value;
    }

    T get_content() const {
        return content;
    }
};

int main() {
    Box<int> intBox;
    intBox.set_content(123);
    std::cout << intBox.get_content() << std::endl; // Outputs: 123

    Box<std::string> stringBox;
    stringBox.set_content("Hello Templates!");
    std::cout << stringBox.get_content() << std::endl; // Outputs: Hello Templates!

    return 0;
}
\end{cpp}

模板实例化后实际生成的类将是这样的(取决于编译器):

\begin{cpp}
class BoxInt { /*Box<int>*/ };
class BoxString { /*Box<int>*/ }
\end{cpp}




