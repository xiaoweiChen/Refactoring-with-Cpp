本章中,将探讨用于创建结构良好且可维护代码的主要软件设计原则。其中最重要的原则是 SOLID,即单一责任原则、开放封闭原则、里氏替换原则、接口隔离原则和依赖倒置原则。这些原则旨在帮助开发人员创建易于理解、测试和修改的代码。我们还将讨论抽象级别的重要性,抽象级别是将复杂系统分解为更小、更易于管理的实践。

此外,还将探讨副作用和可变性的概念,以及它们如何影响软件的整体质量。通过理解和应用这些原则,开发人员可以创建更强大、更可靠和可扩展的软件。
