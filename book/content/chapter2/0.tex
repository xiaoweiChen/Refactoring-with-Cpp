在本章中,我们将探讨用于创建结构良好且可维护代码的主要软件设计原则。其中最重要的原则之一是 SOLID,即单一责任原则、开放封闭原则、里氏替换原则、接口隔离原则和依赖倒置原则。这些原则旨在帮助开发人员创建易于理解、测试和修改的代码。我们还将讨论抽象级别的重要性,抽象级别是将复杂系统分解为更小、更易于管理的部分的实践。
此外,我们将探讨副作用和可变性的概念以及它们如何影响软件的整体质量。通过理解和应用这些原则,开发人员可以创建更强大、更可靠和可扩展的软件。