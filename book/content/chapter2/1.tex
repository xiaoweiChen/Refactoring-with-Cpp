
SOLID 是 Robert C. Martin 于 2000 年在其著作《敏捷软件开发、原则、模式和实践》中首次提出的一套原则。 Robert C. Martin,又名 Uncle Bob,是一名软件工程师、作家和演讲者。他被认为是软件开发行业最具影响力的人物之一,因其在 SOLID 原则方面的工作以及在面向对象编程领域的贡献而闻名。 Martin 从事软件开发已有 40 多年,参与过各种各样的项目,从小型系统到大型企业系统。他还是一位著名的演讲者,曾在世界各地的许多会议和活动中发表过关于软件开发的演讲。他是敏捷方法的倡导者,在《敏捷宣言》的制定中发挥了重要作用。 SOLID 原则的制定旨在通过推广良好的设计实践来帮助开发人员创建更易于维护和可扩展的代码。这些原则基于 Martin 作为软件开发人员的经验以及他的观察:许多软件项目存在设计不佳的问题,这使得它们随着时间的推移难以理解、更改和维护。

SOLID 原则旨在成为面向对象软件设计的指南,其理念是软件应该易于理解、更改和随时间扩展。这些原则旨在与其他软件开发实践(如测试驱动开发和持续集成)结合使用。通过遵循 SOLID 原则,开发人员可以创建更强大、更不容易出错且更易于维护的代码。

\mySubsubsection{2.1.1}{单一责任原则}

单一职责原则 (SRP) 是面向对象软件设计的五项 SOLID 原则之一。它规定一个类应该只有一个更改原因,这意味着一个类应该只有一个职责。此原则旨在促进易于理解、更改和测试的代码。

SRP 背后的理念是,类应该具有单一且明确定义的用途。这样可以更轻松地理解类的行为,并且使对类的更改不太可能产生意外后果。当类只有一个职责时,它也不太可能出现错误,并且更容易为其编写自动化测试。

应用 SRP 可以成为一种改进软件系统设计的有效方法,使其更加模块化,更易于理解。通过遵循这一原则,开发人员可以创建小巧、专注且易于推理的类。这使得随着时间的推移,软件的维护和改进变得更加容易。

让我们看一个支持通过网络发送多种消息类型的消息系统。该系统有一个 Message 类,用于接收发送者和接收者的 ID 以及要发送的原始数据。此外,它还支持将消息保存到磁盘并通过 send 方法发送自身:

\begin{cpp}
class Message {
public:
    Message(SenderId sender_id, ReceiverId receiver_id,
            const RawData& data)
        : sender_id_{sender_id},
          receiver_id_{receiver_id}, raw_data_{data} {}

    SenderId sender_id() const { return sender_id_; }
    ReceiverId receiver_id() const { return receiver_id_; }

    void save(const std::string& file_path) const {
        // serializes a message to raw bytes and saves
        // to file system
    }

    std::string serialize() const {
        // serializes to JSON
        return {"JSON"};
    }

    void send() const {
        auto sender = Communication::get_instance();
        sender.send(sender_id_, receiver_id_, serialize());
    }

private:
    SenderId sender_id_;
    ReceiverId receiver_id_;
    RawData raw_data_;
};
\end{cpp}

Message 类负责多项任务,例如从文件系统保存消息或向文件系统保存消息、序列化数据、 发送消息以及保存发送者和接收者的 ID 和原始数据。最好将这些任务分成不同的类或模块。

Message类只负责存储数据,并将其序列化为JSON格式:

\begin{cpp}
class Message {
public:
    Message(SenderId sender_id, ReceiverId receiver_id,
            const RawData& data)
        : sender_id_{sender_id},
            receiver_id_{receiver_id}, raw_data_{data} {}

    SenderId sender_id() const { return sender_id_; }
    ReceiverId receiver_id() const { return receiver_id_; }

    std::string serialize() const {
        // serializes to JSON
        return {"JSON"};
    }

private:
    SenderId sender_id_;
    ReceiverId receiver_id_;
    RawData raw_data_;
};
\end{cpp}

保存方法可以提取到单独的 MessageSaver 类中,该类具有单一职责:

\begin{cpp}
class MessageSaver {
public:
    MessageSaver(const std::string& target_directory);
    void save(const Message& message) const;
};
\end{cpp}

并且发送方法在专用的 MessageSender 类中实现。这三个类都具有单一而明确的职责,并且对其中任何一个类的任何进一步更改都不会影响其他类。这种方法允许隔离代码库中的更改。这在需要长时间编译的复杂系统中至关重要。

总之, SRP 指出,一个类应该只有一个更改原因,这意味着一个类应该只有一个责任。此原则旨在促进易于理解、更改和测试的代码,并有助于创建更模块化、更易于维护和可扩展的代码库。通过遵循此原则,开发人员可以创建小巧、专注且易于推理的类。

\mySamllsectionNoContent{SRP 的其他应用}

SRP 不仅可以应用于类,还可以应用于更大的组件,例如应用程序。在架构层面, SRP 通常以微服务架构的形式实现。微服务的理念是将软件系统构建为一组小型、独立的服务,这些服务通过网络相互通信,而不是将其构建为单片应用程序。每个微服务负责特定的业务功能,并且可以独立于其他服务进行开发、部署和扩展。这允许更大的灵活性、可扩展性和易于维护,因为对一个服务的更改不会影响整个系统。微服务还可以实现更敏捷的开发过程,因为团队可以并行处理不同的服务,并且还可以采用更细粒度的安全性、监控和测试方法,因为每个服务都可以单独处理。

\mySubsubsection{2.1.2}{开放封闭原则}

开放-封闭原则规定模块或类应该对扩展开放,但对修改封闭。换句话说,应该可以在不修改现有代码的情况下向模块或类添加新功能。这一原则有助于提高软件的可维护性和灵活性。 C++ 中这一原则的一个例子是使用继承和多态性。基类可以编写为能够由派生类扩展的功能,从而允许在不修改基类的情况下添加新功能。另一个示例是使用接口或抽象类为一系列相关类定义契约,从而允许添加符合契约的新类而无需修改现有代码。

开放封闭原则可用于改进我们的消息发送组件。当前版本仅支持一种消息类型。如果我们想添加更多数据,我们需要更改 Message 类:添加字段,将消息类型作为附加变量保存,更不用说基于此变量进行序列化。为了避免现有代码的更改,让我们将 Message 类重写为纯虚拟的,并提供 serialize 方法:

\begin{cpp}
class Message {
public:
    Message(SenderId sender_id, ReceiverId receiver_id)
        : sender_id_{sender_id}, receiver_id_{receiver_id} {}

    SenderId sender_id() const { return sender_id_; }
    ReceiverId receiver_id() const { return receiver_id_; }
    virtual std::string serialize() const = 0;

private:
    SenderId sender_id_;
    ReceiverId receiver_id_;
};
\end{cpp}

现在,假设我们需要添加另外两种消息类型:支持启动延迟的"启动"消息(通常用于调试目的)和支持停止延迟的"停止"消息(可用于调度);它们可以按如下方式实现:

\begin{cpp}
class StartMessage : public Message {
public:
    StartMessage(SenderId sender_id, ReceiverId receiver_id,
                 std::chrono::milliseconds start_delay)
        : Message{sender_id, receiver_id},
            start_delay_{start_delay} {}

    std::string serialize() const override {
        return {"naive serialization to JSON"};
    }

private:
    const std::chrono::milliseconds start_delay_;
};

class StopMessage : public Message {
public:
    StopMessage(SenderId sender_id, ReceiverId receiver_id,
                std::chrono::milliseconds stop_delay)
        : Message{sender_id, receiver_id},
        stop_delay_{stop_delay} {}

    std::string serialize() const override {
        return {"naive serialization to JSON"};
    }
private:
    const std::chrono::milliseconds stop_delay_;
};
\end{cpp}

请注意,这些实现都不需要更改其他类,并且每个实现都提供了自己的 serialize 方法版本。
MessageSender 和 MessageSaver 类不需要进行额外的调整来支持新的消息类层次结构。但是,我们也将更改它们。主要原因是使它们可扩展而无需进行更改。例如,消息不仅可以保存到文件系统,还可以保存到远程存储。在这种情况下, MessageSaver 变为纯虚拟的:

\begin{cpp}
class MessageSaver {
public:
    virtual void save(const Message& message) const = 0;
};
\end{cpp}

负责保存到文件系统的实现是从 MessageSaver 派生的类:

\begin{cpp}
class FilesystemMessageSaver : public MessageSaver {
    public:
    FilesystemMessageSaver(const std::string&
        target_directory);
    void save(const Message& message) const override;
};
\end{cpp}

远程存储保存器是层次结构中的另一个类:

\begin{cpp}
class RemoteMessageSaver : public MessageSaver {
public:
    RemoteMessageSaver(const std::string&
        remote_storage_address);
    void save(const Message& message) const override;
};
\end{cpp}

\mySubsubsection{2.1.3}{Liskov替代原则}

里氏替换原则 (LSP) 是面向对象编程中的一个基本原则,该原则规定超类的对象应该能够被子类的对象替换,而不会影响程序的正确性。该原则也称为里氏原则,以首次提出该原则的 Barbara Liskov 的名字命名。 LSP 基于继承和多态的思想,其中子类可以继承其父类的属性和方法,并且可以与其互换使用。

为了遵循 LSP,子类必须与其父类"行为兼容"。这意味着它们应该具有相同的方法签名并遵循相同的契约,例如输入和输出类型和范围。此外,子类中方法的行为不应违反父类中建立的任何契约。

让我们考虑一种新的消息类型 InternalMessage,它不支持 serialize 方法。人们可能会尝试按以下方式实现它:

\begin{cpp}
class InternalMessage : public Message {
public:
    InternalMessage(SenderId sender_id, ReceiverId
                    receiver_id)
        : Message{sender_id, receiver_id} {}

    std::string serialize() const override {
        throw std::runtime_error{"InternalMessage can't be
            serialized!"};
    }
};
\end{cpp}

在上面的代码中, InternalMessage 是 Message 的子类型,但无法序列化,而是会引发异常。
这种设计存在问题,原因如下:

\begin{itemize}
\item
它违反了里氏替换原则:根据里氏替换原则,如果 InternalMessage 是 Message 的子类型,那么我们应该能够在需要 Message 的任何地方使用 InternalMessage,而不会影响程序的正确性。通过在序列化方法中抛出异常,我们违反了这一原则。

\item
调用者必须处理异常: serialize 的调用者必须处理异常,这在处理其他 Message 类型时可能没有必要。这会增加调用者代码的复杂性和错误的可能性。

\item
程序崩溃:如果异常没有得到妥善处理,可能会导致程序崩溃,这当然不是一个理想的结果。
\end{itemize}

我们可以返回一个空字符串而不是抛出异常,但这仍然违反了 LSP,因为序列化方法应该返回序列化的消息,而不是空字符串。这还会带来歧义,因为不清楚空字符串是成功序列化没有数据的消息的结果,还是序列化 InternalMessage 失败的结果。

更好的方法是将 Message 和 SerializableMessage 的关注点分开,其中只有 SerializableMessag es 具有序列化方法:

\begin{cpp}
class Message {
public:
    virtual ~Message() = default;
    // other common message behaviors
};

class SerializableMessage : public Message {
public:
    virtual std::string serialize() const = 0;
};

class StartMessage : public SerializableMessage {
    // ...
};

class StopMessage : public SerializableMessage {
    // ...
};

class InternalMessage : public Message {
    // InternalMessage doesn't have serialize method now.
};
\end{cpp}

在这个修正的设计中,基础 Message 类不包含 serialize 方法,并引入了包含此方法的新 Seri alizableMessage 类。这样,只有可以序列化的消息才会从 SerializableMessage 继承,并且我们遵守 LSP。

遵守 LSP 可使代码更加灵活且易于维护,因为它支持使用多态性,并允许用子类的对象替换类的对象,而不会影响程序的整体行为。这样,程序可以利用子类提供的新功能,同时保持与超类相同的行为。

\mySubsubsection{2.1.4}{接口隔离原则}

接口隔离原则 (ISP) 是面向对象编程中的一项原则,它规定类应该只实现它使用的接口。换句话说,它建议接口应该是细粒度的和特定于客户端的,而不是拥有一个单一的、庞大的、包罗万象的接口。 ISP 基于这样的理念:最好拥有许多小接口,每个接口定义一组特定的方法,而不是拥有一个定义许多方法的单一大型接口。

ISP 的一个主要优点是它促进了更加模块化和灵活的设计,因为它允许创建针对客户端特定需求量身定制的接口。这样,它减少了客户端需要实现的不必要方法的数量,同时也降低了客户端依赖其不需要的方法的风险。

从 MessagePack 或 JSON 文件创建示例消息时,可以观察到 ISP 的示例。按照最佳实践,我们将创建一个提供两种方法的接口,即 from\_message\_pack 和 from\_json。

当前的实现需要实现这两种方法,但如果某个类不需要支持这两种选项怎么办?接口越小越好。 MessageParser 接口将拆分为两个单独的接口,每个接口都需要实现 JSON 或 Message Pack:

\begin{cpp}
class JsonMessageParser {
public:
    virtual std::unique_ptr<Message>
    parse(const std::vector<uint8_t>& message_pack)
        const = 0;
};

class MessagePackMessageParser {
public:
    virtual std::unique_ptr<Message>
    parse(const std::vector<uint8_t>& message_pack)
        const = 0;
};
\end{cpp}

这种设计允许从 JSON Message Parser 和 MessagePackMessageParser 派生的对象分别了解如何从 JSON 和 MessagePack 构建自身,同时保留每个函数的独立性和功能性。系统保持灵活性,因为仍然可以组合新的较小对象来实现所需的功能。

遵守 ISP 使得代码更易于维护且更不容易出错,因为它减少了客户端需要实现的不必要方法的数量,并且还降低了客户端依赖于不需要的方法的风险。

\mySubsubsection{2.1.5}{依赖倒置原则}

依赖倒置原则基于以下理念:最好依赖抽象而不是具体实现,因为这样可以提供更大的灵活性和可维护性。它允许将高级模块与低级模块分离,使它们更加独立,并且不易受到低级模块更改的影响。这样,就可以轻松地更改低级实现而不会影响高级模块,反之亦然。

如果我们尝试通过另一个类使用所有组件,则可以为我们的消息传递系统说明 DIP。假设有一个类负责消息路由。为了构建这样的类,我们将使用 MessageSender 作为通信模块、基于消息的类和 MessageSaver:

\begin{cpp}
class MessageRouter {
public:
    MessageRouter(ReceiverId id)
        : id_{id} {}

    void route(const Message& message) const {
        if (message.receiver_id() == id_) {
            handler_.handle(message);
        } else {
            try {
                sender_.send(message);
            } catch (const CommunicationError& e) {
                saver_.save(message);
            }
        }
    }

private:
    const ReceiverId id_;
    const MessageHandler handler_;
    const MessageSender sender_;
    const MessageSaver saver_;
};
\end{cpp}

新类仅提供一种路由方法,一旦有新消息可用,就会调用该方法。如果消息的发送者 ID 等于路由器的发送者 ID,则路由器会将消息处理到 MessageHandler 类。否则,路由器会将消息转发给相应的接收者。如果消息传递失败且通信层抛出异常,路由器会通过 MessageSaver 保存消息。这些消息将在其他时间传递。

唯一的问题是,如果需要更改任何依赖项,则必须相应地更新路由器的代码。例如,如果应用程序需要支持几种类型的发送方(TCP 和 UDP),则消息保存器(文件系统与远程) 或消息处理程序的逻辑会发生变化。为了使 MessageRouter 不受此类更改的影响,我们可以使用 DIP 原则重写它:

\begin{cpp}
class BaseMessageHandler {
public:
    virtual ~BaseMessageHandler() {}
    virtual void handle(const Message& message) const = 0;
};

class BaseMessageSender {
public:
    virtual ~BaseMessageSender() {}
    virtual void send(const Message& message) const = 0;
};

class BaseMessageSaver {
public:
    virtual ~BaseMessageSaver() {}
    virtual void save(const Message& message) const = 0;
};

class MessageRouter {
public:
    MessageRouter(ReceiverId id,
                  const BaseMessageHandler& handler,
                  const BaseMessageSender& sender,
                  const BaseMessageSaver& saver)
        : id_{id}, handler_{handler}, sender_{sender},
            saver_{saver} {}

    void route(const Message& message) const {
        if (message.receiver_id() == id_) {
            handler_.handle(message);
        } else {
            try {
                sender_.send(message);
            } catch (const CommunicationError& e) {
                saver_.save(message);
            }
        }
    }

private:
    ReceiverId id_;
    const BaseMessageHandler& handler_;
    const BaseMessageSender& sender_;
    const BaseMessageSaver& saver_;
};

int main() {
    auto id = ReceiverId{42};
    auto handler = MessageHandler{};
    auto sender = MessageSender{
        Communication::get_instance()};
    auto saver =
        FilesystemMessageSaver{"/tmp/undelivered_messages"};
    auto router = MessageRouter{id, sender, saver};
}
\end{cpp}

在此修订版本的代码中, MessageRouter 现在与消息处理、发送和保存逻辑的具体实现分离。相反,它依赖于 BaseMessageHandler、 BaseMessageSender 和 BaseMessageSaver 所代表的抽象。这样,从这些基类派生的任何类都可以与 MessageRouter 一起使用,这使得代码更加灵活,并且将来更容易扩展。路由器不关心消息的处理、发送或保存细节——它只需要知道可以执行这些操作。

遵循 DIP 可使代码更易于维护,更不容易出错。它将高级模块与低级模块分离,使它们更加独立,更不容易受到低级模块更改的影响。它还允许更大的灵活性,可以轻松更改低级实现而不会影响高级模块,反之亦然。在本书后面,依赖项反转将帮助我们在开发单元测试时模拟系统的各个部分。























