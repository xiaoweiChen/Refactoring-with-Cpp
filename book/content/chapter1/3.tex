
可读性、效率、可维护性和可用性都是编写代码时需要考虑的关键因素。

\mySubsubsection{1.3.1}{可读性}

可读性是指人类读者理解一段代码的难易程度。编写良好的代码易于阅读,语句清晰简洁,组织方式合乎逻辑且一致。如果考虑到开发人员阅读代码所花的时间是编写代码的十倍,那么这一点就变得非常重要。

来看看下面这段代码:

\begin{cpp}
class Employee {
public:
    std::string get_name();
    std::string surname();
    uint64_t getId() const;
};
\end{cpp}

此示例是代码不遵循任何代码约定的夸张示例。使用 Employee 类的开发人员可以理解这三个方法都是 getter,但名称的差异使得用户花费更多时间来理解代码或尝试理解名称。这些方法的名称不同是因为开发者不关心类的一致性吗?或者因为,没有 get 前缀的方法就是简单的 getter,而包含 get 前缀的方法则从文件或数据库中提取数据?

此外,没有 const 的方法是否会改变对象的状态(例如通过缓存),或者这是一个错误?你知道可以提出多少问题吗?只有当开发人员进入相应的实现时才能回答这些问题,真的很浪费时间。使代码在整个代码库中看起来统一,有助于开发人员通过查看头文件中的声明,或通过现代代码编辑器中的自动完成来理解类、方法和函数的含义和复杂性。

\mySubsubsection{1.3.2}{效率}

效率是指一段代码高效地执行其预期任务的能力。高效的代码使用很少的资源(例如时间和内存)来完成任务,并且能够处理大量数据和流量而不会减慢或崩溃。通过提高代码效率,开发者可以节省时间和资源,并确保代码具有可扩展性,可满足不断增长的用户群的需求。

有一些万无一失的方法可以优化 C++ 代码,例如通过常量引用传递只读参数以避免不必要的复制,或者在查找单个字符时使用 std::string::find 的字符重载版本,以避免创建字符串:

\begin{cpp}
my_string.find('A');
my_string.find("A")
\end{cpp}

然而,实现和保持代码效率的更系统方法是遵循帕累托原则。该原则应用于软件工程时,大约 20\% 的代码完成了 80\% 的工作。例如,通常不需要在后台守护进程启动时优化代码解析配置文件,这在程序的生命周期中只会发生一次。然而,避免在主流中复制大型数据结构可能很重要。提高效率的最佳方法包括挑选这 20\% 的性能关键代码并为其添加基准。基准测试预计将作为 CI 过程的一部分运行,以确保不会引入任何性能下降。

此外,端到端测试可以衡量应用程序的整体性能。本书在第 13 章讨论了编写单元测试和端到端测试的最佳实践。需要注意的是,自动化工具无法取代工程师对新代码进行代码审查,主要是因为没有工具可以找到完成 80\% 工作的那 20\% 的代码。

\mySubsubsection{1.3.3}{可维护性}

可维护性是指代码随时间推移更新和修改的难易程度。编写良好的代码易于维护,代码清晰且文档齐全,组织方式合乎逻辑且一致。通过提高代码的可维护性,开发者可以让其他人更轻松地更新和修改他们的代码,并确保他们的代码随时间推移保持相关性和实用性。理想情况下,开发新组件时,开发人员应考虑代码正在解决的当前问题,以及代码的未来使用和扩展。例如,开发对数据提供程序的支持时,询问该提供程序是否是唯一受支持的提供程序可能会很有用。如果不是,考虑数据提供程序的标准功能,并将其提取到抽象基类中可能会有所帮助。以下是一个例子:

\begin{cpp}
class BaseDataProvider {
public:
    BaseDataProvider() = default;
    BaseDataProvider(const BaseDataProvider&) = delete;
    BaseDataProvider(BaseDataProvider&&) = default;
    BaseDataProvider& operator = (const BaseDataProvider&) = delete;
    BaseDataProvider& operator = (BaseDataProvider&&) = default;
    virtual ~BaseDataProvider() = default;
    virtual Data getData() const = 0;
};

class NetworkDataProvider : public BaseDataProvider {
public:
    NetworkDataProvider(const Endpoint& endpoint);
    Data getData() const override;
};

class FileDataProvider : public BaseDataProvider {
public:
    FileDataProvider(const std::string& filename);

    Data getData() const override;
};
\end{cpp}

此示例中, DataProvider 类是一个抽象基类,定义了用于提供数据的接口。 NetworkDat aProvider 和 FileDataProvider 类派生自 DataProvider,并重写 getData 虚函数,以分别提供从文件或网络端点读取数据的具体实现。这种设计使得添加新数据源变得很容易,只需创建一个派生自 DataProvider 的新类,并为 getData 虚函数提供适当的实现即可。

从示例中可以清楚地看出,基类接口可能不仅包括功能,还包括对象的复制移动策略。稍后,用户代码可以参考基类接收数据提供程序,并且与提供程序的类型无关,如以下代码段所示:

\begin{cpp}
class DataParser {
public:
    DataParser(const BaseDataProvider& provider);
    void parse();
};
\end{cpp}

此外,在为 DataParser 创建单元测试时,此继承可用于模拟数据提供者。单元测试将在第 13 章中详细介绍。

另外,重要的是不要让代码过于复杂,也不要为其他变化做准备。否则,让一切都可扩展的需求可能会导致出现以下这样的怪物代码:

\begin{cpp}
#define BASE_CLASS(TYPE) \
    template <typename T> \
    class TYPE { \
    public: \
        T value; \
        TYPE(T val) : value(val) {} \
    };

#define DERIVED_CLASS(TYPE, BASE) \
    template <typename T> \
    class TYPE : public BASE<T> { \
    public: \
        TYPE(T val) : BASE<T>(val) {} \
        T getValue() { return value; } \
    };

BASE_CLASS(Base);
DERIVED_CLASS(Derived, Base);

int main() {
    Derived<int> obj(5);
    std::cout << obj.getValue() << std::endl;
    return 0;
}
\end{cpp}

这个类层次结构不必要地复杂,它使用了所有 C++ 功能:继承、模板和宏。虽然使用模板继承是一种常见做法,但如今宏被视为“反模式”。此示例中,与 Base 类相比, Derived 类添加的附加功能非常少,直接将 getValue 方法添加到 Base 类会更简单。使用继承和模板在某些情况下很有用,但重要的是要适当使用,不要过度使用。宏可能特别难以理解和维护,因其在编译代码之前由预处理器扩展,因此很难看到实际代码是什么样子。通常,最好使用函数或模板函数,而非宏。

如果扩展的可能性较低,则最好保持其结构简单并贴近基本需求。如何决定采用哪种方法?嗯,冷静的考虑和代码审查是找出答案的方法。

\mySubsubsection{1.3.4}{可用性}

可用性是指一段代码可供他人使用的难易程度。编写良好的代码易于使用,具有清晰直观的界面和文档,使其他人可以轻松理解和使用代码。通过提高代码的可用性,开发者可以使他们的代码更易于他人访问和使用,并确保代码可广泛采用和使用。

总体而言,可读性、效率、可维护性和可用性都是编写代码时需要考虑的重要因素。通过改善这些因素,开发者可以编写出更易于理解、维护和使用的代码。










