本章中,了解了代码好坏的概念。好代码编写得好、高效、易于理解和维护,遵循编码标准和最佳实践,不易出错。另一方面,坏代码编写得不好、效率低下、难以理解和维护。

本章还介绍了技术债务的概念,技术债务指的是需要重构或重写的劣质代码的积累。修复技术债务可能既昂贵又耗时,并且会阻碍新特性或功能的开发。

本章还强调了代码标准的重要性。代码标准是规定如何编写、格式化和构造代码的指南或规则。遵守代码标准有助于确保代码一致、易于理解和可维护。它还使多个开发人员更容易在同一代码库上工作,并有助于避免错误和漏洞。

总体而言,本章强调了编写高质量代码和遵守代码标准的重要性,以避免技术债务并确保软件项目的长期成功和可维护性。

下一章中,我们将深入探讨软件设计原则的世界。具体来说,将重点介绍 SOLID 原则,这是一组旨在通过使软件系统更易于维护、更灵活和更可扩展来改进软件系统设计的指南。下一章将详细解释每一项原则,并附上如何将其应用于实际软件开发场景的示例。
