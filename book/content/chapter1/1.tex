对于好或干净的代码,并没有严格的定义。此外,也没有自动化工具可以衡量代码的质量。有代码检查器和其他分析器可以帮助改进代码。这些工具非常有价值,也非常值得推荐,但还不够。人工智能可能会接管并开发代码,但它对代码质量的衡量将基于人类对好代码的看法。

编程语言最初为了在机器和开发人员之间提供接口而开发;然而,随着软件产品复杂性的增长。如今它成为开发人员之间交流想法和意图的一种方式,开发人员花在阅读代码上的时间会是编写代码的十倍。为了提高效率,必须尽最大努力使阅读更容易。使这个过程高效的最成功方法是让代码可预测,或者更确切地说是“干货”。所谓干货,是指读者看到代码就知道在功能、性能和副作用方面会有什么期望。以下示例说明了从数据库检索对象的类的接口:

\begin{cpp}
class Database {
public:
    template<typename T>
    std::optional<T> get(const Id& id) const;
    template<typename T>
    std::optional<T> get(const std::string& name) const;
};
\end{cpp}

它支持两种查找模式,按 ID 和按名称;由于 const 修饰符,不应改数据库对象的内部状态;并且只能对数据库实例执行读取操作。想象一下,在错误调查期间,发现它在每次读取操作时都会执行更新操作,有时还会执行删除操作,这会令人非常惊讶。

正如读者所见,还是有几个关键因素可以区分好代码和坏代码。好代码通常写得好、易读且高效,遵循 C+ 语言的惯例和标准,并以合乎逻辑和一致的方式组织。好代码还具有良好的文档记录,通常带有清晰的注释,解释了仅从阅读代码中无法看出的事物的目的和功能。
