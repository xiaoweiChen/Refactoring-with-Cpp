
在 C++ 中,指针是语言的基本组成部分,允许直接访问和操作内存。然而,指针提供的灵活性也伴随着一定的风险和挑战。在这里,我们将探讨现代 C++ 技术如何增强指针安全性。

\mySubsubsection{6.2.1}{原始指针的缺陷}

原始指针虽然功能强大,但也可能成为一把双刃剑。它们不提供有关其指向的对象的所有权的信息,并且很容易成为"悬空"指针,指向已释放的内存。取消引用空指针或悬空指针会导致未定义的行为,从而导致难以诊断的错误。

\mySubsubsection{6.2.2}{使用GSL中的 not\_null}

指南支持库 (GSL) 提供的 not\_null 包装器旨在克服与原始指针相关的挑战。通过使用 not\_null,您可以清楚地表明指针永远不应为空:

\begin{cpp}
#include <gsl/gsl>
void process_data(gsl::not_null<int*> data) {
    // Data is guaranteed not to be null here
}
\end{cpp}

如果用户向此函数传递空指针,如下所示,应用程序将被终止:

\begin{cpp}
int main() {
    int* p = nullptr;
    process_data(p); // this will terminate the program
    return 0;
}
\end{cpp}

但是,如果将指针作为 process\_data(nullptr) 传递,应用程序将在编译时失败:

\begin{shell}
source>: In function 'int main()':
<source>:9:16: error: use of deleted function 'gsl::not_null<T>::not_
null(std::nullptr_t) [with T = int*; std::nullptr_t = std::nullptr_t]'
    9 | process_data(nullptr);
      | ~~~~~~~~~~~^~~~~~~~~
In file included from <source>:1:
/opt/compiler-explorer/libs/GSL/trunk/include/gsl/pointers:131:5:
note: declared here
  131 | not_null(std::nullptr_t) = delete;
      | ^~~~~~~~
\end{shell}

这通过早期捕获潜在的空指针错误来促进代码的健壮性,从而减少运行时错误。

\mySamllsectionNoContent{将 not\_null 扩展为智能指针}

gsl::not\_null 不仅限于原始指针;它还可以与智能指针(如 std::unique\_ptr 和 std::shared\_ptr )一起使用。这允许您将现代内存管理的优势与 not\_null 提供的额外安全保障结合起来。

\mySamllsectionNoContent{使用 std::unique\_ptr}

std::unique\_ptr 确保动态分配对象的所有权是唯一的,并且当不再需要该对象时会自动删除该对象。通过将 not\_null 与 unique\_ptr 一起使用,您还可以确保指针永远不会为空:

\begin{cpp}
#include <gsl/gsl>
#include <memory>

void process_data(gsl::not_null<std::unique_ptr<int>> data) {
    // Data is guaranteed not to be null here
}

int main() {
    auto data = std::make_unique<int>(42);
    process_data(std::move(data)); // Safely passed to the function
}
\end{cpp}

\mySamllsectionNoContent{使用 std::shared\_ptr}

类似地,gsl::not\_null 可以与 std::shared\_ptr 一起使用,从而实现对象的共享所有权。这样您就可以编写接受共享指针的函数,而不必担心 null 性:

\begin{cpp}
#include <gsl/gsl>
#include <memory>

void process_data(gsl::not_null<std::shared_ptr<int>> data) {
    // Data is guaranteed not to be null here
}

int main() {
    auto data = std::make_shared<int>(42);
    process_data(data); // Safely passed to the function
}
\end{cpp}

这些示例展示了 not\_null 如何与现代 C++ 内存管理技术无缝集成。通过强制指针(无论是原始指针还是智能指针)不能为空,您可以进一步降低运行时错误的可能性,并使代码更加健壮和富有表现力。

\mySubsubsection{6.2.3}{利用 std::optional 获取可选值}

有时,指针用于指示可选值,其中 nullptr 表示没有值。 C++17 引入了 std::optional,它提供了一种类型安全的方式来表示可选值:

\begin{cpp}
#include <optional>
std::optional<int> fetch_data() {
    if (/* some condition */)
        return 42;
    else
        return std::nullopt;
}
\end{cpp}

使用 std::optional 提供了清晰的语义并避免了为此目的使用指针所带来的陷阱。

\mySubsubsection{6.2.4}{原始指针与 nullptr 的比较}

not\_null 和 std::optional 都比原始指针更具优势。虽然原始指针可能为空或悬空,从而导致未定义的行为,但 not\_null 可以在编译时防止出现空指针错误,而 std::optional 提供了一种表示可选值的清晰方法。

考虑以下使用原始指针的示例:

\begin{cpp}
int* findValue() {
    // ...
    return nullptr; // No value found
}
\end{cpp}

此代码可能会导致混乱和错误,尤其是当调用者忘记检查 nullptr 时。通过使用 not\_null 和 std::optional,您可以使代码更具表现力且更不容易出错。

\mySubsubsection{6.2.5}{利用 std::expected 获得预期结果和错误}

虽然 std::optional 可以优雅地表示可选值,但有时您需要传达有关值缺失原因的更多信息。
在这种情况下, std::expected 提供了一种返回值或错误代码的方法,使代码更具表现力,错误处理更加稳健。

考虑这样一种情况:您有一个从网络检索值的函数,并且您想要处理网络错误。您可以为各种网络错误定义一个枚举:

\begin{cpp}
enum class NetworkError {
    Timeout,
    ConnectionLost,
    UnknownError
};
\end{cpp}

然后,您可以使用 std::expected 定义一个返回 int 值或 NetworkError 的函数:

\begin{cpp}
#include <expected>
#include <iostream>

std::expected<int, NetworkError> fetch_data_from_network() {
    // Simulating network operation...
    if (/* network timeout */) {
        return std::unexpected(NetworkError::Timeout);
    }
    if (/* connection lost */) {
        return std::unexpected(NetworkError::ConnectionLost);
    }

    return 42; // Successfully retrieved value
}

int main() {
    auto result = fetch_data_from_network();
    if (result) {
        std::cout << "Value retrieved: " << *result << '\n';
    } else {
        std::cout << "Network error: ";
        switch(result.error()) {
            case NetworkError::Timeout:
                std::cout << "Timeout\n";
                break;
            case NetworkError::ConnectionLost:
                std::cout << "Connection Lost\n";
                break;
            case NetworkError::UnknownError:
                std::cout << "Unknown Error\n";
                break;
        }
    }
}
\end{cpp}

在这里, std::expected 捕获成功案例和各种错误场景,从而实现清晰且类型安全的错误处理。此示例说明了现代 C++ 类型(例如 std::expected)如何增强表达能力和安全性,让您能够编写更准确地模拟复杂操作的代码。

通过采用这些现代的 C++ 工具,您可以增强代码中的指针安全性,减少错误并明确您的意图。








































