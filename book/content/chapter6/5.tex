

\mySubsubsection{6.5.1}{使用类型检查编写健壮的代码}

类型检查是程序稳健性和安全性的支柱之一。虽然 C++ 是强类型的,但它确实允许一定的灵活性(或宽容性,取决于您的观点),如果不仔细管理,可能会导致错误。以下是通过利用类型检查编写稳健的 C++ 代码的一些技术和最佳实践。

\mySamllsectionNoContent{使用类型特征进行编译时检查}

C++ 标准库在 <type\_traits> 标头中提供了一组类型特征,可让您在编译时检查并根据类型做出决策。例如,如果您有一个只应接受无符号整数类型的泛型函数,则可以使用 static\_assert 强制执行此操作:

\begin{cpp}
#include <type_traits>

template <typename T>
void foo(T value) {
    static_assert(std::is_unsigned<T>::value, "foo() requires an unsigned integral type");
    // ... function body
}
\end{cpp}

\mySamllsectionNoContent{利用 constexpr if}

C++17 引入了 constexpr if,使您可以编写在编译时进行评估的条件代码。这对于模板代码中特定于类型的操作非常有用:

\begin{cpp}
template <typename T>
void bar(T value) {
    if constexpr (std::is_floating_point<T>::value) {
        // Handle floating-point types
    } else if constexpr (std::is_integral<T>::value) {
        // Handle integral types
    }
}
\end{cpp}

\mySamllsectionNoContent{函数参数的强类型}

C++ 允许类型别名,这有时会使理解函数参数的用途变得困难。例如,声明为 void process( int, int); 的函数信息量不大。第一个整数是长度吗?第二个整数是索引吗?缓解这种情况的一种方法是使用强类型定义,例如:

\begin{cpp}
struct Length { int value; };
struct Index { int value; };

void process(Length l, Index i);
\end{cpp}

现在,函数签名提供了语义含义,使得开发人员不太可能意外交换参数。

\mySubsubsection{6.5.2}{隐式转换和类型强制}

\mySamllsectionNoContent{意外创建文件的情况}

在 C++ 开发中,为了实现灵活性,定义带有构造函数的类是很常见的,这些构造函数可以接受各种参数类型。然而,这会带来无意隐式转换的风险。为了说明这一点,请考虑以下涉及 File 类和 clean 函数的代码片段:

\begin{cpp}
#include <iostream>
class File {
public:
    File(const std::string& path) : path_{path} {
        auto file = fopen(path_.c_str(), "w");
        // check if file is valid
        // handle errors, etc
        std::cout << "File ctor\n";
    }

    auto& path() const {
        return path_;
    }

    // other ctors, dtor, etc

private:
    FILE* file_ = nullptr;
    std::string path_;
};

void clean(const File& file) {
    std::cout << "Removing the file: " << file.path() << std::endl;
}

int main() {
    auto random_string = std::string{"blabla"};
    clean(random_string);
}
\end{cpp}

输出清楚地说明了这个问题:

\begin{shell}
File ctor
Removing the file: blabla
\end{shell}

由于构造函数中缺少显式关键字,编译器会自动将 std::string 转换为 File 对象,从而引发意外的副作用——创建一个新文件。

\mySamllsectionNoContent{显式的效用}

为了减轻此类风险,可以使用显式关键字。通过将构造函数标记为显式,您可以指示编译器禁止该构造函数进行隐式转换。以下是更正后的 File 类:

\begin{cpp}
class File {
    public:
    explicit File(const std::string& path) : path_{path} {
        auto file = fopen(path_.c_str(), "w");
        // check if file is valid
        // handle errors, etc
        std::cout << "File ctor\n";
    }
    // ... rest of the class
};
\end{cpp}

经过此更改后, clean(random\_string); 行将导致编译错误,从而有效地防止意外创建文件。

\mySamllsectionNoContent{一个轻松的警告}

虽然出于教学目的,我们的示例可能有些简化(是的,没有必要自己创建 File 类——我们有这方面的库!),但它强调了 C++ 中类型安全的一个关键方面。如果没有明确防范隐式转换,看似无害的构造函数可能会导致意外行为。

因此,请记住,定义构造函数时,明确说明您的意图是有益的。您永远不知道何时会意外启动您从未打算举办的"文件派对"。















