如何利用 C++ 的类型系统编写更强大的代码的最佳示例之一是 <chrono> 库。此头文件在 C++ 11 中引入,提供了一组实用程序来表示时间持续时间和时间点,以及执行与时间相关的操作。

使用纯整数或诸如 timespec 之类的结构,来管理与时间相关的函数可能是一种容易出错的方法,尤其是在处理不同的时间单位时。试想一下,一个函数接受一个表示以秒为单位的超时时间的整数:

\begin{cpp}
void wait_for_data(int timeout_seconds) {
    sleep(timeout_seconds); // Sleeps for timeout_seconds seconds
}
\end{cpp}

这种方法缺乏灵活性,并且在处理各种时间单位时会导致混乱。例如,调用者错误地传递了毫秒而不是秒,则可能导致意外行为。

相比之下,使用<chrono>定义相同的函数使代码更加健壮和富有表现力:

\begin{cpp}
#include <chrono>
#include <thread>
void wait_for_data(std::chrono::seconds timeout) {
    std::this_thread::sleep_for(timeout); // Sleeps for the specified timeout
}
\end{cpp}

调用者现在可以使用强类型持续时间(例如 std::chrono::seconds(5))传递超时,并且编译器会确保使用正确的单位。此外, <chrono> 提供不同时间单位之间的无缝转换,允许调用者以秒、毫秒或其他单位指定超时,而不会产生歧义。以下代码说明了使用不同时间单位的用法:

\begin{cpp}
wait_for_data(std::chrono::milliseconds(150));
\end{cpp}

通过采用 <chrono> 提供的强类型,代码变得更清晰、更易于维护,并且不易受到与时间表示相关的常见错误的影响。


