代码异味,是指代码库中表明存在底层设计或实现问题的症状。这些症状不一定是错误,但表明存在潜在问题,这些问题可能会使代码更难理解、维护和修改。代码异味通常是编码实践不佳,或技术债务随着时间的推移而积累的结果。虽然代码异味可能不会直接影响程序的功能,但会严重影响整体代码质量,从而增加错误风险,并降低开发人员的工作效率。

解决异味代码的一个方面是识别和应用适当的设计模式,设计模式是软件设计中常见问题的可重用解决方案。它们为解决特定问题提供了经过验证的框架,使开发人员能够借鉴其他开发人员的集体智慧和经验。通过应用这些模式,可以将异味代码重构为更结构化、模块化和更易于维护的形式。

策略模式允许定义一系列算法,将每个算法封装在单独的类中,并在运行时可互换。策略模式对于重构具有多个分支或条件的代码非常有用,这些分支或条件执行类似的任务,但实现略有不同。

考虑一个使用不同存储策略保存数据的应用程序示例,例如:保存到磁盘或远程存储的服务:

\begin{cpp}
#include <iostream>
#include <fstream>
#include <string>
#include <assert>

enum class StorageType {
    Disk,
    Remote
};

class DataSaver {
public:
    DataSaver(StorageType storage_type) : storage_type_(storage_type)
    {}

    void save_data(const std::string& data) const {
        switch (storage_type_) {
            case StorageType::Disk:
            save_to_disk(data);
            break;
            case StorageType::Remote:
            save_to_remote(data);
            break;
            default:
            assert(false && "Unknown storage type.");
        }
    }

    void set_storage_type(StorageType storage_type) {
        storage_type_ = storage_type;
    }

private:
    void save_to_disk(const std::string& data) const {
        // saving to disk
    }

    void save_to_remote(const std::string& data) const {
        // saving data to a remote storage service.
    }
    StorageType storage_type_;
};

int main() {
    DataSaver disk_data_saver(StorageType::Disk);
    disk_data_saver.save_data("Save this data to disk.");

    DataSaver remote_data_saver(StorageType::Remote);
    remote_data_saver.save_data("Save this data to remote storage.");

    // Switch the storage type at runtime.
    disk_data_saver.set_storage_type(StorageType::Remote);
    disk_data_saver.save_data("Save this data to remote storage after switching storage type.");
    return 0;
}
\end{cpp}

这个类中, save\_data 方法在每次调用时检查存储类型,并使用 switch-case 块来决定使用哪种保存方法。这种方法有效,但也有一些缺点:

\begin{itemize}
\item
DataSaver 类负责处理所有不同的存储类型,这使得维护和扩展变得更加困难。

\item
添加新的存储类型需要修改 DataSaver 类和 StorageType 枚举,这会增加引入错误或破坏现有功能的风险。例如,出于某种原因提供了错误的枚举类型,代码就会中止。

\item
与将行为封装在单独类中的策略模式相比,代码的模块化和灵活性较差。
\end{itemize}

通过实现策略模式,可以解决这些缺点,并为 DataSaver 类创建更易于维护、更灵活、更可扩展的设计。首先,定义一个名为 SaveStrategy 的接口,表示保存行为:

\begin{cpp}
class SaveStrategy {
public:
    virtual ~SaveStrategy() {}
    virtual void save_data(const std::string& data) const = 0;
};
\end{cpp}

接下来,为每种存储类型实现具体的 SaveStrategy 类:

\begin{cpp}
class DiskSaveStrategy : public SaveStrategy {
    public:
    void save_data(const std::string& data) const override {
        // ...
    }
};

class RemoteSaveStrategy : public SaveStrategy {
    public:
    void save_data(const std::string& data) const override {
        // ...
    }
};
\end{cpp}

现在,创建一个 DataSaver 类,该类使用策略模式将其保存行为委托给适当的 SaveStrategy 实现:

\begin{cpp}
class DataSaver {
public:
    DataSaver(std::unique_ptr<SaveStrategy> save_strategy)
     : save_strategy_(std::move(save_strategy)) {}

    void save_data(const std::string& data) const {
        save_strategy_->save_data(data);
    }

    void set_save_strategy(std::unique_ptr<SaveStrategy> save_strategy) {
        save_strategy_ = std::move(save_strategy);
    }

private:
    std::unique_ptr<SaveStrategy> save_strategy_;
};
\end{cpp}

最后,有一个如何使用 DataSaver 类和不同的保存策略的示例:

\begin{cpp}
int main() {
    DataSaver disk_data_saver(std::make_unique<DiskSaveStrategy>());
    disk_data_saver.save_data("Save this data to disk.");

    DataSaver remote_data_saver(std::make_unique<RemoteSaveStrategy>());
    remote_data_saver.save_data("Save this data to remote storage.");

    // Switch the saving strategy at runtime.
    disk_data_saver.set_save_strategy(std::make_unique<RemoteSaveStrategy>());
    disk_data_saver.save_data("Save this data to remote storage after switching strategy.");
    return 0;
}
\end{cpp}

在此示例中, DataSaver 类使用策略模式将其保存行为委托给不同的 SaveStrategy 实现,使其能够轻松地在保存到磁盘和保存到远程存储之间切换。这种设计使代码更加模块化、易于维护和灵活,允许在对现有代码进行最少更改的情况下添加新的存储策略。此外,新版本的代码不需要在错误的保存策略类型上终止或抛出异常。

假设有两种格式的文件解析器实现, CSV 和 JSON:

\begin{cpp}
class CsvParser {
public:
    void parse_file(const std::string& file_path) {
        std::ifstream file(file_path);
        if (!file) {
            std::cerr << "Error opening file: " << file_path << std::endl;
            return;
        }

        std::string line;
        while (std::getline(file, line)) {
            process_line(line);
        }

        file.close();
        post_process();
    }

private:
    void process_line(const std::string& line) {
        // Implement the CSV-specific parsing logic.
        std::cout << "Processing CSV line: " << line << std::endl;
    }

    void post_process() {
        std::cout << "CSV parsing completed." << std::endl;
    }
};

class JsonParser {
public:
    void parse_file(const std::string& file_path) {
        std::ifstream file(file_path);
        if (!file) {
            std::cerr << "Error opening file: " << file_path << std::endl;
            return;
        }

        std::string line;
        while (std::getline(file, line)) {
            process_line(line);
        }

        file.close();
        post_process();
    }

private:
    void process_line(const std::string& line) {
        // Implement the JSON-specific parsing logic.
        std::cout << "Processing JSON line: " << line << std::endl;
    }

    void post_process() {
        std::cout << "JSON parsing completed." << std::endl;
    }
};
\end{cpp}

此示例中, CsvParser 和 JsonParser 类分别实现了 parse\_file 方法,其中包含用于打开、读取和关闭文件的重复代码。格式特定的解析逻辑在 process\_line 和 post\_process 方法中实现。

虽然这种设计可行,但也有一些缺点:共享的解析步骤在两个类中重复,使得维护和更新代码变得更加困难,并且添加对新文件格式的支持,需要创建具有类似代码结构的新类,这可能导致更多的代码重复。

通过实现模板方法模式,可以解决这些缺点,并为文件解析器创建更易于维护、扩展和可重用的设计。 FileParser 基类处理常见的解析步骤,而派生类实现特定于格式的解析逻辑。

与前面的示例一样,首先创建一个抽象基类。 FileParser 表示一般的文件解析过程:

\begin{cpp}
class FileParser {
public:
    void parse_file(const std::string& file_path) {
        std::ifstream file(file_path);
        if (!file) {
            std::cerr << "Error opening file: " << file_path << std::endl;
            return;
        }

        std::string line;
        while (std::getline(file, line)) {
            process_line(line);
        }

        file.close();
        post_process();
    }

protected:
    virtual void process_line(const std::string& line) = 0;
    virtual void post_process() = 0;
};
\end{cpp}

FileParser 类有一个 parse\_file 方法,它处理打开文件、逐行读取其内容,以及关闭文件的常见步骤。格式特定的解析逻辑由纯虚 process\_line 和 post\_process 方法实现,这些方法将在派生类中重写。

现在,为不同的文件格式创建派生类:

\begin{cpp}
class CsvParser : public FileParser {
protected:
    void process_line(const std::string& line) override {
        // Implement the CSV-specific parsing logic.
        std::cout << "Processing CSV line: " << line << std::endl;
    }

    void post_process() override {
        std::cout << "CSV parsing completed." << std::endl;
    }
};

class JsonParser : public FileParser {
protected:
    void process_line(const std::string& line) override {
        // Implement the JSON-specific parsing logic.
        std::cout << "Processing JSON line: " << line << std::endl;
    }

    void post_process() override {
        std::cout << "JSON parsing completed." << std::endl;
    }
};
\end{cpp}

此示例中, CsvParser 和 JsonParser 类继承自 FileParser,并在 process\_line 和 post\_process 方法中实现特定于格式的解析逻辑。

以下是如何使用文件解析器的示例:

\begin{cpp}
int main() {
    CsvParser csv_parser;
    csv_parser.parse_file("data.csv");

    JsonParser json_parser;
    json_parser.parse_file("data.json");

    return 0;
}
\end{cpp}

通过实现模板方法模式, FileParser 类提供了一个可重复使用的模板来处理文件解析的常见步骤,同时允许派生类实现特定于格式的解析逻辑。这种设计使得添加对新文件格式的支持变得容易,而无需修改基础 FileParser 类,从而产生更易于维护和扩展的代码库。值得注意的是,实现此设计模式的复杂部分是识别类之间的通用逻辑,实现通常需要对通用逻辑进行某种统一。

另一个有用的模式是观察者模式。上一章提到了其技术实现细节(原始、共享或weak\_ptr指针实现)。然而,在本章中,我想从设计角度介绍它的用法。

观察者模式定义了对象之间的一对多依赖关系,允许在主体状态发生变化时通知多个观察者。在重构涉及事件处理或对多个依赖组件进行更新的代码时,此模式非常有用。

考虑一个汽车系统,其中 Engine 类保存汽车的当前速度和 RPM(每分钟转数)。有几个元素需要了解这些值,例如 Dashboard 和 Controller。仪表板显示来自引擎的最新更新, Controller 根据速度和 RPM 调整汽车的行为。实现这一点的简单方法是,让 Engine 类直接在每个显示元素上调用update方法:

\begin{cpp}
class Dashboard {
public:
    void update(int speed, int rpm) {
        // display the current speed
    }
};

class Controller {
public:
    void update(int speed, int rpm) {
        // Adjust car's behavior based on the speed and RPM.
    }
};

class Engine {
public:
    void set_dashboard(Dashboard* dashboard) {
        dashboard_ = dashboard;
    }

    void set_controller(Controller* controller) {
        controller_ = controller;
    }

    void set_measurements(int speed, int rpm) {
        speed_ = speed;
        rpm_ = rpm;
        measurements_changed();
    }

private:
    void measurements_changed() {
        dashboard_->update(_speed, rpm_);
        controller_->update(_speed, rpm_);
    }

    int speed_;
    int rpm_;
    Dashboard* dashboard_;
    Controller* controller_;
};

int main() {
    Engine engine;
    engine.set_measurements(80, 3000);
    return 0;
}
\end{cpp}

这段代码有几个问题:

\begin{itemize}
\item
Engine 类与 Dashboard 和 Controller 紧密耦合,很难添加或删除可能对汽车速度和RPM 感兴趣的其他组件。

\item
Engine类负责直接更新显示元素,这使代码变得复杂并且灵活性降低。
\end{itemize}

我们可以使用观察者模式重构代码,将 Engine 与显示元素解耦。 Engine 类将成为主体,而Dashboard 和 Controller 将成为观察者:

\begin{cpp}
class Observer {
public:
    virtual ~Observer() {}
    virtual void update(int speed, int rpm) = 0;
};

class Dashboard : public Observer {
public:
    void update(int speed, int rpm) override {
        // display the current speed
    }
};

class Controller : public Observer {
public:
    void update(int speed, int rpm) override {
        // Adjust car's behavior based on the speed and RPM.
    }
};

class Engine {
public:
    void register_observer(Observer* observer) {
        observers_.push_back(observer);
    }

    void remove_observer(Observer* observer) {
        observers_.erase(std::remove(_observers.begin(), observers_.
        end(), observer), observers_.end());
    }

    void set_measurements(int speed, int rpm) {
        speed_ = speed;
        rpm_ = rpm;
        notify_observers();
    }

private:
    void notify_observers() {
        for (auto observer : observers_) {
            observer->update(_speed, _rpm);
        }
    }

    std::vector<Observer*> observers_;
    int speed_;
    int rpm_;
};
\end{cpp}

以下代码片段演示了新类层次结构的用法:

\begin{cpp}
int main() {
    Engine engine;
    Dashboard dashboard;
    Controller controller;

    // Register observers
    engine.register_observer(&dashboard);
    engine.register_observer(&controller);

    // Update measurements
    engine.set_measurements(80, 3000);

    // Remove an observer
    engine.remove_observer(&dashboard);

    // Update measurements again
    engine.set_measurements(100, 3500);

    return 0;
}
\end{cpp}

此示例中,仪表板和控制器已注册为引擎主题的观察者。当引擎的速度和 RPM 发生变化时,将调用 set\_measurements,从而触发 notify\_observers,后者又会调用每个已注册观察者上的 update 方法,这样仪表板和控制器便可以接收更新的速度和 RPM 值。

然后, Dashboard 不再作为观察者注册。当发动机的速度和 RPM 再次更新时,只有控制器会收到更新的值。

通过此设置,添加或删除观察者就像在 Engine 上调用 register\_observer 或 remove\_observer 一样简单,并且在添加新类型的观察者时无需修改 Engine 类。Engine 类现在与特定的观察者类解耦,使系统更加灵活且更易于维护。

另一个很棒的模式是状态机。它不是经典模式,但可能是最强大的模式。状态机,也称为有限状态机 (FSM),是计算的数学模型,用于表示和控制硬件和软件设计中的执行流程。状态机具有有限数量的状态,并且在给定时间内,都处于其中一种状态。它会根据外部输入或预定义条件,从一种状态转换为另一种状态。

在硬件领域,状态机经常用于数字系统的设计,作为从微型微控制器到大型中央处理器 (CPU) 等所有设备的控制逻辑。它们控制操作顺序,确保操作按正确的顺序发生,并确保系统对不同的输入或条件做出适当的响应。

在软件中,状态机同样有用,特别是在程序流程受一系列状态和这些状态之间的转换影响的系统中。应用范围从嵌入式系统中的简单按钮去抖动,到复杂的游戏角色行为和通信协议管理。

状态机非常适合以下情况:系统具有一组定义明确的循环状态,并且状态之间的转换由特定事件或条件触发。当系统的行为不仅取决于当前输入,还取决于系统的历史时,状态机尤其有用。状态机以当前状态的形式封装此历史,使其明确且易于管理。

使用状态机有很多好处,可以简化复杂的条件逻辑,使其更易于理解、调试和维护。还可以轻松添加新状态或转换,而不会干扰现有代码,从而增强模块化和灵活性。此外,可使系统的行为明确且可预测,从而降低了意外行为的风险。

考虑一个分布式计算系统的真实场景,其中提交了一项作业以供处理。此作业会经历各种状态,例如:已提交、已排队、正在运行、已完成和已失败。我们将使用 Boost.Statechart 库对此进行建模。 Boost.Statechart 是一个 C++ 库,提供了一个用于构建状态机的框架。它是 Boost 库集合的一部分。该库有助于开发分层状态机,可以对具有复杂状态和转换的复杂系统进行建模。其旨在使在处理复杂状态逻辑时更容易编写结构良好、模块化且可维护的代码。 Boost.Statechart 提供编译时和运行时检查,以帮助确保状态机行为的正确性。

首先,包含必要的头文件并设置一些命名空间:

\begin{cpp}
#include <boost/statechart/state_machine.hpp>
#include <boost/statechart/simple_state.hpp>
#include <boost/statechart/transition.hpp>
#include <iostream>

namespace sc = boost::statechart;
\end{cpp}

接下来,定义事件: JobSubmitted、 JobQueued、 JobRunning、 JobCompleted 和 JobFailed:

\begin{cpp}
struct EventJobSubmitted : sc::event< EventJobSubmitted > {};
struct EventJobQueued : sc::event< EventJobQueued > {};
struct EventJobRunning : sc::event< EventJobRunning > {};
struct EventJobCompleted : sc::event< EventJobCompleted > {};
struct EventJobFailed : sc::event< EventJobFailed > {};
\end{cpp}

然后定义状态,每个状态都是从 sc::simple\_state 继承的一个类。我们将有五种状态:已提交、已排队、正在运行、已完成和失败:

\begin{cpp}
struct Submitted;
struct Queued;
struct Running;
struct Completed;
struct Failed;

struct Submitted : sc::simple_state< Submitted, Job > {
    typedef sc::transition< EventJobQueued, Queued > reactions;

    Submitted() { std::cout << "Job Submitted\n"; }
};

struct Queued : sc::simple_state< Queued, Job > {
    typedef sc::transition< EventJobRunning, Running > reactions;

    Queued() { std::cout << "Job Queued\n"; }
};

struct Running : sc::simple_state< Running, Job > {
    typedef boost::mpl::list<
        sc::transition< EventJobCompleted, Completed >,
        sc::transition< EventJobFailed, Failed >
    > reactions;
    Running() { std::cout << "Job Running\n"; }
};

struct Completed : sc::simple_state< Completed, Job > {
    Completed() { std::cout << "Job Completed\n"; }
};

struct Failed : sc::simple_state< Failed, Job > {
    Failed() { std::cout << "Job Failed\n"; }
};
\end{cpp}

最后,定义状态机 Job,从已提交状态启动。

\begin{cpp}
struct Job : sc::state_machine< Job, Submitted > {};
\end{cpp}

主函数中,可以创建 Job 状态机的实例,并处理一些事件:

\begin{cpp}
int main() {
    Job my_job;
    my_job.initiate();
    my_job.process_event(EventJobQueued());
    my_job.process_event(EventJobRunning());
    my_job.process_event(EventJobCompleted());
    return 0;
}
\end{cpp}

这将输出以下内容:

\begin{shell}
Job Submitted
Job Queued
Job Running
Job Completed
\end{shell}

这个简单的例子展示了,如何使用状态机来对具有多个状态和转换的流程进行建模。我们使用事件来触发状态之间的转换。另一种方法是使用状态反应,其中状态可以根据条件或它所拥有的数据决定何时转换。

这可以通过使用 Boost.Statechart 中的自定义反应来实现。自定义反应是在处理事件时调用的成员函数,可以决定要做什么:忽略事件、在不转换的情况下使用事件或转换到新状态。

让我们修改作业状态机,使其根据作业的完成状态决定何时从"正在运行"转换为"已完成"或"失败"。

首先,定义一个新的事件 EventJobUpdate,将携带作业的完成状态:

\begin{cpp}
struct EventJobUpdate : sc::event< EventJobUpdate > {
    EventJobUpdate(bool is_complete) : is_complete(is_complete) {}
    bool is_complete;
};
\end{cpp}

然后,在运行状态下,将为该事件定义一个自定义反应:

\begin{cpp}
struct Running : sc::simple_state< Running, Job > {
    typedef sc::custom_reaction< EventJobUpdate > reactions;

    sc::result react(const EventJobUpdate& event) {
        if (event.is_complete) {
            return transit<Completed>();
        } else {
            return transit<Failed>();
        }
    }
    Running() { std::cout << "Job Running\n"; }
};
\end{cpp}

现在,运行状态将根据 EventJobUpdate 事件的 is\_complete 字段,决定何时转换为完成或失败
。

主函数中,可以处理 EventJobUpdat 事件:

\begin{cpp}
int main() {
    Job my_job;
    my_job.initiate();
    my_job.process_event(EventJobQueued());
    my_job.process_event(EventJobRunning());
    my_job.process_event(EventJobUpdate(true)); // The job is complete.
    return 0;
}
\end{cpp}

这将输出以下内容:

\begin{shell}
Job Submitted
Job Queued
Job Running
Job Completed
\end{shell}

如果用 false 处理 EventJobUpdate:

\begin{cpp}
my_job.process_event(EventJobUpdate(false)); // The job is not complete.
\end{cpp}

将输出以下内容:

\begin{shell}
Job Submitted
Job Queued
Job Running
Job Failed
\end{shell}

这表明了状态如何根据条件或拥有的数据来决定何时转换。

通过在状态机中添加新状态和状态之间的转换规则,可以轻松扩展以状态机形式实现的逻辑,但有时状态机可能包含太多状态(比如说,超过七个)。这通常是代码有问题的症状。因为有太多状态实现多个状态机,所以状态机超负荷。例如,分布式系统可以实现为状态机本身。系统可以有自己的状态,例如 Idle、 ProcessingJobs 和 SystemFailure。 ProcessingJobs 状态将进一步包含 Job 状态机作为子状态机。 System 状态机可以通过处理事件与 Job 子状态机进行通信。当 System 转换到 ProcessingJobs 状态时,可以处理 Eve ntJobSubmitted 事件以启动 Job 子状态机。当 Job 转换到 Completed 或 Failed 状态时,可以处理 EventJobFinished 事件以通知 System。

首先,定义 EventJobFinished 事件:

\begin{cpp}
struct EventJobFinished : sc::event< EventJobFinished > {};
\end{cpp}

然后,在Job状态机的Completed和Failed状态中,处理EventJobFinished事件:

\begin{cpp}
struct Completed : sc::simple_state< Completed, Job > {
    Completed() {
        std::cout << "Job Completed\n";
        context< Job >().outermost_context().process_
        event(EventJobFinished());
    }
};

struct Failed : sc::simple_state< Failed, Job > {
    Failed() {
        std::cout << "Job Failed\n";
        context< Job >().outermost_context().process_event(EventJobFinished());
    }
};
\end{cpp}

在系统状态机的 ProcessingJobs 状态中,可为 EventJobFinished 事件定义一个自定义反应:

\begin{cpp}
struct ProcessingJobs : sc::state< ProcessingJobs, System, Job > {
    typedef sc::custom_reaction< EventJobFinished > reactions;

    sc::result react(const EventJobFinished&) {
        std::cout << "Job Finished\n";
        return transit<Idle>();
    }

    ProcessingJobs(my_context ctx) : my_base(ctx) {
        std::cout << "System Processing Jobs\n";
        context< System >().process_event(EventJobSubmitted());
    }
};
\end{cpp}

主函数中,可以创建系统状态机的实例并启动:

\begin{cpp}
int main() {
    System my_system;
    my_system.initiate();
    return 0;
}
\end{cpp}

这将输出以下内容:

\begin{shell}
System Idle
System Processing Jobs
Job Submitted
Job Queued
Job Running
Job Completed
Job Finished
System Idle
\end{shell}

这显示了系统状态机如何与作业子状态机交互。系统在作业转换到 ProcessingJobs 状态时启动作业,作业在完成时通知系统,并系统管理作业的生命周期并对其状态变化做出反应。

总体而言,状态机是一种强大的工具,可以稳健且易于理解地管理复杂行为。尽管状态机很实用,但它并非总是构建代码的首选,这可能是由于感知到的复杂性或缺乏熟悉度。但在处理以复杂的条件逻辑网络为特征的系统时,考虑使用状态机可能是明智之举。它是一种强大的工具,可以为软件设计带来清晰度和稳健性,使其成为 C++ 或任何其他语言中重构工具包的重要组成部分。
