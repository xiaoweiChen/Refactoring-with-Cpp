在本章中,我们探讨了一些有助于重构遗留 C++ 代码的关键设计模式,包括策略模式、模板方法模式和观察者模式。如果明智地应用这些模式,可以显著改善代码结构,使其更灵活、更易于维护、更能适应变化。

虽然我们提供了实际的现实世界示例来说明这些模式的使用,但这绝不是详尽的介绍。设计模式是一个庞大而深奥的主题,还有更多的模式和变体可供探索。为了更全面地了解设计模式,我强烈建议您深入研究由 Erich Gamma、 Richard Helm、 Ralph Johnson 和 John Vlis sides 撰写的开创性著作《设计模式:可重用面向对象软件的元素》,该书通常被称为"四人帮"一书。

此外,为了跟上最新发展和新兴最佳实践,请考虑使用 Fedor G. Pikus 编写的《动手实践 C ++ 设计模式:使用现代设计模式解决常见 C++ 问题并构建强大的应用程序》和 Anthony W illiams 编写的 C++ 并发实践等资源。这些作品将为您提供更广阔的视角,让您更深入地了解设计模式在制作高质量 C++ 软件中所发挥的强大作用。

请记住,重构和应用设计模式的目标不仅仅是编写可以运行的代码,而且是从长远来看编写干净、易于理解、易于修改和易于维护的代码。

在下一章中,我们将深入探讨 C++ 的世界,特别关注命名约定、它们在编写干净且可维护的代码中的重要性以及社区建立的最佳实践。