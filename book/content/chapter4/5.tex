本章中,我们探讨了一些有助于重构遗留 C++ 代码的关键设计模式,包括策略模式、模板方法模式和观察者模式。如果明智地应用这些模式,可以显著改善代码结构,使其更灵活、更易于维护、更能适应变化。

虽然提供了实际的现实世界示例来说明这些模式的使用,但这绝不是详尽的介绍。设计模式是一个庞大而深奥的主题,还有更多的模式和变体可供探索。为了更全面地了解设计模式,我强烈建议您深入研究由 Erich Gamma、 Richard Helm、 Ralph Johnson 和 John Vlissides 撰写的开创性著作《设计模式:可重用面向对象软件的元素》。

此外,为了跟上最新发展和新兴最佳实践,请考虑使用 FedorG. Pikus 编写的《动手实践 C ++ 设计模式:使用现代设计模式解决常见 C++ 问题并构建强大的应用程序》和 Anthony Williams 编写的 《C++ 并发实践》等资源。这些作品将为您提供更广阔的视角,能更深入地了解设计模式在制作高质量 C++ 软件中所发挥的强大作用。

请记住,重构和应用设计模式的目标不仅仅是编写可以运行的代码,而且是从长远来看编写干净、易于理解、易于修改和易于维护的代码。

下一章中,将深入探讨 C++ 的世界,特别关注命名约定在编写干净且可维护的代码中的重要性,以及社区建立的最佳实践。
