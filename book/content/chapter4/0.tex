重构是软件开发中的一项关键技术,涉及对现有代码进行更改以改善其结构、可读性和可维护性,而不会改变其行为。

重构有助于消除技术债务并提高代码库的整体质量。开发人员可以通过删除冗余或重复的代码、简化复杂的代码,以及提高代码的可读性来实现这一点,从而产生更易于维护和更强大的软件。

重构有利于未来的开发。通过重构代码使其更加模块化,开发人员可以更有效地重用现有代码,从而节省未来开发的时间和精力。这使得代码更加灵活,更能适应变化,从而更容易添加新功能、修复错误和优化性能。

结构良好且易于维护的代码,可让多名开发人员更轻松地在项目上进行有效协作。重构有助于标准化代码实践、降低复杂性并改进文档,让开发人员更轻松地理解代码库并为其做出贡献。

最终,从长远来看,重构可以降低与软件开发相关的成本。通过提高代码质量和可维护性,重构可以帮助减少错误修复、更新和其他维护任务所需的时间和精力。

本章中,将重点介绍如何在 C++ 项目中识别适合重构的代码段。但识别适合重构的代码段可能具有挑战性,尤其是在大型复杂系统中。因此,了解使代码段成为重构理想候选者的因素至关重要。我们将探讨这些因素,并提供在 C++ 中识别适合重构的候选者的指南,还将讨论可用于提高 C++ 代码质量的常见重构技术和工具。
