重构是软件开发中的一项关键技术,它涉及对现有代码进行更改以改善其结构、可读性和可维护性,而不会改变其行为。它之所以至关重要有几个原因。

它有助于消除技术债务并提高代码库的整体质量。开发人员可以通过删除冗余或重复的代码、简化复杂的代码以及提高代码的可读性来实现这一点,从而产生更易于维护和更强大的软件。
重构有利于未来的开发。通过重构代码使其更加模块化,开发人员可以更有效地重用现有代码,从而节省未来开发的时间和精力。这使得代码更加灵活,更能适应变化,从而更容易添加新功能、修复错误和优化性能。

结构良好且易于维护的代码可让多名开发人员更轻松地在项目上进行有效协作。重构有助于标准化代码实践、降低复杂性并改进文档,让开发人员更轻松地理解代码库并为其做出贡献。

最终,从长远来看,重构可以降低与软件开发相关的成本。通过提高代码质量和可维护性,重构可以帮助减少错误修复、更新和其他维护任务所需的时间和精力。

在本章中,我们将重点介绍如何在 C++ 项目中识别适合重构的代码段。但是,识别适合重构的代码段可能具有挑战性,尤其是在大型复杂系统中。因此,了解使代码段成为重构理想候选者的因素至关重要。在本章中,我们将探讨这些因素并提供在 C++ 中识别适合重构的候选者的指南。我们还将讨论可用于提高 C++ 代码质量的常见重构技术和工具。