重构遗留的 C++ 代码是一项重大任务,它有可能为老化的代码库注入新的活力。遗留代码通常使用 C++ 的旧方言编写,例如 C++98 或 C++03,这些没有利用 C++11、 C++14、 C++17 和 C++20 中引入的新语言功能和标准库改进。

现代化的一个常见领域是内存管理。旧版 C++ 代码通常使用原始指针来管理动态内存,从而导致内存泄漏和空指针取消引用的潜在问题。此类代码可以重构为使用智能指针,例如 std::unique\_ptr 和 std::shared\_ptr会自动管理指向的对象的生命周期,从而降低内存泄漏的风险。

另一个现代化机会在于采用 C++11 中引入的基于范围的 for 循环。带有显式迭代器或索引变量的旧循环可以用更简洁、更直观的基于范围的循环替换。这不仅使代码更易于阅读,而且还降低了出现 off-by-one 和迭代器无效错误的可能性。

如前几章所述,旧版 C++ 代码库通常大量使用原始数组和 C 风格字符串。此类代码可以重构为使用 std::array、 std::vector 和 std::string,其更安全、更灵活,并提供有用的成员函数。

最后,现代 C++ 在改进并发支持方面取得了重大进展,在 C++11 中引入了 std::thread、 std::async 和 std::future,随后在后续标准中进一步增强。使用平台特定线程或较旧并发库的旧代码,可以通过重构来使用这些现代的可移植并发工具。

从使用 pthread 创建新线程的旧代码示例开始。此线程将执行一个简单的计算:

\begin{cpp}
#include <pthread.h>
#include <iostream>

void* calculate(void* arg) {
    int* result = new int(0);
    for (int i = 0; i < 10000; ++i)
        *result += i;
    pthread_exit(result);
}

int main() {
    pthread_t thread;
    if (pthread_create(&thread, nullptr, calculate, nullptr)) {
        std::cerr << "Error creating thread\n";
        return 1;
    }

    int* result = nullptr;
    if (pthread_join(thread, (void**)&result)) {
        std::cerr << "Error joining thread\n";
        return 2;
    }

    std::cout << "Result: " << *result << '\n';
    delete result;

    return 0;
}
\end{cpp}

现在,可以使用C++11中的std::async来重构这段代码:

\begin{cpp}
#include <future>
#include <iostream>

int calculate() {
    int result = 0;
    for (int i = 0; i < 10000; ++i)
        result += i;
    return result;
}

int main() {
    std::future<int> future = std::async(std::launch::async,
    calculate);
    try {
        int result = future.get();
        std::cout << "Result: " << result << '\n';
    } catch (const std::exception& e) {
        std::cerr << "Error: " << e.what() << '\n';
        return 1;
    }
    return 0;
}
\end{cpp}

重构版本中,使用 std::async 启动新任务,使用 std::future::get 获取结果。 calculate 函数直接将结果作为 int 返回,这比在 pthread 版本中分配内存更简单、更安全。有几点需要注意。对 std::future::get 的调用会阻止执行,直到异步完成。此外,该示例使用 std::launch::async,这确保任务在单独的线程中启动。 C++11 标准允许实现决定默认策略是什么:单独的线程或延迟执行。在撰写本文时, Microsoft Visual C++、 GCC 和 Clang 默认在单独的线程中运行任务。唯一的区别是,虽然 GCC 和 Clang 为每个任务创建一个新线程,但 Microsoft Visual C++ 会重用内部线程池中的线程。错误处理也更简单,计算函数抛出的异常都将被 std::future::get 捕获。

遗留代码通常会使用面向对象的包装器,来包装 pthread 和其他特定于平台的 API。用标准 C++ 实现替换它们可以减少开发人员必须支持的代码量,并使代码更具可移植性。但多线程是一个复杂的主题,因此如果现有代码具有一些丰富的线程相关逻辑,则确保其保持完整非常重要。

现代 C++ 提供的内置算法可以提高旧代码的可读性和维护性。开发人员通常需要检查数组是否包含某个值。 C++11 之前的语言允许执行如下操作:

\begin{cpp}
#include <vector>
#include <iostream>

int main() {
    std::vector<int> numbers = {1, 2, 3, 4, 5, 6};

    bool has_even = false;
    for (size_t i = 0; i < numbers.size(); ++i) {
        if (numbers[i] % 2 == 0) {
            has_even = true;
            break;
        }
    }

    if (has_even)
        std::cout << "The vector contains an even number.\n";
    else
        std::cout << "The vector does not contain any even numbers.\n";

    return 0;
}
\end{cpp}

C++11可以使用 std::any\_of,这是一种新算法,用于检查范围内的任何元素是否满足谓词。能够更简洁、更富有表现力地编写代码:

\begin{cpp}
#include <vector>
#include <algorithm>
#include <iostream>

int main() {
    std::vector<int> numbers = {1, 2, 3, 4, 5, 6};

    bool has_even = std::any_of(numbers.begin(), numbers.end(),
                                [](int n) { return n % 2 == 0; });

    if (has_even)
        std::cout << "The vector contains an even number.\n";
    else
        std::cout << "The vector does not contain any even numbers.\n";

    return 0;
}
\end{cpp}

这个重构版本中,使用 lambda 函数作为 std::any\_of 的谓词。这使得代码更简洁,意图更清晰。诸如 std::all\_of和 std::none\_of之类的算法可以清楚地表达类似的检查。请记住,重构应逐步进行,对每个更改进行彻底测试,以确保不会引入新的错误或回归。这可能是一个耗时的过程,但在提高代码质量、可维护性和性能方面的好处是巨大的。











