
与设计模式相反,反模式是解决问题的常见方法,但从长远来看,这些方法会适得其反或造成危害。识别和避免反模式对于解决代码问题至关重要,应用反模式可能会加剧现有问题并引入新问题。反模式的一些示例包括单例、上帝对象、复制粘贴编程、过早优化和意大利面条式代码。

众所周知,单例违反了依赖倒置和开放/封闭原则。它会创建一个全局实例,这会导致类之间隐藏依赖关系,并使代码难以理解和维护。违反了依赖倒置原则,鼓励高级模块依赖于底层模块,而不是依赖于抽象。此外,单例模式通常很难用不同的实现替换单例实例,例如在扩展类或测试期间。这违反了开放/封闭原则,需要修改代码才能更改或扩展行为。在下面的代码示例中,我们有一个单例类 Database,由 OrderManager 类使用:

\begin{cpp}
class Database {
    public:
    static Database& get_instance() {
        static Database instance;
        return instance;
    }

    template<typename T>
    std::optional<T> get(const Id& id) const;

    template<typename T>
    void save(const T& data);

private:
    Database() {} // Private constructor
    Database(const Database&) = delete; // Delete copy constructor
    Database& operator=(const Database&) = delete; // Delete copy assignment operator
};

class OrderManager {
public:
    void addOrder(const Order& order) {
        auto db = Database::get_instance();
        // check order validity
        // notify other components about the new order, etc
        db.save(order);
    }
};
\end{cpp}

将数据库连接表示为单例的想法非常合乎逻辑:应用程序允许每个应用程序实例只有一个数据库连接,并且数据库在代码中随处可用。使用单例隐藏了 OrderManager 依赖于数据库的事实,这使得代码不那么明显和可预测。使用单例几乎不可能在不运行数据库的真实实例的情况下,通过单元测试来测试 OrderManager 的业务逻辑。

可以通过在主函数的开头某处创建数据库实例,并将其传递给所有需要数据库连接的类来解决该问题:

\begin{cpp}
class OrderManager {
    public:
    OrderManager(Database& db);
    // the rest of the code is the same
};

int main() {
    auto db = Database{};
    auto order_manager = OrderManager{db};
}
\end{cpp}

尽管 Database 不再是单例(即构造函数是公共的),但仍然无法复制。从技术上讲,这允许开发人员临时创建新实例,这不是理想的行为。根据我的经验,可以通过团队内部的知识共享和代码审查,轻松避免这种情况。那些认为这还不够的开发人员可以保持 Database 不变,但确保 get\_instance 仅调用一次并通过引用传递:

\begin{cpp}
int main() {
    auto db = Database::get_instance();
    auto order_manager = OrderManager{db};
}
\end{cpp}

如果代码异味涉及一个具有太多职责的类,则应用上帝对象反模式是不合适的,只会使类更加复杂和难以维护。一般来说,上帝类违反了单一职责原则。让我们看看以下类, EcommerceSystem:

\begin{cpp}
class ECommerceSystem {
public:
    // Product management
    void add_product(int id, const std::string& name, uint64_t price)
    {
        products_[id] = {name, price};
    }

    void remove_product(int id) {
        products_.erase(id);
    }

    void update_product(int id, const std::string& name, uint64_t
    price) {
        products_[id] = {name, price};
    }

    void list_products() {
        // print the list of products
    }

    // Cart management
    void add_to_cart(int product_id, int quantity) {
        cart_[product_id] += quantity;
    }

    void remove_from_cart(int product_id) {
        cart_.erase(product_id);
    }

    void update_cart(int product_id, int quantity) {
        cart_[product_id] = quantity;
    }

    uint64_t calculate_cart_total() {
        uint64_t total = 0;
        for (const auto& item : cart_) {
            total += products_[item.first].second * item.second;
        }
        return total;
    }

    // Order management
    void place_order() {

        // Process payment, update inventory, send confirmation email,
        etc.
        // ...
        cart_.clear();
    }

    // Persistence
    void save_to_file(const std::string& file_name) {
        // serializing the state to a file
    }

    void load_from_file(const std::string& file_name) {
        // loading a file and parsing it
    }

private:
    std::map<int, std::pair<std::string, uint64_t>> products_;
    std::map<int, int> cart_;
};
\end{cpp}

本例中,ECommerceSystem类承担多种职责,例如:产品管理、购物车管理、订单管理和持久性(保存和加载文件中的数据)。这个类很难维护、理解和修改。

更好的方法是将电子商务系统分解为更小、更集中的类,每个类处理特定的职责:

\begin{itemize}
\item
ProductManager 类管理产品

\item
CartManager 类管理购物车

\item
OrderManager 类管理订单和相关任务(例如:处理付款和发送确认电子邮件)

\item
PersistenceManager 类负责保存和加载文件中的数据
\end{itemize}

这些类可以按如下方式实现:

\begin{cpp}
class ProductManager {
public:
    void add_product(int id, const std::string& name, uint64_t price)
    {
        products_[id] = {name, price};
    }

    void remove_product(int id) {
        products_.erase(id);
    }

    void update_product(int id, const std::string& name, uint64_t price) {
        products_[id] = {name, price};
    }

    std::pair<std::string, uint64_t> get_product(int id) {
        return products_[id];
    }

    void list_products() {
        // print the list of products
    }

private:
    std::map<int, std::pair<std::string, uint64_t>> products_;
};

class CartManager {
public:
    void add_to_cart(int product_id, int quantity) {
        cart_[product_id] += quantity;
    }

    void remove_from_cart(int product_id) {
        cart_.erase(product_id);
    }

    void update_cart(int product_id, int quantity) {
        cart_[product_id] = quantity;
    }

    std::map<int, int> get_cart_contents() {
        return cart_;
    }

    void clear_cart() {
        cart_.clear();
    }

private:
    std::map<int, int> cart_;
};

class OrderManager {
public:
    OrderManager(ProductManager& product_manager, CartManager& cart_manager)
    : product_manager_(product_manager), cart_manager_(cart_manager) {}
        uint64_t calculate_cart_total() {
        // calculate cart's total the same as before
    }

    void place_order() {
        // Process payment, update inventory, send confirmation email,
        etc.
        // ...
        cart_manager_.clear_cart();
    }

private:
    ProductManager& product_manager_;
    CartManager& cart_manager_;
};

class PersistenceManager {
public:
    PersistenceManager(ProductManager& product_manager)
        : product_manager_(product_manager) {}
    void save_to_file(const std::string& file_name) {
        // saving
    }

    void load_from_file(const std::string& file_name) {
        // loading
    }

private:
    ProductManager& product_manager_;
};
\end{cpp}

最终,拥有新类并为其功能提供代理方法的 ECommerce 类:

\begin{cpp}
// include the new classes
class ECommerce {
public:
    void add_product(int id, const std::string& name, uint64_t price)
    {
        product_manager_.add_product(id, name, price);
    }

    void remove_product(int id) {
        product_manager_.remove_product(id);
    }

    void update_product(int id, const std::string& name, uint64_t price) {
        product_manager_.update_product(id, name, price);
    }

    void list_products() {
        product_manager_.list_products();
    }

    void add_to_cart(int product_id, int quantity) {
        cart_manager_.add_to_cart(product_id, quantity);
    }

    void remove_from_cart(int product_id) {
        cart_manager_.remove_from_cart(product_id);
    }

    void update_cart(int product_id, int quantity) {
        cart_manager_.update_cart(product_id, quantity);
    }

    uint64_t calculate_cart_total() {
        return order_manager_.calculate_cart_total();
    }

    void place_order() {
        order_manager_.place_order();
    }

    void save_to_file(const std::string& filename) {
        persistence_manager_.save_to_file(filename);
    }

    void load_from_file(const std::string& filename) {
        persistence_manager_.load_from_file(filename);
    }
private:
    ProductManager product_manager_;
    CartManager cart_manager_;
    OrderManager order_manager_{product_manager_, cart_manager_};
    PersistenceManager persistence_manager_{product_manager_};
};

int main() {
    ECommerce e_commerce;

    e_commerce.add_product(1, "Laptop", 999.99);
    e_commerce.add_product(2, "Smartphone", 699.99);
    e_commerce.add_product(3, "Headphones", 99.99);

    e_commerce.list_products();

    e_commerce.add_to_cart(1, 1); // Add 1 Laptop to the cart

    e_commerce.add_to_cart(3, 2); // Add 2 Headphones to the cart

    uint64_t cart_total = e_commerce.calculate_cart_total();
    std::cout << "Cart Total: $" << cart_total << std::endl;

    e_commerce.place_order();
    std::cout << "Order placed successfully!" << std::endl;

    e_commerce.save_to_file("products.txt");
    e_commerce.remove_product(1);
    e_commerce.remove_product(2);
    e_commerce.remove_product(3);

    std::cout << "Loading products from file..." << std::endl;
    e_commerce.load_from_file("products.txt");
    e_commerce.list_products();

    return 0;
}
\end{cpp}

通过将职责划分到多个较小的类中,代码变得更加模块化、更易于维护,并且更适合实际应用。其中一个子类的内部业务逻辑的微小变化不会导致 ECommerce 类的更新。在 C++ 中,由于编译时间问题严重,这可能更为重要。单独测试这些类或完全替换其中一个类的实现会更容易,例如:将数据保存到远程存储,而非磁盘。

\mySubsubsection{4.3.1}{魔数陷阱 --- 数据分块案例研究}

考虑以下 C++ 函数 send,该函数旨在将数据块分块发送到某个目的地。该函数如下所示:

\begin{cpp}
#include <cstddef>
#include <algorithm>

// Actually sending the data
void do_send(const std::uint8_t* data, size_t size);

void send(const std::uint8_t* data, size_t size) {
    for (std::size_t position = 0; position < size;) {
        std::size_t length = std::min(size_t{256}, size - position);
        // 256 is a magic number
        do_send(data + position, position + length);
        position += length;
    }
}
\end{cpp}

\mySamllsectionNoContent{代码起什么作用?}

send 函数接受一个指向 std::uint8\_t 数组 (data) 的指针及其大小 (size)。然后,将这些数据分块发送到 do\_send 函数,该函数负责实际的发送过程。每个块的最大大小为 256 字节,如send 函数中定义的那样。

\mySamllsectionNoContent{为什么魔法数字有问题?}

数字 256 直接嵌入到代码中,并且没有解释它代表什么。这是一个典型的魔法数字示例。阅读此代码的人都会去猜测为什么选择 256。这是硬件限制?协议约束?性能调整参数?

\mySamllsectionNoContent{constexpr 解决方案}

提高此代码清晰度的一种方法是用命名的 constexpr 变量替换魔法数字:

\begin{cpp}
#include <cstddef>
#include <algorithm>

constexpr std::size_t MAX_DATA_TO_SEND = 256; // Named constant replaces magic number

// Actually sending the data
void do_send(const std::uint8_t* data, size_t size);

void send(const std::uint8_t* data, size_t size) {
    for (std::size_t position = 0; position < size;) {
        std::size_t length = std::min(MAX_DATA_TO_SEND, size - position); // Use the named constant
        do_send(data + position, position + length);
        position += length;
    }
}
\end{cpp}

\mySamllsectionNoContent{使用 constexpr 的优点}

将魔数替换为 MAX\_DATA\_TO\_SEND 可以更容易理解为什么存在此限制。此外,如果有另一个函数(例如 read),也需要以 256 字节的块读取数据,则使用 constexpr 变量可确保一致性。如果需要更改块大小,只需在一个地方更新它,从而降低出现错误和不一致的风险。

处理代码异味时,了解代码异味的根本原因,并应用正确的模式或避免反模式,来有效地重构代码至关重要。如果代码异味涉及重复代码,则应避免复制粘贴编程,而应应用模板方法或策略模式等模式来促进代码重用并减少重复。同样,如果代码异味涉及紧密耦合的模块或类,则应应用适配器或依赖倒置原则等模式来减少耦合并提高模块化。

重要的是要记住,重构有异味的代码应该是一个迭代和增量的过程。开发人员应该不断审查和评估他们的代码库中的异味,进行小的、有针对性的更改,逐步提高代码的质量和可维护性。这种方法可以更好地管理风险管理,其可最大限度地减少了在重构过程中引入新错误或问题的可能性。实现这一点的最佳方法是单元测试,有助于验证重构后的代码是否仍然满足其原始要求并按预期运行,即使在修改其内部结构或组织之后也是如此。

通过在开始重构过程之前进行一组强大的测试,开发人员可以确信他们的更改不会对应用程序的行为产生负面影响。这使他们能够专注于改进代码的设计、可读性和可维护性,而不必担心无意中破坏功能。我们将在第 13 章中探讨单元测试。

总之,代码异味是一个术语,描述了代码库中可能存在设计或实现问题的症状。解决代码异味需要识别和应用适当的设计模式,以及避免可能损害代码质量的反模式。通过了解代码异味的根本原因并有效地使用模式和反模式,开发人员可以重构他们的代码库,使其更易于维护、更易读、更能适应未来的变化。持续评估和增量重构,是防止代码异味和确保高质量、高效的代码库能够适应不断变化的需求的关键。














