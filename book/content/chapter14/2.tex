
整个项目中提交消息和结构的一致性可提高可读性和可预测性,使团队成员更容易浏览项目历史记录。遵守预定义的格式或一组约定(例如常规提交规范),可确保提交消息具有统一的结构和信息量。这些可能包括以祈使语气的动词开头提交消息,指定更改类型(例如修复、功能或重构),以及可选地包含范围以阐明项目的哪一部分受到影响。

\mySubsubsection{14.2.1}{将代码链接到上下文}

良好的提交可将代码更改与其更广泛的上下文(例如项目问题跟踪系统中的票据或相关文档)联系起来,从而提高可追溯性。在提交消息中包含引用可在技术实现,与其解决的要求或问题之间建立切实的联系,从而有助于更好地了解和跟踪项目进度。

在提交消息中加入问题跟踪器 ID、票号或其他相关标识符可以显著提高更改的可追溯性,并使其与项目目标或报告的问题更容易关联。通常看起来类似于修复(FP-1234):更正了用户身份验证流程,其中 FP-1234 是问题跟踪系统中的票号。

从本质上讲,良好的提交是项目开发历史中一个连贯、独立的故事。通过遵守这些原则,开发人员不仅可以提高代码库的可维护性和可读性,还可以在版本控制实践中培养一种一丝不苟、负责任的文化。通过有纪律地创建良好的提交,项目历史成为协作、审查和了解软件演变的宝贵资产。

创建良好提交消息的最佳方法之一是遵循常规提交规范。常规提交规范是提交消息格式的结构化框架,其设计有两个目的:简化创建可读提交日志的过程,并启用自动化工具以促进版本管理和发布说明生成。此规范描述了提交消息的标准化格式,旨在清楚地传达版本控制系统(如 Git)中更改的性质和意图。

\mySubsubsection{14.2.2}{概述和目的}

其核心中,常规提交规范规定了一种格式,其中包括类型、可选范围和简洁的描述。格式通常遵循以下结构:

\begin{shell}
<type>[optional scope]: <description>

[optional body]

[optional footer(s)]
\end{shell}

类型根据提交引入的更改的性质对其进行分类,例如 feat 表示新功能, fix 表示错误修复。范围虽然是可选的,但它提供了上下文信息,通常表明受更改影响的代码库部分。

\mySubsubsection{14.2.3}{选项和用法}

提交 linting(尤其是在遵守常规提交规范的情况下)可确保提交的结构清晰、可预测且有用。以下是一些符合提交 linting 规则的提交示例,展示了软件项目中可能发生的各种类型的更改:

\begin{itemize}
\item
添加新功能:

\begin{shell}
feat(authentication): add biometric authentication support
\end{shell}

该提交消息表明已添加一项新功能(feat)(具体来说,就是生物特征认证支持),并且该功能的范围在应用程序的认证模块内。

\item
修复错误:

\begin{shell}
feat(authentication): add biometric authentication support
\end{shell}

这里正在提交一个错误修复(fix),解决数据库模块内的竞争条件问题,特别是在用户数据检索过程中。

\item
改进文档:

\begin{shell}
docs(readme): update installation instructions
\end{shell}

此示例显示了文档更新(docs),其中对项目的 README 文件所做的更改以更新安装说明。

\item
代码重构:

\begin{shell}
refactor(ui): simplify button component logic
\end{shell}

在此提交中,现有代码已重构(重构),但未添加新功能或修复错误。重构的范围是 UI,具体简化了按钮组件中使用的逻辑。

\item
风格调整:

\begin{shell}
style(css): remove unused CSS classes
\end{shell}

此提交消息表示样式更改(样式),其中未使用的 CSS 类将被删除。值得注意的是,这种类型的提交不会影响代码的功能。

\item
添加测试:

\begin{shell}
test(api): add tests for new user endpoint
\end{shell}

这里,为API中的新用户端点添加了新的测试(test),表明项目的测试覆盖率得到了增强。

\item
杂项任务:

\begin{shell}
chore(build): update build script for deployment
\end{shell}

这个提交代表一项杂项任务(chore),通常是一项维护或设置任务,不会直接修改源代码或添加功能,例如更新用于部署的构建脚本。

\item
重大变化:

\begin{shell}
feat(database): change database schema for users table

BREAKING CHANGE: The database schema modification requires resetting the database. This change will affect all services interacting with the users table.
\end{shell}

另一种表示重大更改的方法是,在提交消息中的类型和范围之后但在冒号之前添加感叹号 (!)。这种方法简洁明了,而且视觉上很明显:

\begin{shell}
feat!(api): overhaul authentication system
\end{shell}

这次提交引入了与用户表的数据库模式相关的新功能(feat),但也包括一项重大变化。

\end{itemize}

这些示例说明了在传统提交规范的指导下,提交 审查 如何促进清晰、结构化和信息丰富的提交消息,从而增强项目可维护性和协作。

\mySubsubsection{14.2.4}{起源与采用}

传统提交规范的灵感来自于,简化可读且自动化的变更日志创建过程的需求。它以 AngularJS 团队的早期实践为基础,此后已被各种开源和企业项目采用,旨在标准化提交消息传递,从而提高项目的可维护性和协作性。

\mySubsubsection{14.2.5}{常规提交的优点}

遵守传统提交规范有很多好处:

\begin{itemize}
\item
自动化处理:通过对提交进行分类,工具可以根据更改的语义含义自动确定版本变化,并遵守语义版本控制 (Semver) 原则

\item
简化说明:自动化工具可以通过解析结构化的提交消息来生成全面、清晰的发布说明和变更日志,从而大大减少手动工作量并增强发布文档

\item
增强可读性:标准化格式提高了提交历史的可读性,使开发人员更容易导航和了解项目演变

\item
便于审查:清晰的变更分类和描述有助于代码审查过程,使审查人员能够快速掌握变更的范围和意图
\end{itemize}















































