
有效版本控制实践的核心是"良好提交"的概念,这是一个基本的变更单位,体现了代码库中的清晰度、原子性和目的性原则。对于努力保持干净、可导航且信息丰富的项目历史记录的开发人员来说,了解什么是良好提交至关重要。本节深入探讨定义提交质量的关键属性,并提供有关开发人员如何增强其版本控制实践的见解。

\mySubsubsection{14.1.1}{单一焦点原则}

良好的提交遵循原子性原则,这意味着它将代码库中的单个逻辑更改封装起来。这种单一的关注点可确保每个提交都具有独立意义,并且可以通过还原或调整单个提交来安全轻松地还原或修改项目。原子提交简化了代码审查流程,使团队成员更容易理解和评估每个更改,而不会受到无关修改的干扰。例如,不应将新功能实现与单独的错误修复合并在一次提交中,而应将它们拆分为两个不同的提交,每个提交都有明确的目的和范围。

\mySubsubsection{14.1.2}{沟通的艺术}

良好提交的本质还在于其清晰度,这在提交消息中尤为明显。清晰的提交消息简洁地描述了更改的内容和原因,可作为简明的文档供将来参考。这种清晰度不仅限于直接团队,还可以帮助与代码库交互的任何人,包括新团队成员、外部合作者,甚至未来的自己。在长时间后重新访问代码库时,这一点尤为重要,因为提交消息是项目发展的历史记录。这种方法对于开源项目至关重要,因为它允许贡献者了解更改背后的背景和理由,从而促进协作和包容的环境。
结构良好的提交消息通常包含简洁的标题行来总结更改,然后是空白行和更详细的解释(如果需要)。解释可以深入探讨更改背后的原因、可能产生的影响以及任何有助于理解提交目的的其他上下文。建议将主题行保持在 50 个字符以内。这可确保消息适合大多数终端的标准宽度,不会被 GitHub 或其他平台终止,并且易于扫描。 GitHub 会截断短于 72 个字符的主题,因此 72 个字符是硬限制, 50 个字符是软限制。例如,提交消息

\begin{shell}
feat: Added a lot of need include directives to make things compileappropriate
\end{shell}

将被 GitHub 截断,如下所示:

\myGraphic{0.8}{content/chapter14/images/1.png}{图 14.1 – 截断的提交消息}

GitHub 正确地截断了最后一个单词,开发人员必须点击提交消息才能阅读它。这不是什么大问题,但这是一个小小的不便,只要保持主题行简短,就可以轻松避免。

更重要的是,它迫使作者简明扼要。

其他有用的做法包括在主题行中使用祈使语气,这是提交消息中的常见惯例。这意味着主题行应该以命令或指令的形式表达,例如修复错误或添加功能。这种写作风格更直接,符合提交代表正在应用于代码库的更改的想法。

最好不要以句号结束主题行,因为句号不是一个完整的句子,而且不利于保持消息简短。
提交消息的正文可以提供其他背景信息,例如更改的动机、解决的问题以及有关实施的任何相关详细信息。

最好将正文换行到 72 个字符,因为 Git 不会自动换行。这是一个常见的惯例,它使消息在各种环境中(例如终端窗口、文本编辑器和版本控制工具)更具可读性。通过配置代码编辑器可以轻松实现这一点。

因此,在最终确定提交之前花在反思上的时间不仅是为了确保消息的清晰度;它还在于重申变更本身背后的价值和意图。这是一个机会,可以确保对代码库的每次贡献都是经过深思熟虑、有意义的,并且与项目目标保持一致。从这个角度来看,花时间编写精确且内容翔实的提交消息不仅是一种良好做法,而且证明了开发人员对质量和协作的承诺。

\mySubsubsection{14.1.3}{精致的艺术}

在将功能分支合并到主分支之前,开发人员应谨慎考虑其提交历史的整洁性和清晰度。压缩中间提交是一种深思熟虑的做法,可以简化提交日志,使其对任何探索项目历史的人来说都更具可读性和意义。

当您即将整合工作时,请花点时间回顾一下开发过程中积累的提交消息。问问自己,每条提交消息是否有助于了解项目的演变,还是只会用冗余或过于详细的细节来混淆历史记录。在许多情况下,您为获得最终解决方案而采取的迭代步骤(例如小错误修复、响应代码审查的调整或对单元测试的更正)可能对其他贡献者或未来的您没有重大价值。

考虑一个提交历史记录,其中包含诸如修复错误、修复单元测试或多个修复 cr 条目之类的消息。此类消息虽然表明了开发过程,但不一定能提供有关更改或它们对项目的影响的有意义的见解。将这些中间提交压缩为一个精心设计的提交不仅可以整理提交日志,还可以确保历史记录中的每个条目都传达项目开发中的重要一步。

通过压缩提交,您可以将这些迭代更改整合为一个连贯的叙述,重点介绍新功能的引入、 重大错误的解决或关键重构的实施。此精选历史记录可帮助当前贡献者和未来维护者浏览和了解项目的进展,从而增强协作和效率。

总之,在合并之前,请从更广阔的视角考虑项目的提交历史。压缩中间提交是一种正念练习,可确保提交日志仍然是所有贡献者的宝贵且可导航的资源,以清晰简洁的方式概括每次更改的本质。














